\documentclass[12pt]{article}
\usepackage[spanish]{babel}
\usepackage{geometry}
\usepackage{amsmath}
\usepackage{authblk} % Paquete para manejar autores y afiliaciones en 'article'
\usepackage{hyperref}
\usepackage{fancyhdr} % Paquete para encabezados y pies de página
\usepackage{colortbl} % Paquete para colorear celdas en tablas
\usepackage{multirow} % Paquete para combinar filas en tablas
\usepackage{booktabs} % Paquete para mejorar el aspecto de las tablas
\usepackage{graphicx}

% Quitar la sangría en todo el documento
\setlength{\parindent}{0pt}

\geometry{letterpaper, margin=1in}

\title{\Large\textbf{Caracterización de las Propiedades Morfológicas, Ópticas y Térmicas de la Regolita en la Región de los Cráteres Garavito}}
\author[1]{\raggedright Eduardo A. Delgadillo Monsalve}
\author[2]{\raggedright PhD Mario Armando Higuera Garzón (Director)}
\author[3]{\raggedright PhD David Ardila (Codirector)}
\affil[1]{\raggedright\small MSc estudiante - Astronomía (eadelgadillom@unal.edu.co) \newline \small Observatorio Astronómico Nacional, Universidad Nacional de Colombia}
\affil[2]{\raggedright\small Observatorio Astronómico Nacional, Universidad Nacional de Colombia}
\affil[3]{\raggedright\small Jet Propulsion Laboratory, NASA}

\date{}

\pagestyle{fancy}
\fancyhf{} % Limpia los encabezados y pies
\fancyhead[L]{Caracterización de la Regolita Lunar}
\fancyhead[R]{\today}

\begin{document}

\maketitle

\section*{Resumen de Investigación}
Este trabajo tiene como objetivo general caracterizar las propiedades morfológicas, ópticas y térmicas de la regolita presente en la región de los cráteres Garavito, situados en la cara oculta de la Luna. La investigación incluye los siguientes objetivos específicos:
\begin{itemize}
    \item Determinar los parámetros que definen la temperatura de la regolita en la región de los cráteres Garavito.
    \item Reproducir las condiciones de radiancia solar observadas en esta región durante un día lunar.
    \item Establecer las curvas de emisión en las componentes ópticas y térmicas a lo largo de un día lunar.
\end{itemize}

\section*{Motivación}
El interés global en la exploración lunar, especialmente en su lado oculto, ha crecido considerablemente debido a la posibilidad de establecer futuras bases y colonias lunares. Con la creciente disponibilidad de datos de misiones de exploración, se pueden realizar estudios detallados sobre la topografía, composición y condiciones de iluminación en regiones como los cráteres Garavito. Este estudio aporta al conocimiento sobre la composición y comportamiento térmico de la regolita lunar, información clave para futuras misiones.

\section*{Introducción y Planteamiento del Problema}
La observación de la Luna ha fascinado a la humanidad desde tiempos antiguos. En particular, la región de los cráteres Garavito —nombrada en honor al astrónomo colombiano Julio Garavito— presenta características topográficas únicas, y se encuentra en el lado oculto de la Luna, que es significativamente más accidentado y menos explorado que el lado visible. Las condiciones de iluminación y sombras en esta región influyen en la reflectancia, la emisión térmica y las variaciones de temperatura de la regolita, elementos clave en este trabajo de caracterización.

\end{document}
