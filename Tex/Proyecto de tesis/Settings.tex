%All the packages the project needs
%--------------------------------------------
%Margins
\usepackage[top=2cm,
bottom=2cm,
left=1.6cm,
right=1.6cm,
headsep=15pt,
headheight=25pt,
letterpaper,
includehead,includefoot]{geometry} 
%-------------------------------------------

%-------------------------------------------
%Codification and language
%Document language
\usepackage[spanish,es-noshorthands]{babel}
% Required for including letters with accents
\usepackage[utf8]{inputenc} 
% Use 8-bit encoding that has 256 glyphs
\usepackage[T1]{fontenc}
%-------------------------------------------

%-------------------------------------------
%Colours
\usepackage{color}
%\usepackage{xcolor}


%-------------------------------------------

%-------------------------------------------
%Fonts and text
%Add a new Font
\usepackage{addfont}
\addfont{OT1}{rsfs10}{\rsfs}
%Allows to modify the font of a section
\usepackage{sectsty}
%Latin Modern fonts - Allows to resize the font
\usepackage{lmodern}
%Allows to write code
\usepackage{verbatim}
%Dumb text
\usepackage{lipsum}
%Include currency symbols
\usepackage{textcomp} 
%-------------------------------------------

%-------------------------------------------
%Pictures and graphics
%Allows to include images
\usepackage{graphicx}
\usepackage{sidecap}
%Most powerful image package
\usepackage{tikz}
\usetikzlibrary{babel}
%Sometimes used images packages
\usepackage{pst-all}
\usepackage{pstricks}
%Allows to position figures inside a paragraph
\usepackage{wrapfig}
%Information needed
%https://en.wikibooks.org/wiki/LaTeX/Floats,_Figures_and_Captions
%-------------------------------------------
%-------------------------------------------
%Math packages
\usepackage{mathtools}
\usepackage{amsmath}
\usepackage{amsfonts}
%Nice cancelations arrows
\usepackage{cancel}
%Vector arrows on a character
\usepackage{esvect}
%-------------------------------------------

%-------------------------------------------
%Table and listing packages
\usepackage{multirow}
\usepackage{array}
\usepackage{diagbox}
\usepackage{listings}
\usepackage[shortlabels]{enumitem} 
%Differents images and tables in the same space
\usepackage{subcaption}
%Use to places the float at precisely the location in the LaTeX code´and other stuff
\usepackage{float}
%Flexible tables
\usepackage{tabu}
\usepackage{multicol}
%-------------------------------------------

%-------------------------------------------
%Refs and cites
\usepackage[hidelinks]{hyperref}
%\usepackage{cite}
%-------------------------------------------

%-------------------------------------------
%Boxes
\usepackage{tcolorbox}
\usepackage{fancybox}
\usepackage{colortbl}
%-------------------------------------------


%-------------------------------------------
%National University of Colombia - Logo
% Definimos el color del logo

\newgray{ungris}{0.6}

% Dimension del logo 

\newdimen\UNlong

% Genera la U (PsPicture)

\newcommand{\UNU}{
 	\pscustom{
 		\psline[linearc=0.45]{C-C}(0,1)(0,0)(1.102,0)
		\psline(1.102,1)(0.92,1)(0.92,0.487)
		\psline[linearc=0.35](0.92,0.152)(0.553,0.152)(0.153,0.152)(0.153,0.487)
		\psline(0.153,1)(0,1)
	\fill[linecolor=ungris,fillstyle=solid,fillcolor=ungris]}}

% Genera la UN

\newcommand{\UNlog}[1]{
	\UNlong=#1
	\psset{unit=1\UNlong,linewidth=0.1pt,linecolor=ungris}
  \begin{pspicture}(-0.1,-0.1)(2.386,1.1)
 		\rput(0,0){\UNU}
		\rput{180}(2.286,1){\UNU}
	\end{pspicture}
    }
% Fin de la definición del logo
%Rehacer el logo de UN en tikz para que compile en pdflatex
%-------------------------------------------
%-------------------------------------------
%Page style
\usepackage[usestackEOL]{stackengine}
\usepackage{fancyhdr}
\pagestyle{fancy}
\newsavebox{\UNl}
\newsavebox{\UNd}
\setlength{\headheight}{2.5cm}
\lhead{\usebox{\UNl}\usebox{\UNd}}
\rhead{\slshape Proyecto de Tesis}
\cfoot{}
\rfoot{\thepage}
%\lfoot{\today}
\renewcommand{\headrulewidth}{1pt}

%-------------------------------------------