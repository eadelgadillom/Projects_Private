\documentclass[12pt]{article}
%All the packages the project needs
%--------------------------------------------
%Margins
\usepackage[top=2cm,
bottom=2cm,
left=1.6cm,
right=1.6cm,
headsep=15pt,
headheight=25pt,
letterpaper,
includehead,includefoot]{geometry} 
%-------------------------------------------

%-------------------------------------------
%Codification and language
%Document language
\usepackage[spanish,es-noshorthands]{babel}
% Required for including letters with accents
\usepackage[utf8]{inputenc} 
% Use 8-bit encoding that has 256 glyphs
\usepackage[T1]{fontenc}
%-------------------------------------------

%-------------------------------------------
%Colours
\usepackage{color}
\usepackage{xcolor}
%-------------------------------------------

%-------------------------------------------
%Fonts and text
%Add a new Font
\usepackage{addfont}
\addfont{OT1}{rsfs10}{\rsfs}
%Allows to modify the font of a section
\usepackage{sectsty}
%Latin Modern fonts - Allows to resize the font
\usepackage{lmodern}
%Allows to write code
\usepackage{verbatim}
%Dumb text
\usepackage{lipsum}
%Include currency symbols
\usepackage{textcomp} 
%-------------------------------------------

%-------------------------------------------
%Pictures and graphics
%Allows to include images
\usepackage{graphicx}
\usepackage{sidecap}
%Most powerful image package
\usepackage{tikz}
\usetikzlibrary{babel}
\usetikzlibrary{positioning}
%Sometimes used images packages

\usepackage{pst-all}
\usepackage{pstricks}
%Allows to position figures inside a paragraph
\usepackage{wrapfig}
%Information needed
%https://en.wikibooks.org/wiki/LaTeX/Floats,_Figures_and_Captions
%-------------------------------------------
%-------------------------------------------
%Math packages
\usepackage{mathtools}
\usepackage{amsmath}
\usepackage{amssymb}
\usepackage{amsfonts}
%Nice cancelations arrows
\usepackage{cancel}
%Vector arrows on a character
\usepackage{esvect}
%-------------------------------------------

%-------------------------------------------
%Table and listing packages
\usepackage{multirow}
\usepackage{array}
\usepackage{diagbox}
\usepackage{listings}
\usepackage[shortlabels]{enumitem} 
%Differents images and tables in the same space
\usepackage{subcaption}
%Use to places the float at precisely the location in the LaTeX code´and other stuff
\usepackage{float}
%Flexible tables
\usepackage{tabu}
\usepackage{multicol}
%-------------------------------------------

%-------------------------------------------
%Refs and cites
\usepackage[hidelinks]{hyperref}
%\usepackage{cite}
%-------------------------------------------

%-------------------------------------------
%Boxes
\usepackage{tcolorbox}
\usepackage{fancybox}
\usepackage{colortbl}
%-------------------------------------------

%-------------------------------------------
%Page style
\usepackage[usestackEOL]{stackengine}
\usepackage{fancyhdr}
\pagestyle{fancy}
\newsavebox{\UNl}
\newsavebox{\UNd}
\setlength{\headheight}{2.5cm}
\lhead{\usebox{\UNl}\usebox{\UNd}}
\rhead{\slshape Proyecto de Tesis}
\cfoot{}
\rfoot{\thepage}
%\lfoot{\today}
\renewcommand{\headrulewidth}{1pt}

%-------------------------------------------

%-------------------------------------------
%others
\usepackage{pgfplots,caption}
\usepackage{comment}
\usepackage{pdflscape}
\usepackage{rotating}
\usepackage{parskip}
\usepackage{hhline}
\usepackage{longtable,lscape}
\usepackage{booktabs,bigstrut}

\addbibresource{ref.bib}
\title{Proyecto de Tesis}
\author{Eduardo Andres Delgadillo Monsalve}
\date{\today}
%Inicio del documento
\pgfplotsset{compat=1.17}

\begin{document}
\renewcommand{\tablename}{Tabla}%% Para cambiar "Cuadro" a "Tabla"
\renewcommand{\listtablename}{Índice de tablas}
%Inicio de la portada
\begin{titlepage}
    \centering
    \thispagestyle{empty}
    \begin{center}
        \begin{figure}
        \centering%
        \includegraphics{images/EscudoUN.png}
    \end{figure}
    
    \vspace{3cm}
    
    \textbf{CARACTERIZACIÓN DE LAS PROPIEDADES DE LA REGOLITA LUNAR MEDIANTE EL DISEÑO E IMPLEMENTACIÓN DE UN MODELO ANÁLOGO MORFOLÓGICO Y UN MODELO ESPECTRAL}\\[1in]    
    Eduardo Andrés Delgadillo Monsalve \\  [3in]

   \textbf{Universidad Nacional de Colombia}\\
   Facultad de Ciencias\\
   Bogotá D.C.\\
   \today
    \end{center}
\end{titlepage}

\newpage
\begin{titlepage}
    \thispagestyle{empty}
    \begin{center}
        \begin{figure}
        \centering%
        \includegraphics{images/EscudoUN.png}
    \end{figure}
    
    \vspace{1cm}
    
        \textbf{CARACTERIZACIÓN DE LAS PROPIEDADES DE LA REGOLITA LUNAR MEDIANTE EL DISEÑO E IMPLEMENTACIÓN DE UN MODELO ANÁLOGO MORFOLÓGICO Y UN MODELO ESPECTRAL}\\[1in]    
    Eduardo Andrés Delgadillo Monsalve \\  [1.5 cm]
 Proyecto de tesis presentado como requisito parcial paara optar al titulo de:\\[3mm] \textbf {\large{Magister en Ciencias Astronomia}}\\[1cm]
 Director:\\[3mm] PhD Mario Armando Higuera Garzon\\[1mm] Observatorio Astronómico Nacional\\[1cm]
 Codirector:\\[3mm] PhD David Ardila\\ Jet Propulsion Laboratory\\[1in]
 \textbf{Universidad Nacional de Colombia  }\\
   Facultad de Ciencias\\
   Bogotá D.C.\\
   \today
    \end{center}
\end{titlepage}

\newpage
\thispagestyle{empty}
\tableofcontents
\thispagestyle{empty}
\newpage

% \listoffigures

% \listoftables

\newpage

\setcounter{page}{1}
\section{Introducción}
La Luna ha sido uno de los objetos mas estudiados y de mayor interés para la humanidad desde épocas antiguas. En la mayoría de civilizaciones y culturas,
de las que se tienen registro, la Luna ha tenido un papel fundamental dentro de su cosmogonía. Este satélite natural se considero por mucho tiempo como 
la contraparte de la estrella que domina el cielo diurno, el Sol. Eventos naturales como los eclipses de Sol, los eclipses de Luna y el ciclo lunar, fueron interpretados
y estudiados a lo largo de toda la historia, incluso uno de los mayores logros como especie humana ha sido el alunizaje de humanos sobre la superficie de la Luna, 
en el año 1969. Desde entonces, el estudio de la Luna se ha incrementado y representa una de las ramas de investigación mas importantes en la búsqueda por habitar su superficie.

Debido al proceso de formación y evolución, la superficie de la Luna esta compuesta principalmente de polvo y regolito. Las diferentes misiones de exploración lunares de las distintas 
agencias espaciales han recopilado muestras que han sido analizadas y caracterizadas. Aunque algunas propiedades de este material se conocen bien, algunas otras y que son muy importantes 
requieren aún de una investigación mas profunda. Debido a la dificultad de obtener dichas muestras de este material, técnicas como la reproducción o la implementación de un modelo análogo 
que replique sus propiedades son necesarias. La caracterización del material de la superficie lunar puede hacerse utilizando su morfología, es decir su composición, 
forma, tamaño, estructura y estado, también utilizando su emisión térmica, que presenta variaciones debido a las diferencias de temperatura que se pueden presentar por su ciclo 
y a su abrupto relieve, y por ultimo utilizando la distribución espectral de energía (SED) del material de la superficie, principalmente los granos de polvo.

El conocimiento completo de las propiedades del material de la superficie de la Luna es de gran importancia sobre todo para posteriores misiones de alunizaje. 
El conjunto de todas estas propiedades definirán las características del lanzamiento y alunizaje de cualquier misión de exploración.
\section{Objetivos}
\subsection{Objetivo General}
Caracterizar las propiedades de la regolita lunar mediante el diseño e  implementación de un modelo análogo morfológico y un modelo espectral.

\subsection{Objetivos específicos}
\begin{itemize}
    \item Estudiar y caracterizar las propiedades de la regolita lunar a traves de un modelo análogo morfologico que permita reproducir las condiciones en la superficie lunar.
    \item Reproducir las condiciones de iluminación que se observan sobre la region que contiene los cinco crateres Garavito.
    %\item Diseñar un modelo análogo morfológico que permita reproducir las condiciones en la superficie lunar %Para la region del crater Garavito Armero
    \item Diseñar un modelo de emisión que permita reproducir las características térmicas y ópticas de la superficie lunar.
\end{itemize}

\section{Antecedentes}
El interés hacia la Luna ha sobre pasado los primeros pensamientos de los humanos mas primitivos, pasado las historias mitológicas y la ciencia ficción y es actualmente una de las propuestas mas 
importantes de investigación de las diferentes agencias espaciales. Las diferentes misiones de exploración lunar que comenzaron desde los años 50 y que actualmente tienen incluso mucha mas relevancia, 
han tenido siempre como objetivo la habitabilidad del hombre sobre la superficie lunar. 

La historia de la Luna como el único satélite natural de la tierra esta estrechamente relacionada con la historia del planeta. A su vez el sistema solar como cualquier otro, depende de 
la evolución estelar de su estrella o estrellas, en este caso el Sol. En los últimos años, se han hecho grandes avances para identificar y generar un catalogo completo de estrellas que puedan tener planetas 
u otros cuerpos en su orbita. El estudio de estrellas y su evolución y de como estas generan discos proto-planetarios en el proceso, es una rama importante de la astronomía y la astrofísica actual.
\subsection{Estado del Arte}
\subsubsection{Origen y Composicion del Polvo Cosmico}
La evolución de las estrellas es un proceso muy complejo que ha sido estudiado ampliamente, sin embargo con los avances tecnológicos de cada dia es posible aclarar y profundizar aun mas este estudio. 
Una de las características mas importantes que se ha encontrado en el evolución estelar es el intercambio y la interacción de material con su entorno. Por lo general en el ciclo de vida de una estrella 
existen algunas etapas. Una primera etapa, donde a partir de una nube de gas y polvo  y de la interacción gravitacional y la acreció de material se generan las primeras proto-estrellas,a travez de procesos 
termonucleares. A medida que estos procesos se incrementan también lo hacen la radiación, la temperatura y  los vientos estelares, los cuales llenan el medio interestelar. La evolucion de la estrella 
dependerá de su masa y temperatura \cite{Whittet2022DustIT}.Estos elementos definen el brillo de la estrella, la cual es su característica principal y con la cual es posible hacer una clasificacion del tipo de estrella y del periodo 
de su vida en la cual se encuentra. 

Terminar con el ciclo estelar(discos proto-planetarios, novas, supernovas, gigantes rojas, nebulosas)

Hablar de diagrama HR... y de la relación de luminosidad y temperatura para introducir la Ley de Stefan-Boltzmann

\subsubsection{Formacion Estelar y Formacion Planetaria}
El ciclo de vida de las estrellas y del polvo esta directamente relacionado. Una de las principales fuentes de granos de polvo y su presencia en el medio interestelar (ISM) se da durante las ultimas etapas 
del ciclo estelar. En esta etapa muchas estrellas comienzan a disminuir su masa por medio de la expulsion de material fuera de su superficie. Material rico en carbono (C), oxigeno (O), Silicio (Si), Magnesio (Mg) 
and iron(Fe) tambien gases como hidrogeno (H) y helio (He). Gran parte de estos materiales pesados están condensados en partículas sub-milimétricas, también llamadas polvo. LA presencia de este polvo inunda todo 
el meido interestelar incluso llegando a tener interaccion con el medio intergalactico a tra vez de nucleos activos de galaxias(AGNs), supernovas (SNe)o el transporte por vientos  interestelares \cite{mathis1977size}

\subsubsection{Radiacion del Cuerpo Negro y SED: Distribucion Espacial de Energia}

\subsubsection{Astronomía de Posición}
\subsubsection{Coordenadas Selenograficas}
\subsubsection{La Luna}
Morfologia Luna \cite{PhysicsandAstronomyMoon}

Temperatura en la superficie lunar \cite{Zhengling2024}

Analisis de distribucion espectral de energia (SED) con la herramienta CIGALE \cite{Boquien2019}

\section{Planteamiento del problema}

\printbibliography
\end{document}