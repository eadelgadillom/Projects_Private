\documentclass[12pt]{article}
\usepackage[utf8]{inputenc}
\usepackage{graphicx}
\usepackage{float}
\usepackage{booktabs,bigstrut}
\usepackage{multirow,multicol}
\usepackage{natbib}
\usepackage{amsmath,amssymb,amsfonts,latexsym,cancel}
\usepackage{tikz}
\usetikzlibrary{positioning}
\usepackage{pgfplots,caption}
\usepackage{comment}
\usepackage{pdflscape}
\usepackage{colortbl}
\usepackage{longtable}
\usepackage{xcolor}
\usepackage{rotating}
\usepackage{cancel}
\usepackage{parskip}
\usepackage{multicol}
\usepackage{hhline}
\usepackage{array}
\usepackage{longtable,lscape}
\usepackage[most]{tcolorbox}%Para colorear respuestas de ecuaciones
%\tcbset{enhanced,colframe=blue,colback={black!5!white},drop shadow}
%\tcboxmath[colframe=blue]{\mathbf{0.41\%}}


%All the packages the project needs
%--------------------------------------------
%Margins
\usepackage[top=2cm,
bottom=2cm,
left=1.6cm,
right=1.6cm,
headsep=15pt,
headheight=25pt,
letterpaper,
includehead,includefoot]{geometry} 
%-------------------------------------------

%-------------------------------------------
%Codification and language
%Document language
\usepackage[spanish,es-noshorthands]{babel}
% Required for including letters with accents
\usepackage[utf8]{inputenc} 
% Use 8-bit encoding that has 256 glyphs
\usepackage[T1]{fontenc}
%-------------------------------------------

%-------------------------------------------
%Colours
\usepackage{color}
\usepackage{xcolor}
%-------------------------------------------

%-------------------------------------------
%Fonts and text
%Add a new Font
\usepackage{addfont}
\addfont{OT1}{rsfs10}{\rsfs}
%Allows to modify the font of a section
\usepackage{sectsty}
%Latin Modern fonts - Allows to resize the font
\usepackage{lmodern}
%Allows to write code
\usepackage{verbatim}
%Dumb text
\usepackage{lipsum}
%Include currency symbols
\usepackage{textcomp} 
%-------------------------------------------

%-------------------------------------------
%Pictures and graphics
%Allows to include images
\usepackage{graphicx}
\usepackage{sidecap}
%Most powerful image package
\usepackage{tikz}
\usetikzlibrary{babel}
\usetikzlibrary{positioning}
%Sometimes used images packages

\usepackage{pst-all}
\usepackage{pstricks}
%Allows to position figures inside a paragraph
\usepackage{wrapfig}
%Information needed
%https://en.wikibooks.org/wiki/LaTeX/Floats,_Figures_and_Captions
%-------------------------------------------
%-------------------------------------------
%Math packages
\usepackage{mathtools}
\usepackage{amsmath}
\usepackage{amssymb}
\usepackage{amsfonts}
%Nice cancelations arrows
\usepackage{cancel}
%Vector arrows on a character
\usepackage{esvect}
%-------------------------------------------

%-------------------------------------------
%Table and listing packages
\usepackage{multirow}
\usepackage{array}
\usepackage{diagbox}
\usepackage{listings}
\usepackage[shortlabels]{enumitem} 
%Differents images and tables in the same space
\usepackage{subcaption}
%Use to places the float at precisely the location in the LaTeX code´and other stuff
\usepackage{float}
%Flexible tables
\usepackage{tabu}
\usepackage{multicol}
%-------------------------------------------

%-------------------------------------------
%Refs and cites
\usepackage[hidelinks]{hyperref}
%\usepackage{cite}
%-------------------------------------------

%-------------------------------------------
%Boxes
\usepackage{tcolorbox}
\usepackage{fancybox}
\usepackage{colortbl}
%-------------------------------------------

%-------------------------------------------
%Page style
\usepackage[usestackEOL]{stackengine}
\usepackage{fancyhdr}
\pagestyle{fancy}
\newsavebox{\UNl}
\newsavebox{\UNd}
\setlength{\headheight}{2.5cm}
\lhead{\usebox{\UNl}\usebox{\UNd}}
\rhead{\slshape Proyecto de Tesis}
\cfoot{}
\rfoot{\thepage}
%\lfoot{\today}
\renewcommand{\headrulewidth}{1pt}

%-------------------------------------------

%-------------------------------------------
%others
\usepackage{pgfplots,caption}
\usepackage{comment}
\usepackage{pdflscape}
\usepackage{rotating}
\usepackage{parskip}
\usepackage{hhline}
\usepackage{longtable,lscape}
\usepackage{booktabs,bigstrut}

\title{Proyecto de Tesis}
\author{Eduardo Andres Delgadillo Monsalve}
\date{\today}
%Inicio del documento
\pgfplotsset{compat=1.17}
\begin{document}
\renewcommand{\tablename}{Tabla}%% Para cambiar "Cuadro" a "Tabla"
\renewcommand{\listtablename}{Índice de tablas}
%Inicio de la portada
\begin{titlepage}
    \thispagestyle{empty}
    \begin{center}
        \begin{figure}
        \centering%
        \includegraphics{images/EscudoUN.png}
    \end{figure}
    
    \vspace{1cm}
    
        \textbf{PROYECTO DE TESIS\\"CARACTERIZACION DE LAS PROPIEDADES DE LA REGOLITA LUNAR MEDIANTE EL DISEÑO E IMPLEMENTACION DE UN MODELO ANALOGO}\\[1.3in]
      
    EDUARDO ANDRÉS DELGADILLO MONSALVE \\  [1.5in]
 Trabajo presentado como requisito parcial de la asignatura\\  Proyecto de tesis  Directo:\\[3mm] \textbf {\large{Mario Armando Higuera Garzon}}\\[1.5in]
   \textbf{UNIVERSIDAD NACIONAL DE COLOMBIA}\\
   FACULTAD  DE CIENCIAS\\
   OBSERVATORIO ASTRONOMICO NACIONAL\\
   BOGOTÁ D.C.\\
   \today
    \end{center}
\end{titlepage}

\newpage
\thispagestyle{empty}
\tableofcontents
\thispagestyle{empty}
\newpage

% \listoffigures

% \listoftables

\newpage

\setcounter{page}{1}
\section{Descripción del proyecto}

El proyecto se encuentra ubicado en la ciudad de Bogotá D.C en la microzonificación sísmica de "Lacustre 100". Consiste en una edificación  de cuatro pisos con una altura libre entre losas de 2.5 metros, que tiene un sistema estructural de pórtico de concreto reforzado resistente a momentos con capacidad moderada de disipación de energía (DMO) y cuyo tipo de uso es vivienda. La losa de entrepiso y de cubierta es aligerada y armada en una dirección.\\

Para encontrar el espesor mínimo de la losa se utilizaron las alturas promedio dadas por las tablas C.R.9.5 y C.9.5(a) de la NSR-10, para no incurrir en un costo alto de losa, obteniéndose un valor de $\mathbf{h_{losa}=0.5m}$. De acuerdo a C.8.13.2 se determina un ancho de viguetas de \textbf{0.10 m} y a partir de C.8.13.3 se encuentra una separación entre viguetas que no supere 1.2 m. Se colocan riostras transversales a las viguetas de manera tal que su separación no exceda 4.0 m de acuerdo con C.8.13.3.1. A partir de C.8.13.6, se determina un espesor de plaqueta superior de \textbf{0.05 m} que se vaciara en un casetón removible.\\


\section{Definición de la carga viva}

Para el cálculo de la carga viva en la estructura se utiliza el valor indicado en la norma NSR-10, específicamente en la \textit{tabla B.4.2.1.1-Cargas vivas  uniformemente distribuidas}, con el cual se obtiene que el valor de la carga viva corresponde a $ \mathbf{1.8 \tfrac{kN}{m^{2}}}$ para el uso de la edificación (residencial). Adicionalmente, de acuerdo con la norma, se establece una carga de $ \mathbf{3.0 \tfrac{kN}{m^{2}}}$ para la escalera. Con respecto a la cubierta se toma una carga viva mínima de $ \mathbf{1.8 \tfrac{kN}{m^{2}}}$ de acuerdo con la tabla B.4.2.1-2. Para el ascensor se considera una carga viva de $ \mathbf{20 \tfrac{kN}{m^{2}}}$  en donde se tiene en cuenta la carga de impacto.\\ 

\section{Evaluación de las cargas muertas}

La evaluación de la carga muerta se realiza teniendo en cuenta que el material que se utilizará para construir la estructura es el concreto armado, cuyo peso por unidad de volumen corresponde a 24 $\frac{kN}{m^{2}}$ de acuerdo con la tabla B.3.2-1 de la NSR-10.\\