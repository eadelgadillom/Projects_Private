\documentclass[12pt]{article}
\usepackage[utf8]{inputenc}
\usepackage{graphicx}
\usepackage{float}
\usepackage{booktabs,bigstrut}
\usepackage{multirow,multicol}
\usepackage{natbib}
\usepackage{amsmath,amssymb,amsfonts,latexsym,cancel}
\usepackage{tikz}
\usetikzlibrary{positioning}
\usepackage{pgfplots,caption}
\usepackage{comment}
\usepackage{pdflscape}
\usepackage{colortbl}
\usepackage{longtable}
\usepackage{xcolor}
\usepackage{rotating}
\usepackage{cancel}
\usepackage{parskip}
\usepackage{multicol}
\usepackage{hhline}
\usepackage{array}
\usepackage{longtable,lscape}
\usepackage[most]{tcolorbox}%Para colorear respuestas de ecuaciones
%\tcbset{enhanced,colframe=blue,colback={black!5!white},drop shadow}
%\tcboxmath[colframe=blue]{\mathbf{0.41\%}}


%All the packages the project needs
%--------------------------------------------
%Margins
\usepackage[top=2cm,
bottom=2cm,
left=1.6cm,
right=1.6cm,
headsep=15pt,
headheight=25pt,
letterpaper,
includehead,includefoot]{geometry} 
%-------------------------------------------

%-------------------------------------------
%Codification and language
%Document language
\usepackage[spanish,es-noshorthands]{babel}
% Required for including letters with accents
\usepackage[utf8]{inputenc} 
% Use 8-bit encoding that has 256 glyphs
\usepackage[T1]{fontenc}
%-------------------------------------------

%-------------------------------------------
%Colours
\usepackage{color}
\usepackage{xcolor}
%-------------------------------------------

%-------------------------------------------
%Fonts and text
%Add a new Font
\usepackage{addfont}
\addfont{OT1}{rsfs10}{\rsfs}
%Allows to modify the font of a section
\usepackage{sectsty}
%Latin Modern fonts - Allows to resize the font
\usepackage{lmodern}
%Allows to write code
\usepackage{verbatim}
%Dumb text
\usepackage{lipsum}
%Include currency symbols
\usepackage{textcomp} 
%-------------------------------------------

%-------------------------------------------
%Pictures and graphics
%Allows to include images
\usepackage{graphicx}
\usepackage{sidecap}
%Most powerful image package
\usepackage{tikz}
\usetikzlibrary{babel}
\usetikzlibrary{positioning}
%Sometimes used images packages

\usepackage{pst-all}
\usepackage{pstricks}
%Allows to position figures inside a paragraph
\usepackage{wrapfig}
%Information needed
%https://en.wikibooks.org/wiki/LaTeX/Floats,_Figures_and_Captions
%-------------------------------------------
%-------------------------------------------
%Math packages
\usepackage{mathtools}
\usepackage{amsmath}
\usepackage{amssymb}
\usepackage{amsfonts}
%Nice cancelations arrows
\usepackage{cancel}
%Vector arrows on a character
\usepackage{esvect}
%-------------------------------------------

%-------------------------------------------
%Table and listing packages
\usepackage{multirow}
\usepackage{array}
\usepackage{diagbox}
\usepackage{listings}
\usepackage[shortlabels]{enumitem} 
%Differents images and tables in the same space
\usepackage{subcaption}
%Use to places the float at precisely the location in the LaTeX code´and other stuff
\usepackage{float}
%Flexible tables
\usepackage{tabu}
\usepackage{multicol}
%-------------------------------------------

%-------------------------------------------
%Refs and cites
\usepackage[hidelinks]{hyperref}
%\usepackage{cite}
%-------------------------------------------

%-------------------------------------------
%Boxes
\usepackage{tcolorbox}
\usepackage{fancybox}
\usepackage{colortbl}
%-------------------------------------------

%-------------------------------------------
%Page style
\usepackage[usestackEOL]{stackengine}
\usepackage{fancyhdr}
\pagestyle{fancy}
\newsavebox{\UNl}
\newsavebox{\UNd}
\setlength{\headheight}{2.5cm}
\lhead{\usebox{\UNl}\usebox{\UNd}}
\rhead{\slshape Proyecto de Tesis}
\cfoot{}
\rfoot{\thepage}
%\lfoot{\today}
\renewcommand{\headrulewidth}{1pt}

%-------------------------------------------

%-------------------------------------------
%others
\usepackage{pgfplots,caption}
\usepackage{comment}
\usepackage{pdflscape}
\usepackage{rotating}
\usepackage{parskip}
\usepackage{hhline}
\usepackage{longtable,lscape}
\usepackage{booktabs,bigstrut}

\title{Proyecto de Tesis}
\author{Eduardo Andres Delgadillo Monsalve}
\date{\today}
%Inicio del documento
\pgfplotsset{compat=1.17}
\begin{document}
\renewcommand{\tablename}{Tabla}%% Para cambiar "Cuadro" a "Tabla"
\renewcommand{\listtablename}{Índice de tablas}
%Inicio de la portada
\begin{titlepage}
    \centering
    \thispagestyle{empty}
    \begin{center}
        \begin{figure}
        \centering%
        \includegraphics{images/EscudoUN.png}
    \end{figure}
    
    \vspace{3cm}
    
        \textbf{CARACTERIZACION DE LAS PROPIEDADES DE LA REGOLITA LUNAR MEDIANTE EL DISEÑO E IMPLEMENTACION DE UN MODELO ANALOGO}\\[2in]
    Eduardo Andrés Delgadillo Monsalve \\  [3in]

   \textbf{Universidad Nacional de Colombia}\\
   Facultad de Ciencias\\
   Bogotá D.C.\\
   \today
    \end{center}
\end{titlepage}

\newpage
\begin{titlepage}
    \thispagestyle{empty}
    \begin{center}
        \begin{figure}
        \centering%
        \includegraphics{images/EscudoUN.png}
    \end{figure}
    
    \vspace{1cm}
    
        \textbf{CARACTERIZACION DE LAS PROPIEDADES DE LA REGOLITA LUNAR MEDIANTE EL DISEÑO E IMPLEMENTACION DE UN MODELO ANALOGO}\\[1in]    
    Eduardo Andrés Delgadillo Monsalve \\  [1.5 cm]
 Proyecto de tesis presentado como requisito parcial paara optar al titulo de:\\[3mm] \textbf {\large{Magister en Ciencias Astronomia}}\\[1cm]
 Director:\\[3mm] Mario Armando Higuera Garzon\\[1mm] Observatorio Astronomico Nacional\\[1cm]
 Codirector:\\[3mm] David Ardila\\ Jet Propulsion Laboratory\\[1in]
 \textbf{Universidad Nacional de Colombia  }\\
   Facultad de Ciencias\\
   Bogotá D.C.\\
   \today
    \end{center}
\end{titlepage}

\newpage
\thispagestyle{empty}
\tableofcontents
\thispagestyle{empty}
\newpage

% \listoffigures

% \listoftables

\newpage

\setcounter{page}{1}
\section{Introduccion}
La luna ha sido uno de los objetos mas estudiados y de mayor interés para la humanidad desde epocas antiguas. En la mayoria de civilizaciones y culturas,
de las que se tienen registro, la Luna ha tenido un papel fundamental dentro de la cosmogonia propia. Este satelite natural se considero por mucho tiempo como 
la contraparte de la estrella que domina el cielo diurno, el Sol. Eventos naturales como los eclipses de Sol, los eclipses de Luna y el ciclo lunar, fueron interpretados
y estudiado a lo laergo de toda la historia humana, incluso uno de los mayores logros de la astronomia y de la humanidad ha sido el alunizaje de seres humanos sobre la superficie 
de la Luna, en el año 1969. Desde entonces y sobre todo en los ultimos años la busqueda por habitar la superficie lunar


\section{Objetivo}

Caracterizar las propiedades de la regolita lunar mediante el diseño e implementación  de un modelo análogo

\subsection{Objetivos especificos}
\begin{itemize}
    \item Estudiar  las propiedades fisicoquímicas  de cuerpos rocosos menores en el sistema solar
    \item Caracterizar  la estructura y composición de la regolita en la superficie lunar
    \item Diseñar un modelo análogo que permita reproducir la estructura y composición de la regolita lunar
\end{itemize}


% % Table generated by Excel2LaTeX from sheet '1'
\begin{table}[htbp]
  \centering
  \caption{Tabla de resultados del software para viga sin apoyo 1}
  \resizebox{\linewidth}{!}{
    \begin{tabular}{rrrlrrrrr}
    \rowcolor[rgb]{ .2,  .8,  .8} \multicolumn{1}{l}{\textbf{TABLE:  Element Forces - Frames}} &     &     &     &     &     & \cellcolor[rgb]{ 1,  1,  1} & \cellcolor[rgb]{ 1,  1,  1} & \cellcolor[rgb]{ 1,  1,  1} \bigstrut[b]\\
\cline{1-6}\cline{8-9}    \rowcolor[rgb]{ .8,  1,  1} \multicolumn{1}{|c|}{\textbf{Frame}} & \multicolumn{1}{c|}{\textbf{Station}} & \multicolumn{1}{c|}{\textbf{OutputCase}} & \multicolumn{1}{c|}{\textbf{CaseType}} & \multicolumn{1}{c|}{\textbf{V2}} & \multicolumn{1}{c|}{\textbf{M3}} & \multicolumn{1}{r|}{\cellcolor[rgb]{ 1,  1,  1}} & \multicolumn{1}{c|}{\textbf{V2}} & \multicolumn{1}{c|}{\textbf{M3}} \bigstrut\\
\cline{1-6}\cline{8-9}    \rowcolor[rgb]{ .8,  1,  1} \multicolumn{1}{|c|}{Text} & \multicolumn{1}{c|}{m} & \multicolumn{1}{c|}{Text} & \multicolumn{1}{c|}{Text} & \multicolumn{1}{c|}{KN} & \multicolumn{1}{c|}{KN-m} & \multicolumn{1}{r|}{\cellcolor[rgb]{ 1,  1,  1}} & \multicolumn{1}{c|}{KN} & \multicolumn{1}{c|}{KN-m} \bigstrut\\
\cline{1-6}\cline{8-9}    \multicolumn{1}{l}{7} & 0   & \multicolumn{1}{l}{PU} & LinStatic & 22.577 & 51.5467 &     & 22.577 & 51.5467 \bigstrut[t]\\
    \multicolumn{1}{l}{7} & 0.48571 & \multicolumn{1}{l}{PU} & LinStatic & 22.577 & 40.5809 &     & 22.577 & 40.5809 \\
    \multicolumn{1}{l}{7} & 0.97143 & \multicolumn{1}{l}{PU} & LinStatic & 22.577 & 29.6151 &     & 22.577 & 29.6151 \\
    \multicolumn{1}{l}{7} & 1.45714 & \multicolumn{1}{l}{PU} & LinStatic & 22.577 & 18.6493 &     & 22.577 & 18.6493 \\
    \multicolumn{1}{l}{7} & 1.94286 & \multicolumn{1}{l}{PU} & LinStatic & 22.577 & 7.6835 &     & 22.577 & 7.6835 \\
    \multicolumn{1}{l}{7} & 2.42857 & \multicolumn{1}{l}{PU} & LinStatic & 22.577 & -3.2823 &     & 22.577 & -3.2823 \\
    \multicolumn{1}{l}{7} & 2.91429 & \multicolumn{1}{l}{PU} & LinStatic & 22.577 & -14.2481 &     & 22.577 & -14.2481 \\
    \multicolumn{1}{l}{7} & 3.4 & \multicolumn{1}{l}{PU} & LinStatic & 22.577 & -25.2139 &     & 22.577 & -25.2139 \\
    \multicolumn{1}{l}{7} & 3.88571 & \multicolumn{1}{l}{PU} & LinStatic & 22.577 & -36.1797 &     & 22.577 & -36.1797 \\
    \multicolumn{1}{l}{7} & 4.37143 & \multicolumn{1}{l}{PU} & LinStatic & 22.577 & -47.1455 &     & 22.577 & -47.1455 \\
    \multicolumn{1}{l}{7} & 4.85714 & \multicolumn{1}{l}{PU} & LinStatic & 22.577 & -58.1113 &     & 22.577 & -58.1113 \\
    \multicolumn{1}{l}{7} & 5.34286 & \multicolumn{1}{l}{PU} & LinStatic & 22.577 & -69.0771 &     & 22.577 & -69.0771 \\
    \multicolumn{1}{l}{7} & 5.82857 & \multicolumn{1}{l}{PU} & LinStatic & 22.577 & -80.0429 &     & 22.577 & -80.0429 \\
    \multicolumn{1}{l}{7} & 6.31429 & \multicolumn{1}{l}{PU} & LinStatic & 22.577 & -91.0087 &     & 22.577 & -91.0087 \\
    \multicolumn{1}{l}{7} & 6.8 & \multicolumn{1}{l}{PU} & LinStatic & 22.577 & -101.9745 &     & 22.577 & -101.9745 \\
    \multicolumn{1}{l}{8} & 0   & \multicolumn{1}{l}{PU} & LinStatic & -13.806 & -74.8394 &     & -13.806 & -74.8394 \\
    \multicolumn{1}{l}{8} & 0.5 & \multicolumn{1}{l}{PU} & LinStatic & -13.806 & -67.9366 &     & -13.806 & -67.9366 \\
    \multicolumn{1}{l}{8} & 1   & \multicolumn{1}{l}{PU} & LinStatic & -13.806 & -61.0337 &     & -13.806 & -61.0337 \\
    \multicolumn{1}{l}{8} & 1.5 & \multicolumn{1}{l}{PU} & LinStatic & -13.806 & -54.1308 &     & -13.806 & -54.1308 \\
    \multicolumn{1}{l}{8} & 2   & \multicolumn{1}{l}{PU} & LinStatic & -13.806 & -47.2279 &     & -13.806 & -47.2279 \\
    \multicolumn{1}{l}{8} & 2.5 & \multicolumn{1}{l}{PU} & LinStatic & -13.806 & -40.325 &     & -13.806 & -40.325 \\
    \multicolumn{1}{l}{8} & 3   & \multicolumn{1}{l}{PU} & LinStatic & -13.806 & -33.4221 &     & -13.806 & -33.4221 \\
    \multicolumn{1}{l}{8} & 3.5 & \multicolumn{1}{l}{PU} & LinStatic & -13.806 & -26.5192 &     & -13.806 & -26.5192 \\
    \multicolumn{1}{l}{8} & 4   & \multicolumn{1}{l}{PU} & LinStatic & -13.806 & -19.6163 &     & -13.806 & -19.6163 \\
    \multicolumn{1}{l}{8} & 4.5 & \multicolumn{1}{l}{PU} & LinStatic & -13.806 & -12.7135 &     & -13.806 & -12.7135 \\
    \multicolumn{1}{l}{8} & 5   & \multicolumn{1}{l}{PU} & LinStatic & -13.806 & -5.8106 &     & -13.806 & -5.8106 \\
    \multicolumn{1}{l}{8} & 5.5 & \multicolumn{1}{l}{PU} & LinStatic & -13.806 & 1.0923 &     & -13.806 & 1.0923 \\
    \multicolumn{1}{l}{8} & 6   & \multicolumn{1}{l}{PU} & LinStatic & -13.806 & 7.9952 &     & -13.806 & 7.9952 \\
    \multicolumn{1}{l}{8} & 6.5 & \multicolumn{1}{l}{PU} & LinStatic & -13.806 & 14.8981 &     & -13.806 & 14.8981 \\
    \multicolumn{1}{l}{8} & 7   & \multicolumn{1}{l}{PU} & LinStatic & -13.806 & 21.801 &     & -13.806 & 21.801 \\
    \multicolumn{1}{l}{9} & 0   & \multicolumn{1}{l}{PU} & LinStatic & 2.608 & 16.0804 &     & 2.608 & 16.0804 \\
    \multicolumn{1}{l}{9} & 0.49286 & \multicolumn{1}{l}{PU} & LinStatic & 2.608 & 14.7948 &     & 2.608 & 14.7948 \\
    \multicolumn{1}{l}{9} & 0.98571 & \multicolumn{1}{l}{PU} & LinStatic & 2.608 & 13.5092 &     & 2.608 & 13.5092 \\
    \multicolumn{1}{l}{9} & 1.47857 & \multicolumn{1}{l}{PU} & LinStatic & 2.608 & 12.2236 &     & 2.608 & 12.2236 \\
    \multicolumn{1}{l}{9} & 1.97143 & \multicolumn{1}{l}{PU} & LinStatic & 2.608 & 10.938 &     & 2.608 & 10.938 \\
    \multicolumn{1}{l}{9} & 2.46429 & \multicolumn{1}{l}{PU} & LinStatic & 2.608 & 9.6524 &     & 2.608 & 9.6524 \\
    \multicolumn{1}{l}{9} & 2.95714 & \multicolumn{1}{l}{PU} & LinStatic & 2.608 & 8.3668 &     & 2.608 & 8.3668 \\
    \multicolumn{1}{l}{9} & 3.45 & \multicolumn{1}{l}{PU} & LinStatic & 2.608 & 7.0812 &     & 2.608 & 7.0812 \\
    \multicolumn{1}{l}{9} & 3.94286 & \multicolumn{1}{l}{PU} & LinStatic & 2.608 & 5.7956 &     & 2.608 & 5.7956 \\
    \multicolumn{1}{l}{9} & 4.43571 & \multicolumn{1}{l}{PU} & LinStatic & 2.608 & 4.51 &     & 2.608 & 4.51 \\
    \multicolumn{1}{l}{9} & 4.92857 & \multicolumn{1}{l}{PU} & LinStatic & 2.608 & 3.2244 &     & 2.608 & 3.2244 \\
    \multicolumn{1}{l}{9} & 5.42143 & \multicolumn{1}{l}{PU} & LinStatic & 2.608 & 1.9388 &     & 2.608 & 1.9388 \\
    \multicolumn{1}{l}{9} & 5.91429 & \multicolumn{1}{l}{PU} & LinStatic & 2.608 & 0.6532 &     & 2.608 & 0.6532 \\
    \multicolumn{1}{l}{9} & 6.40714 & \multicolumn{1}{l}{PU} & LinStatic & 2.608 & -0.6324 &     & 2.608 & -0.6324 \\
    \multicolumn{1}{l}{9} & 6.9 & \multicolumn{1}{l}{PU} & LinStatic & 2.608 & -1.918 &     & 2.608 & -1.918 \\
        &     &     &     &     &     &     &     &  \\
        &     &     & Max & 22.577 & 51.5467 &     &     &  \\
        &     &     & Min & -13.806 & -101.9745 &     &     &  \\
    \end{tabular}%
    }
  \label{tab:addlabel}%
\end{table}%

% % Table generated by Excel2LaTeX from sheet '2'
\begin{table}[H]
  \centering
  \caption{Tabla de resultados del software para viga sin apoyo 2}
  \resizebox{\linewidth}{!}{
    \begin{tabular}{rrrlrr}
    \rowcolor[rgb]{ .2,  .8,  .8} \multicolumn{1}{l}{\textbf{TABLE:  Element Forces - Frames}} &     &     &     &     &  \bigstrut[b]\\
    \hline
    \rowcolor[rgb]{ .8,  1,  1} \multicolumn{1}{|c|}{\textbf{Frame}} & \multicolumn{1}{c|}{\textbf{Station}} & \multicolumn{1}{c|}{\textbf{OutputCase}} & \multicolumn{1}{c|}{\textbf{CaseType}} & \multicolumn{1}{c|}{\textbf{V2}} & \multicolumn{1}{c|}{\textbf{M3}} \bigstrut\\
    \hline
    \rowcolor[rgb]{ .8,  1,  1} \multicolumn{1}{|c|}{Text} & \multicolumn{1}{c|}{m} & \multicolumn{1}{c|}{Text} & \multicolumn{1}{c|}{Text} & \multicolumn{1}{c|}{KN} & \multicolumn{1}{c|}{KN-m} \bigstrut\\
    \hline
    \multicolumn{1}{l}{7} & 0   & \multicolumn{1}{l}{SI} & LinStatic & -68.206 & -115.7989 \bigstrut[t]\\
    \multicolumn{1}{l}{7} & 0.48571 & \multicolumn{1}{l}{SI} & LinStatic & -68.206 & -82.6703 \\
    \multicolumn{1}{l}{7} & 0.97143 & \multicolumn{1}{l}{SI} & LinStatic & -68.206 & -49.5417 \\
    \multicolumn{1}{l}{7} & 1.45714 & \multicolumn{1}{l}{SI} & LinStatic & -68.206 & -16.4131 \\
    \multicolumn{1}{l}{7} & 1.94286 & \multicolumn{1}{l}{SI} & LinStatic & -68.206 & 16.7155 \\
    \multicolumn{1}{l}{7} & 2.42857 & \multicolumn{1}{l}{SI} & LinStatic & -68.206 & 49.8441 \\
    \multicolumn{1}{l}{7} & 2.91429 & \multicolumn{1}{l}{SI} & LinStatic & -68.206 & 82.9727 \\
    \multicolumn{1}{l}{7} & 3.4 & \multicolumn{1}{l}{SI} & LinStatic & -68.206 & 116.1012 \\
    \multicolumn{1}{l}{7} & 3.88571 & \multicolumn{1}{l}{SI} & LinStatic & -68.206 & 149.2298 \\
    \multicolumn{1}{l}{7} & 4.37143 & \multicolumn{1}{l}{SI} & LinStatic & -68.206 & 182.3584 \\
    \multicolumn{1}{l}{7} & 4.85714 & \multicolumn{1}{l}{SI} & LinStatic & -68.206 & 215.487 \\
    \multicolumn{1}{l}{7} & 5.34286 & \multicolumn{1}{l}{SI} & LinStatic & -68.206 & 248.6156 \\
    \multicolumn{1}{l}{7} & 5.82857 & \multicolumn{1}{l}{SI} & LinStatic & -68.206 & 281.7442 \\
    \multicolumn{1}{l}{7} & 6.31429 & \multicolumn{1}{l}{SI} & LinStatic & -68.206 & 314.8728 \\
    \multicolumn{1}{l}{7} & 6.8 & \multicolumn{1}{l}{SI} & LinStatic & -68.206 & 348.0014 \\
    \multicolumn{1}{l}{8} & 0   & \multicolumn{1}{l}{SI} & LinStatic & 86.908 & 354.2352 \\
    \multicolumn{1}{l}{8} & 0.5 & \multicolumn{1}{l}{SI} & LinStatic & 86.908 & 310.7814 \\
    \multicolumn{1}{l}{8} & 1   & \multicolumn{1}{l}{SI} & LinStatic & 86.908 & 267.3276 \\
    \multicolumn{1}{l}{8} & 1.5 & \multicolumn{1}{l}{SI} & LinStatic & 86.908 & 223.8738 \\
    \multicolumn{1}{l}{8} & 2   & \multicolumn{1}{l}{SI} & LinStatic & 86.908 & 180.42 \\
    \multicolumn{1}{l}{8} & 2.5 & \multicolumn{1}{l}{SI} & LinStatic & 86.908 & 136.9662 \\
    \multicolumn{1}{l}{8} & 3   & \multicolumn{1}{l}{SI} & LinStatic & 86.908 & 93.5124 \\
    \multicolumn{1}{l}{8} & 3.5 & \multicolumn{1}{l}{SI} & LinStatic & 86.908 & 50.0586 \\
    \multicolumn{1}{l}{8} & 4   & \multicolumn{1}{l}{SI} & LinStatic & 86.908 & 6.6048 \\
    \multicolumn{1}{l}{8} & 4.5 & \multicolumn{1}{l}{SI} & LinStatic & 86.908 & -36.849 \\
    \multicolumn{1}{l}{8} & 5   & \multicolumn{1}{l}{SI} & LinStatic & 86.908 & -80.3028 \\
    \multicolumn{1}{l}{8} & 5.5 & \multicolumn{1}{l}{SI} & LinStatic & 86.908 & -123.7566 \\
    \multicolumn{1}{l}{8} & 6   & \multicolumn{1}{l}{SI} & LinStatic & 86.908 & -167.2104 \\
    \multicolumn{1}{l}{8} & 6.5 & \multicolumn{1}{l}{SI} & LinStatic & 86.908 & -210.6642 \\
    \multicolumn{1}{l}{8} & 7   & \multicolumn{1}{l}{SI} & LinStatic & 86.908 & -254.118 \\
    \multicolumn{1}{l}{9} & 0   & \multicolumn{1}{l}{SI} & LinStatic & -30.252 & -185.9331 \\
    \multicolumn{1}{l}{9} & 0.49286 & \multicolumn{1}{l}{SI} & LinStatic & -30.252 & -171.0231 \\
    \multicolumn{1}{l}{9} & 0.98571 & \multicolumn{1}{l}{SI} & LinStatic & -30.252 & -156.113 \\
    \multicolumn{1}{l}{9} & 1.47857 & \multicolumn{1}{l}{SI} & LinStatic & -30.252 & -141.203 \\
    \multicolumn{1}{l}{9} & 1.97143 & \multicolumn{1}{l}{SI} & LinStatic & -30.252 & -126.2929 \\
    \multicolumn{1}{l}{9} & 2.46429 & \multicolumn{1}{l}{SI} & LinStatic & -30.252 & -111.3829 \\
    \multicolumn{1}{l}{9} & 2.95714 & \multicolumn{1}{l}{SI} & LinStatic & -30.252 & -96.4729 \\
    \multicolumn{1}{l}{9} & 3.45 & \multicolumn{1}{l}{SI} & LinStatic & -30.252 & -81.5628 \\
    \multicolumn{1}{l}{9} & 3.94286 & \multicolumn{1}{l}{SI} & LinStatic & -30.252 & -66.6528 \\
    \multicolumn{1}{l}{9} & 4.43571 & \multicolumn{1}{l}{SI} & LinStatic & -30.252 & -51.7427 \\
    \multicolumn{1}{l}{9} & 4.92857 & \multicolumn{1}{l}{SI} & LinStatic & -30.252 & -36.8327 \\
    \multicolumn{1}{l}{9} & 5.42143 & \multicolumn{1}{l}{SI} & LinStatic & -30.252 & -21.9226 \\
    \multicolumn{1}{l}{9} & 5.91429 & \multicolumn{1}{l}{SI} & LinStatic & -30.252 & -7.0126 \\
    \multicolumn{1}{l}{9} & 6.40714 & \multicolumn{1}{l}{SI} & LinStatic & -30.252 & 7.8975 \\
    \multicolumn{1}{l}{9} & 6.9 & \multicolumn{1}{l}{SI} & LinStatic & -30.252 & 22.8075 \\
        &     &     &     &     &  \\
        &     &     & Max & 86.908 & 354.2352 \\
        &     &     & Min & -68.206 & -254.118 \\
    \end{tabular}%
  \label{tab:addlabel}%
  }
\end{table}%

% % Table generated by Excel2LaTeX from sheet '3'
\begin{table}[htbp]
  \centering
  \caption{Tabla de resultados del software para viga sin apoyo 3}
  \resizebox{\linewidth}{!}{
    \begin{tabular}{rrrlrr}
    \rowcolor[rgb]{ .2,  .8,  .8} \multicolumn{1}{l}{\textbf{TABLE:  Element Forces - Frames}} &     &     &     &     &  \bigstrut[b]\\
    \hline
    \rowcolor[rgb]{ .8,  1,  1} \multicolumn{1}{|c|}{\textbf{Frame}} & \multicolumn{1}{c|}{\textbf{Station}} & \multicolumn{1}{c|}{\textbf{OutputCase}} & \multicolumn{1}{c|}{\textbf{CaseType}} & \multicolumn{1}{c|}{\textbf{V2}} & \multicolumn{1}{c|}{\textbf{M3}} \bigstrut\\
    \hline
    \rowcolor[rgb]{ .8,  1,  1} \multicolumn{1}{|c|}{Text} & \multicolumn{1}{c|}{m} & \multicolumn{1}{c|}{Text} & \multicolumn{1}{c|}{Text} & \multicolumn{1}{c|}{KN} & \multicolumn{1}{c|}{KN-m} \bigstrut\\
    \hline
    \multicolumn{1}{l}{7} & 0   & \multicolumn{1}{l}{SI} & LinStatic & 36.026 & 26.5515 \bigstrut[t]\\
    \multicolumn{1}{l}{7} & 0.48571 & \multicolumn{1}{l}{SI} & LinStatic & 36.026 & 9.0532 \\
    \multicolumn{1}{l}{7} & 0.97143 & \multicolumn{1}{l}{SI} & LinStatic & 36.026 & -8.4452 \\
    \multicolumn{1}{l}{7} & 1.45714 & \multicolumn{1}{l}{SI} & LinStatic & 36.026 & -25.9435 \\
    \multicolumn{1}{l}{7} & 1.94286 & \multicolumn{1}{l}{SI} & LinStatic & 36.026 & -43.4419 \\
    \multicolumn{1}{l}{7} & 2.42857 & \multicolumn{1}{l}{SI} & LinStatic & 36.026 & -60.9402 \\
    \multicolumn{1}{l}{7} & 2.91429 & \multicolumn{1}{l}{SI} & LinStatic & 36.026 & -78.4385 \\
    \multicolumn{1}{l}{7} & 3.4 & \multicolumn{1}{l}{SI} & LinStatic & 36.026 & -95.9369 \\
    \multicolumn{1}{l}{7} & 3.88571 & \multicolumn{1}{l}{SI} & LinStatic & 36.026 & -113.4352 \\
    \multicolumn{1}{l}{7} & 4.37143 & \multicolumn{1}{l}{SI} & LinStatic & 36.026 & -130.9336 \\
    \multicolumn{1}{l}{7} & 4.85714 & \multicolumn{1}{l}{SI} & LinStatic & 36.026 & -148.4319 \\
    \multicolumn{1}{l}{7} & 5.34286 & \multicolumn{1}{l}{SI} & LinStatic & 36.026 & -165.9303 \\
    \multicolumn{1}{l}{7} & 5.82857 & \multicolumn{1}{l}{SI} & LinStatic & 36.026 & -183.4286 \\
    \multicolumn{1}{l}{7} & 6.31429 & \multicolumn{1}{l}{SI} & LinStatic & 36.026 & -200.927 \\
    \multicolumn{1}{l}{7} & 6.8 & \multicolumn{1}{l}{SI} & LinStatic & 36.026 & -218.4253 \\
    \multicolumn{1}{l}{8} & 0   & \multicolumn{1}{l}{SI} & LinStatic & -101.139 & -297.5059 \\
    \multicolumn{1}{l}{8} & 0.5 & \multicolumn{1}{l}{SI} & LinStatic & -101.139 & -246.9363 \\
    \multicolumn{1}{l}{8} & 1   & \multicolumn{1}{l}{SI} & LinStatic & -101.139 & -196.3668 \\
    \multicolumn{1}{l}{8} & 1.5 & \multicolumn{1}{l}{SI} & LinStatic & -101.139 & -145.7972 \\
    \multicolumn{1}{l}{8} & 2   & \multicolumn{1}{l}{SI} & LinStatic & -101.139 & -95.2276 \\
    \multicolumn{1}{l}{8} & 2.5 & \multicolumn{1}{l}{SI} & LinStatic & -101.139 & -44.6581 \\
    \multicolumn{1}{l}{8} & 3   & \multicolumn{1}{l}{SI} & LinStatic & -101.139 & 5.9115 \\
    \multicolumn{1}{l}{8} & 3.5 & \multicolumn{1}{l}{SI} & LinStatic & -101.139 & 56.4811 \\
    \multicolumn{1}{l}{8} & 4   & \multicolumn{1}{l}{SI} & LinStatic & -101.139 & 107.0507 \\
    \multicolumn{1}{l}{8} & 4.5 & \multicolumn{1}{l}{SI} & LinStatic & -101.139 & 157.6202 \\
    \multicolumn{1}{l}{8} & 5   & \multicolumn{1}{l}{SI} & LinStatic & -101.139 & 208.1898 \\
    \multicolumn{1}{l}{8} & 5.5 & \multicolumn{1}{l}{SI} & LinStatic & -101.139 & 258.7594 \\
    \multicolumn{1}{l}{8} & 6   & \multicolumn{1}{l}{SI} & LinStatic & -101.139 & 309.3289 \\
    \multicolumn{1}{l}{8} & 6.5 & \multicolumn{1}{l}{SI} & LinStatic & -101.139 & 359.8985 \\
    \multicolumn{1}{l}{8} & 7   & \multicolumn{1}{l}{SI} & LinStatic & -101.139 & 410.4681 \\
    \multicolumn{1}{l}{9} & 0   & \multicolumn{1}{l}{SI} & LinStatic & 77.554 & 401.4768 \\
    \multicolumn{1}{l}{9} & 0.49286 & \multicolumn{1}{l}{SI} & LinStatic & 77.554 & 363.2537 \\
    \multicolumn{1}{l}{9} & 0.98571 & \multicolumn{1}{l}{SI} & LinStatic & 77.554 & 325.0306 \\
    \multicolumn{1}{l}{9} & 1.47857 & \multicolumn{1}{l}{SI} & LinStatic & 77.554 & 286.8076 \\
    \multicolumn{1}{l}{9} & 1.97143 & \multicolumn{1}{l}{SI} & LinStatic & 77.554 & 248.5845 \\
    \multicolumn{1}{l}{9} & 2.46429 & \multicolumn{1}{l}{SI} & LinStatic & 77.554 & 210.3614 \\
    \multicolumn{1}{l}{9} & 2.95714 & \multicolumn{1}{l}{SI} & LinStatic & 77.554 & 172.1383 \\
    \multicolumn{1}{l}{9} & 3.45 & \multicolumn{1}{l}{SI} & LinStatic & 77.554 & 133.9152 \\
    \multicolumn{1}{l}{9} & 3.94286 & \multicolumn{1}{l}{SI} & LinStatic & 77.554 & 95.6921 \\
    \multicolumn{1}{l}{9} & 4.43571 & \multicolumn{1}{l}{SI} & LinStatic & 77.554 & 57.4691 \\
    \multicolumn{1}{l}{9} & 4.92857 & \multicolumn{1}{l}{SI} & LinStatic & 77.554 & 19.246 \\
    \multicolumn{1}{l}{9} & 5.42143 & \multicolumn{1}{l}{SI} & LinStatic & 77.554 & -18.9771 \\
    \multicolumn{1}{l}{9} & 5.91429 & \multicolumn{1}{l}{SI} & LinStatic & 77.554 & -57.2002 \\
    \multicolumn{1}{l}{9} & 6.40714 & \multicolumn{1}{l}{SI} & LinStatic & 77.554 & -95.4233 \\
    \multicolumn{1}{l}{9} & 6.9 & \multicolumn{1}{l}{SI} & LinStatic & 77.554 & -133.6464 \\
        &     &     &     &     &  \\
        &     &     & Max & 77.554 & 410.4681 \\
        &     &     & Min & -101.139 & -297.5059 \\
    \end{tabular}%
    }
  \label{tab:addlabel}%
\end{table}%

% % Table generated by Excel2LaTeX from sheet '4'
\begin{table}[htbp]
  \centering
  \caption{Tabla de resultados del software para viga sin apoyo 4}
  \resizebox{\linewidth}{!}{
    \begin{tabular}{rrrlrr}
    \rowcolor[rgb]{ .2,  .8,  .8} \multicolumn{1}{l}{\textbf{TABLE:  Element Forces - Frames}} &     &     &     &     &  \bigstrut[b]\\
    \hline
    \rowcolor[rgb]{ .8,  1,  1} \multicolumn{1}{|c|}{\textbf{Frame}} & \multicolumn{1}{c|}{\textbf{Station}} & \multicolumn{1}{c|}{\textbf{OutputCase}} & \multicolumn{1}{c|}{\textbf{CaseType}} & \multicolumn{1}{c|}{\textbf{V2}} & \multicolumn{1}{c|}{\textbf{M3}} \bigstrut\\
    \hline
    \rowcolor[rgb]{ .8,  1,  1} \multicolumn{1}{|c|}{Text} & \multicolumn{1}{c|}{m} & \multicolumn{1}{c|}{Text} & \multicolumn{1}{c|}{Text} & \multicolumn{1}{c|}{KN} & \multicolumn{1}{c|}{KN-m} \bigstrut\\
    \hline
    \multicolumn{1}{l}{7} & 0   & \multicolumn{1}{l}{SI} & LinStatic & -5.533 & -3.9763 \bigstrut[t]\\
    \multicolumn{1}{l}{7} & 0.48571 & \multicolumn{1}{l}{SI} & LinStatic & -5.533 & -1.289 \\
    \multicolumn{1}{l}{7} & 0.97143 & \multicolumn{1}{l}{SI} & LinStatic & -5.533 & 1.3983 \\
    \multicolumn{1}{l}{7} & 1.45714 & \multicolumn{1}{l}{SI} & LinStatic & -5.533 & 4.0856 \\
    \multicolumn{1}{l}{7} & 1.94286 & \multicolumn{1}{l}{SI} & LinStatic & -5.533 & 6.7729 \\
    \multicolumn{1}{l}{7} & 2.42857 & \multicolumn{1}{l}{SI} & LinStatic & -5.533 & 9.4602 \\
    \multicolumn{1}{l}{7} & 2.91429 & \multicolumn{1}{l}{SI} & LinStatic & -5.533 & 12.1475 \\
    \multicolumn{1}{l}{7} & 3.4 & \multicolumn{1}{l}{SI} & LinStatic & -5.533 & 14.8348 \\
    \multicolumn{1}{l}{7} & 3.88571 & \multicolumn{1}{l}{SI} & LinStatic & -5.533 & 17.5221 \\
    \multicolumn{1}{l}{7} & 4.37143 & \multicolumn{1}{l}{SI} & LinStatic & -5.533 & 20.2094 \\
    \multicolumn{1}{l}{7} & 4.85714 & \multicolumn{1}{l}{SI} & LinStatic & -5.533 & 22.8967 \\
    \multicolumn{1}{l}{7} & 5.34286 & \multicolumn{1}{l}{SI} & LinStatic & -5.533 & 25.584 \\
    \multicolumn{1}{l}{7} & 5.82857 & \multicolumn{1}{l}{SI} & LinStatic & -5.533 & 28.2713 \\
    \multicolumn{1}{l}{7} & 6.31429 & \multicolumn{1}{l}{SI} & LinStatic & -5.533 & 30.9586 \\
    \multicolumn{1}{l}{7} & 6.8 & \multicolumn{1}{l}{SI} & LinStatic & -5.533 & 33.6459 \\
    \multicolumn{1}{l}{8} & 0   & \multicolumn{1}{l}{SI} & LinStatic & 28.695 & 45.4615 \\
    \multicolumn{1}{l}{8} & 0.5 & \multicolumn{1}{l}{SI} & LinStatic & 28.695 & 31.1139 \\
    \multicolumn{1}{l}{8} & 1   & \multicolumn{1}{l}{SI} & LinStatic & 28.695 & 16.7664 \\
    \multicolumn{1}{l}{8} & 1.5 & \multicolumn{1}{l}{SI} & LinStatic & 28.695 & 2.4188 \\
    \multicolumn{1}{l}{8} & 2   & \multicolumn{1}{l}{SI} & LinStatic & 28.695 & -11.9288 \\
    \multicolumn{1}{l}{8} & 2.5 & \multicolumn{1}{l}{SI} & LinStatic & 28.695 & -26.2763 \\
    \multicolumn{1}{l}{8} & 3   & \multicolumn{1}{l}{SI} & LinStatic & 28.695 & -40.6239 \\
    \multicolumn{1}{l}{8} & 3.5 & \multicolumn{1}{l}{SI} & LinStatic & 28.695 & -54.9715 \\
    \multicolumn{1}{l}{8} & 4   & \multicolumn{1}{l}{SI} & LinStatic & 28.695 & -69.319 \\
    \multicolumn{1}{l}{8} & 4.5 & \multicolumn{1}{l}{SI} & LinStatic & 28.695 & -83.6666 \\
    \multicolumn{1}{l}{8} & 5   & \multicolumn{1}{l}{SI} & LinStatic & 28.695 & -98.0142 \\
    \multicolumn{1}{l}{8} & 5.5 & \multicolumn{1}{l}{SI} & LinStatic & 28.695 & -112.3618 \\
    \multicolumn{1}{l}{8} & 6   & \multicolumn{1}{l}{SI} & LinStatic & 28.695 & -126.7093 \\
    \multicolumn{1}{l}{8} & 6.5 & \multicolumn{1}{l}{SI} & LinStatic & 28.695 & -141.0569 \\
    \multicolumn{1}{l}{8} & 7   & \multicolumn{1}{l}{SI} & LinStatic & 28.695 & -155.4045 \\
    \multicolumn{1}{l}{9} & 0   & \multicolumn{1}{l}{SI} & LinStatic & -46.253 & -211.727 \\
    \multicolumn{1}{l}{9} & 0.49286 & \multicolumn{1}{l}{SI} & LinStatic & -46.253 & -188.9308 \\
    \multicolumn{1}{l}{9} & 0.98571 & \multicolumn{1}{l}{SI} & LinStatic & -46.253 & -166.1347 \\
    \multicolumn{1}{l}{9} & 1.47857 & \multicolumn{1}{l}{SI} & LinStatic & -46.253 & -143.3386 \\
    \multicolumn{1}{l}{9} & 1.97143 & \multicolumn{1}{l}{SI} & LinStatic & -46.253 & -120.5424 \\
    \multicolumn{1}{l}{9} & 2.46429 & \multicolumn{1}{l}{SI} & LinStatic & -46.253 & -97.7463 \\
    \multicolumn{1}{l}{9} & 2.95714 & \multicolumn{1}{l}{SI} & LinStatic & -46.253 & -74.9501 \\
    \multicolumn{1}{l}{9} & 3.45 & \multicolumn{1}{l}{SI} & LinStatic & -46.253 & -52.154 \\
    \multicolumn{1}{l}{9} & 3.94286 & \multicolumn{1}{l}{SI} & LinStatic & -46.253 & -29.3578 \\
    \multicolumn{1}{l}{9} & 4.43571 & \multicolumn{1}{l}{SI} & LinStatic & -46.253 & -6.5617 \\
    \multicolumn{1}{l}{9} & 4.92857 & \multicolumn{1}{l}{SI} & LinStatic & -46.253 & 16.2345 \\
    \multicolumn{1}{l}{9} & 5.42143 & \multicolumn{1}{l}{SI} & LinStatic & -46.253 & 39.0306 \\
    \multicolumn{1}{l}{9} & 5.91429 & \multicolumn{1}{l}{SI} & LinStatic & -46.253 & 61.8268 \\
    \multicolumn{1}{l}{9} & 6.40714 & \multicolumn{1}{l}{SI} & LinStatic & -46.253 & 84.6229 \\
    \multicolumn{1}{l}{9} & 6.9 & \multicolumn{1}{l}{SI} & LinStatic & -46.253 & 107.4191 \\
        &     &     &     &     &  \\
        &     &     & Max & 28.695 & 107.4191 \\
        &     &     & Min & -46.253 & -211.727 \\
    \end{tabular}%
  \label{tab:addlabel}%
  }
\end{table}%

\end{document}