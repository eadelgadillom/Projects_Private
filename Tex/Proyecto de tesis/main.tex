\documentclass[12pt]{article}
\usepackage[utf8]{inputenc}
\usepackage{graphicx}
\usepackage{float}
\usepackage{booktabs,bigstrut}
\usepackage{multirow,multicol}
\usepackage{natbib}
\usepackage{amsmath,amssymb,amsfonts,latexsym,cancel}
\usepackage{tikz}
\usetikzlibrary{positioning}
\usepackage{pgfplots,caption}
\usepackage{comment}
\usepackage{pdflscape}
\usepackage{colortbl}
\usepackage{longtable}
\usepackage{xcolor}
\usepackage{rotating}
\usepackage{cancel}
\usepackage{parskip}
\usepackage{multicol}
\usepackage{hhline}
\usepackage{array}
\usepackage{longtable,lscape}
\usepackage[most]{tcolorbox}%Para colorear respuestas de ecuaciones
%\tcbset{enhanced,colframe=blue,colback={black!5!white},drop shadow}
%\tcboxmath[colframe=blue]{\mathbf{0.41\%}}


%All the packages the project needs
%--------------------------------------------
%Margins
\usepackage[top=2cm,
bottom=2cm,
left=1.6cm,
right=1.6cm,
headsep=15pt,
headheight=25pt,
letterpaper,
includehead,includefoot]{geometry} 
%-------------------------------------------

%-------------------------------------------
%Codification and language
%Document language
\usepackage[spanish,es-noshorthands]{babel}
% Required for including letters with accents
\usepackage[utf8]{inputenc} 
% Use 8-bit encoding that has 256 glyphs
\usepackage[T1]{fontenc}
%-------------------------------------------

%-------------------------------------------
%Colours
\usepackage{color}
\usepackage{xcolor}
%-------------------------------------------

%-------------------------------------------
%Fonts and text
%Add a new Font
\usepackage{addfont}
\addfont{OT1}{rsfs10}{\rsfs}
%Allows to modify the font of a section
\usepackage{sectsty}
%Latin Modern fonts - Allows to resize the font
\usepackage{lmodern}
%Allows to write code
\usepackage{verbatim}
%Dumb text
\usepackage{lipsum}
%Include currency symbols
\usepackage{textcomp} 
%-------------------------------------------

%-------------------------------------------
%Pictures and graphics
%Allows to include images
\usepackage{graphicx}
\usepackage{sidecap}
%Most powerful image package
\usepackage{tikz}
\usetikzlibrary{babel}
\usetikzlibrary{positioning}
%Sometimes used images packages

\usepackage{pst-all}
\usepackage{pstricks}
%Allows to position figures inside a paragraph
\usepackage{wrapfig}
%Information needed
%https://en.wikibooks.org/wiki/LaTeX/Floats,_Figures_and_Captions
%-------------------------------------------
%-------------------------------------------
%Math packages
\usepackage{mathtools}
\usepackage{amsmath}
\usepackage{amssymb}
\usepackage{amsfonts}
%Nice cancelations arrows
\usepackage{cancel}
%Vector arrows on a character
\usepackage{esvect}
%-------------------------------------------

%-------------------------------------------
%Table and listing packages
\usepackage{multirow}
\usepackage{array}
\usepackage{diagbox}
\usepackage{listings}
\usepackage[shortlabels]{enumitem} 
%Differents images and tables in the same space
\usepackage{subcaption}
%Use to places the float at precisely the location in the LaTeX code´and other stuff
\usepackage{float}
%Flexible tables
\usepackage{tabu}
\usepackage{multicol}
%-------------------------------------------

%-------------------------------------------
%Refs and cites
\usepackage[hidelinks]{hyperref}
%\usepackage{cite}
%-------------------------------------------

%-------------------------------------------
%Boxes
\usepackage{tcolorbox}
\usepackage{fancybox}
\usepackage{colortbl}
%-------------------------------------------

%-------------------------------------------
%Page style
\usepackage[usestackEOL]{stackengine}
\usepackage{fancyhdr}
\pagestyle{fancy}
\newsavebox{\UNl}
\newsavebox{\UNd}
\setlength{\headheight}{2.5cm}
\lhead{\usebox{\UNl}\usebox{\UNd}}
\rhead{\slshape Proyecto de Tesis}
\cfoot{}
\rfoot{\thepage}
%\lfoot{\today}
\renewcommand{\headrulewidth}{1pt}

%-------------------------------------------

%-------------------------------------------
%others
\usepackage{pgfplots,caption}
\usepackage{comment}
\usepackage{pdflscape}
\usepackage{rotating}
\usepackage{parskip}
\usepackage{hhline}
\usepackage{longtable,lscape}
\usepackage{booktabs,bigstrut}

\title{TALLER1}
\author{Juan David Torres}
\date{\today}
%Inicio del documento
\pgfplotsset{compat=1.17}
\begin{document}
\renewcommand{\tablename}{Tabla}%% Para cambiar "Cuadro" a "Tabla"
\renewcommand{\listtablename}{Índice de tablas}
%Inicio de la portada
\begin{titlepage}
    \thispagestyle{empty}
    \begin{center}
        \begin{figure}
        \centering%
        \includegraphics{images/EscudoUN.png}
    \end{figure}
    
    \vspace{1cm}
    
        \textbf{PROYECTO DISEÑO ESTRUCTURAL\\MEMORIAS DE CÁLCULO}\\[1.3in]
      
    JUAN MANUEL CÓRDOBA PEÑA \\  
    JUAN DAVID TORRES TURRIAGO \\
    EDUARDO ANDRÉS DELGADILLO MONSALVE \\  [1.5in]
 Trabajo presentado como requisito parcial de la asignatura\\  DISEÑO ESTRUCTURAL  al docente:\\[3mm] \textbf {\large{Ing. Ismael Santana Santana}}\\[1.5in]
   \textbf{UNIVERSIDAD NACIONAL DE COLOMBIA}\\
   FACULTAD DE INGENIERÍA CIVIL Y AGRÍCOLA\\
   BOGOTÁ D.C.\\
   \today
    \end{center}
\end{titlepage}

\newpage
\thispagestyle{empty}
\tableofcontents
\thispagestyle{empty}
\newpage

% \listoffigures

% \listoftables

\newpage

\setcounter{page}{1}
\section{Descripción del proyecto}

El proyecto se encuentra ubicado en la ciudad de Bogotá D.C en la microzonificación sísmica de "Lacustre 100". Consiste en una edificación  de cuatro pisos con una altura libre entre losas de 2.5 metros, que tiene un sistema estructural de pórtico de concreto reforzado resistente a momentos con capacidad moderada de disipación de energía (DMO) y cuyo tipo de uso es vivienda. La losa de entrepiso y de cubierta es aligerada y armada en una dirección.\\

Para encontrar el espesor mínimo de la losa se utilizaron las alturas promedio dadas por las tablas C.R.9.5 y C.9.5(a) de la NSR-10, para no incurrir en un costo alto de losa, obteniéndose un valor de $\mathbf{h_{losa}=0.5m}$. De acuerdo a C.8.13.2 se determina un ancho de viguetas de \textbf{0.10 m} y a partir de C.8.13.3 se encuentra una separación entre viguetas que no supere 1.2 m. Se colocan riostras transversales a las viguetas de manera tal que su separación no exceda 4.0 m de acuerdo con C.8.13.3.1. A partir de C.8.13.6, se determina un espesor de plaqueta superior de \textbf{0.05 m} que se vaciara en un casetón removible.\\


\section{Definición de la carga viva}

Para el cálculo de la carga viva en la estructura se utiliza el valor indicado en la norma NSR-10, específicamente en la \textit{tabla B.4.2.1.1-Cargas vivas  uniformemente distribuidas}, con el cual se obtiene que el valor de la carga viva corresponde a $ \mathbf{1.8 \tfrac{kN}{m^{2}}}$ para el uso de la edificación (residencial). Adicionalmente, de acuerdo con la norma, se establece una carga de $ \mathbf{3.0 \tfrac{kN}{m^{2}}}$ para la escalera. Con respecto a la cubierta se toma una carga viva mínima de $ \mathbf{1.8 \tfrac{kN}{m^{2}}}$ de acuerdo con la tabla B.4.2.1-2. Para el ascensor se considera una carga viva de $ \mathbf{20 \tfrac{kN}{m^{2}}}$  en donde se tiene en cuenta la carga de impacto.\\ 

\section{Evaluación de las cargas muertas}

La evaluación de la carga muerta se realiza teniendo en cuenta que el material que se utilizará para construir la estructura es el concreto armado, cuyo peso por unidad de volumen corresponde a 24 $\frac{kN}{m^{2}}$ de acuerdo con la tabla B.3.2-1 de la NSR-10.\\

\subsection{Evaluación carga muerta piso tipo}

Teniendo en cuenta que el material de la estructura consiste en concreto armado, se calcula la carga muerta del piso tipo por cada  elementos que compone a la losa aligerada en una dirección: la plaqueta, viguetas, riostras y el casetón. Esto se realiza teniendo en cuenta el peso de los muros divisorios, los acabados y las instalaciones que reposan sobre cada piso, pues con este se realiza el aligeramiento. A continuación, se presenta una tabla resumen de cada elemento y el peso por metro cuadrado con su correspondiente operación.



% Table generated by Excel2LaTeX from sheet 'Eval. Cargas'
\begin{table}[H]
  \centering
  
    \begin{tabular}{cc|r|c|}
    \rowcolor[rgb]{ .2,  .247,  .31} \multicolumn{4}{|c}{\textcolor[rgb]{ 1,  1,  1}{\textbf{CARGAS MUERTAS}}} \bigstrut[b]\\
    \hline
    \rowcolor[rgb]{ .2,  .247,  .31} \multicolumn{2}{|c|}{\textcolor[rgb]{ 1,  1,  1}{Elemento}} & \multicolumn{1}{c|}{\textcolor[rgb]{ 1,  1,  1}{Operación }} & \textcolor[rgb]{ 1,  1,  1}{Resultado} \bigstrut[b]\\
    \hline
    \rowcolor[rgb]{ .2,  .247,  .31} \multicolumn{2}{|c|}{\textcolor[rgb]{ 1,  1,  1}{Plaqueta [kN/m²]}} & \multicolumn{1}{c|}{\cellcolor[rgb]{ 1,  1,  1}0.05 $m\cdot $ 24$\frac{kN}{m^{3}}$} & \cellcolor[rgb]{ 1,  1,  1}1.2 \bigstrut\\
    \hline
    \rowcolor[rgb]{ .2,  .247,  .31} \multicolumn{2}{|c|}{\textcolor[rgb]{ 1,  1,  1}{Viguetas [kN/m²]}} & \multicolumn{1}{c|}{\cellcolor[rgb]{ 1,  1,  1}$\dfrac{0.1\;m\cdot0.45\;m\cdot 24\frac{kN}{m^{3}}}{ 0.87\; m}$} & \cellcolor[rgb]{ 1,  1,  1}1.24 \bigstrut\\
    \hline
    \rowcolor[rgb]{ .2,  .247,  .31} \multicolumn{2}{|c|}{\textcolor[rgb]{ 1,  1,  1}{Riostras [kN/m²]}} & \multicolumn{1}{c|}{\cellcolor[rgb]{ 1,  1,  1}$\dfrac{0.1\;m\cdot0.45\;m\cdot 24\frac{kN}{m^{3}}}{6.95\;m}$} & \cellcolor[rgb]{ 1,  1,  1}0.16 \bigstrut\\
    \hline
    \rowcolor[rgb]{ .2,  .247,  .31} \multicolumn{2}{|c|}{\textcolor[rgb]{ 1,  1,  1}{Casetón[kN/m²]}} & \multicolumn{1}{c|}{\cellcolor[rgb]{ 1,  1,  1}0.35} & \cellcolor[rgb]{ 1,  1,  1}0.35 \bigstrut\\
    \hline
    \rowcolor[rgb]{ .2,  .247,  .31} \multicolumn{2}{|c|}{\textcolor[rgb]{ 1,  1,  1}{Muros (residencial)[kN/m²]}} & \multicolumn{1}{c|}{\cellcolor[rgb]{ 1,  1,  1}3} & \cellcolor[rgb]{ 1,  1,  1}3 \bigstrut\\
    \hline
    \rowcolor[rgb]{ .2,  .247,  .31} \multicolumn{2}{|c|}{\textcolor[rgb]{ 1,  1,  1}{Acabado superior}} & \multicolumn{1}{c|}{\cellcolor[rgb]{ 1,  1,  1}$0.05\;m \cdot 21 \dfrac{kN}{m^3}$} & \cellcolor[rgb]{ 1,  1,  1}1.05 \bigstrut\\
    \hline
    \rowcolor[rgb]{ .2,  .247,  .31} \multicolumn{2}{|c|}{\textcolor[rgb]{ 1,  1,  1}{Acabado inferior}} & \multicolumn{1}{c|}{\cellcolor[rgb]{ 1,  1,  1}0.3} & \cellcolor[rgb]{ 1,  1,  1}0.3 \bigstrut\\
    \hline
    \rowcolor[rgb]{ .2,  .247,  .31} \multicolumn{2}{|c|}{\textcolor[rgb]{ 1,  1,  1}{Instalaciones}} & \multicolumn{1}{c|}{\cellcolor[rgb]{ 1,  1,  1}0.2} & \cellcolor[rgb]{ 1,  1,  1}0.2 \bigstrut\\
    \hline
    \rowcolor[rgb]{ .2,  .247,  .31} \multicolumn{2}{c|}{\textcolor[rgb]{ 1,  1,  1}{\textbf{TOTAL}}} & \cellcolor[rgb]{ 1,  1,  1} & \cellcolor[rgb]{ 1,  1,  1}7.50 \bigstrut\\
\cline{3-4}    \end{tabular}%
    \caption{Evaluación cargas muertas piso tipo}
  \label{tab:CMPT}%
\end{table}%




Sumando estos elementos tenemos los siguientes resultados por el piso tipo resumidos en la siguiente tabla, calculando la carga última como el resultado mayor entre $1.2D+1.6L$ y $1.4D$, el factor de carga como la división de la carga última entre la carga total.

% Table generated by Excel2LaTeX from sheet 'Eval. Cargas'
\begin{table}[H]
  \centering
  
    \begin{tabular}{|c|c|c|c|}
    \hline
    \rowcolor[rgb]{ .2,  .247,  .31} \multicolumn{2}{|c|}{\textcolor[rgb]{ 1,  1,  1}{\textbf{CARGA MUERTA (D) [kN/m²]}}} & \cellcolor[rgb]{ 1,  1,  1} ver tabla \ref{tab:CMPT} & \cellcolor[rgb]{ 1,  1,  1}7.5 \bigstrut\\
    \hline
    \rowcolor[rgb]{ .2,  .247,  .31} \multicolumn{2}{|c|}{\textcolor[rgb]{ 1,  1,  1}{\textbf{CARGA VIVA (L) [kN/m²]}}} & \cellcolor[rgb]{ 1,  1,  1}(tomado de la norma) & \cellcolor[rgb]{ 1,  1,  1}1.8 \bigstrut\\
    \hline
    \rowcolor[rgb]{ .2,  .247,  .31} \multicolumn{2}{|c|}{\textcolor[rgb]{ 1,  1,  1}{\textbf{CARGA TOTAL [kN/m²]}}} & \cellcolor[rgb]{ 1,  1,  1}$7.5 \tfrac{kN}{m^2} + 1.8  \tfrac{kN}{m^2}$  & \cellcolor[rgb]{ 1,  1,  1}9.30\bigstrut\\
    \hline
    \rowcolor[rgb]{ .2,  .247,  .31} \multicolumn{2}{|c|}{\textcolor[rgb]{ 1,  1,  1}{\textbf{CARGA ÚLTIMA [kN/m²]}}} & \cellcolor[rgb]{ 1,  1,  1}$1.2 \cdot7.5 \tfrac{kN}{m^2} + 1.6 \cdot 1.8  \tfrac{kN}{m^2}$ & \cellcolor[rgb]{ 1,  1,  1}11.88 \bigstrut\\
    \hline
    \rowcolor[rgb]{ .2,  .247,  .31} \multicolumn{2}{|c|}{\textcolor[rgb]{ 1,  1,  1}{\textbf{FACTOR DE CARGA}}} & \cellcolor[rgb]{ 1,  1,  1}$\dfrac{11.88}{9.30}$ & \cellcolor[rgb]{ 1,  1,  1}1.28 \bigstrut\\
    \hline
    \end{tabular}%
    \caption{Resultados de la carga del piso tipo}
  \label{tab:Resultados de la carga del piso tipo}%
\end{table}%



\subsection{Evaluación carga muerta de cubierta}

De manera análoga a la evaluación de carga muerta del piso tipo se realiza la evaluación de cargas de la cubierta, teniendo en cuenta los elementos que componen la losa aligerada en una dirección: la plaqueta, las viguetas, riostras y el casetón. Adicionalmente en este caso se aplicará un impermeabilizante, para evitar la exposición a la intemperie de los elementos internos de la edificación.   El material escogido es la tela asfáltica, que tiene un valor de $0.03\: kN/m^2$ de acuerdo a la tabla B.3.4.1-4.\\

Se realiza un análisis de los muros perimetrales de la edificación teniendo en cuenta la geometría de la misma. El peso de los muros corresponde al producto entre el peso que aportan los muros exteriores y la razón entre al área vertical y el área de la losa. Se toman muros exteriores con enchape en ladrillo tomando el valor de la tabla B.3.4.2-4, el cual corresponde a un valor de $2.5 kN/m^2$ por área vertical y para los muros internos se tomó mampostería pañetada de espesor de 15 cm que tiene el mismo peso de los muros perimetrales de acuerdo a la norma NSR-10. A continuación, se presenta una tabla resumen de cada elemento y el peso por metro cuadrado con su correspondiente operación.

% Table generated by Excel2LaTeX from sheet 'Eval. Cargas'
\begin{table}[H]
  \centering
 
    \begin{tabular}{cc|r|c|}
    \rowcolor[rgb]{ .2,  .247,  .31} \multicolumn{4}{|c}{\textcolor[rgb]{ 1,  1,  1}{\textbf{CARGAS MUERTAS}}} \bigstrut[b]\\
    \hline
    \rowcolor[rgb]{ .2,  .247,  .31} \multicolumn{2}{|c|}{\textcolor[rgb]{ 1,  1,  1}{Elemento}} & \multicolumn{1}{c|}{\textcolor[rgb]{ 1,  1,  1}{Operación }} & \textcolor[rgb]{ 1,  1,  1}{Resultado} \bigstrut[b]\\
    \hline
    \rowcolor[rgb]{ .2,  .247,  .31} \multicolumn{2}{|c|}{\textcolor[rgb]{ 1,  1,  1}{Plaqueta [kN/m²]}} & \multicolumn{1}{c|}{\cellcolor[rgb]{ 1,  1,  1}0.05 $m\cdot $ 24$\frac{kN}{m^{3}}$} & \cellcolor[rgb]{ 1,  1,  1}1.2 \bigstrut\\
    \hline
    \rowcolor[rgb]{ .2,  .247,  .31} \multicolumn{2}{|c|}{\textcolor[rgb]{ 1,  1,  1}{Viguetas [kN/m²]}} & \multicolumn{1}{c|}{\cellcolor[rgb]{ 1,  1,  1}$\dfrac{0.1\;m\cdot0.45\;m\cdot 24\frac{kN}{m^{3}}}{ 0.87\; m}$} & \cellcolor[rgb]{ 1,  1,  1}1.24 \bigstrut\\
    \hline
    \rowcolor[rgb]{ .2,  .247,  .31} \multicolumn{2}{|c|}{\textcolor[rgb]{ 1,  1,  1}{Riostras [kN/m²]}} & \multicolumn{1}{c|}{\cellcolor[rgb]{ 1,  1,  1}$\dfrac{0.1\;m\cdot0.45\;m\cdot 24\frac{kN}{m^{3}}}{6.95\;m}$} & \cellcolor[rgb]{ 1,  1,  1}0.16\bigstrut\\
    \hline
    \rowcolor[rgb]{ .2,  .247,  .31} \multicolumn{2}{|c|}{\textcolor[rgb]{ 1,  1,  1}{Casetón[kN/m²]}} & \multicolumn{1}{c|}{\cellcolor[rgb]{ 1,  1,  1}0.35} & \cellcolor[rgb]{ 1,  1,  1}0.35 \bigstrut\\
    \hline
    \rowcolor[rgb]{ .2,  .247,  .31} \multicolumn{2}{|c|}{\textcolor[rgb]{ 1,  1,  1}{Muros (perimetrales y de vacíos) [kN/m²]}} & \multicolumn{1}{c|}{\cellcolor[rgb]{ 1,  1,  1}$2.5\tfrac{kN}{m^2} \cdot\frac{(82.2\;m + 6.54\;m)\cdot 1.5\;m}{397.48\; m^2}$} & \cellcolor[rgb]{ 1,  1,  1}0.84 \bigstrut\\
    \hline
    \rowcolor[rgb]{ .2,  .247,  .31} \multicolumn{2}{|c|}{\textcolor[rgb]{ 1,  1,  1}{Pendientado}} & \multicolumn{1}{c|}{\cellcolor[rgb]{ 1,  1,  1}$0.05\;m \cdot 21 \dfrac{kN}{m^3}$} & \cellcolor[rgb]{ 1,  1,  1}1.05 \bigstrut\\
    \hline
    \rowcolor[rgb]{ .2,  .247,  .31} \multicolumn{2}{|c|}{\textcolor[rgb]{ 1,  1,  1}{Acabados inferior}} & \multicolumn{1}{c|}{\cellcolor[rgb]{ 1,  1,  1}0.3} & \cellcolor[rgb]{ 1,  1,  1}0.3 \bigstrut\\
    \hline
    \rowcolor[rgb]{ .2,  .247,  .31} \multicolumn{2}{|c|}{\textcolor[rgb]{ 1,  1,  1}{Acabados (impermeabilización)}} & \multicolumn{1}{c|}{\cellcolor[rgb]{ 1,  1,  1}0.03} & \cellcolor[rgb]{ 1,  1,  1}0.03 \bigstrut\\
    \hline
    \rowcolor[rgb]{ .2,  .247,  .31} \multicolumn{2}{|c|}{\textcolor[rgb]{ 1,  1,  1}{Instalaciones}} & \multicolumn{1}{c|}{\cellcolor[rgb]{ 1,  1,  1}0.2} & \cellcolor[rgb]{ 1,  1,  1}0.2 \bigstrut\\
    \hline
    \rowcolor[rgb]{ .2,  .247,  .31} \multicolumn{2}{c|}{\textcolor[rgb]{ 1,  1,  1}{\textbf{TOTAL}}} & \cellcolor[rgb]{ 1,  1,  1} & \cellcolor[rgb]{ 1,  1,  1}5.37 \bigstrut\\
\cline{3-4}    \end{tabular}%
 \caption{Evaluación de cargas muertas de cubierta}
  \label{tab:EvaDC}%
\end{table}%





De igual manera, se suman todos los elementos, se encuentra el valor de la carga última y el factor de carga.



% Table generated by Excel2LaTeX from sheet 'Eval. Cargas'
\begin{table}[H]
  \centering

    \begin{tabular}{|c|c|c|c|}
    
    \rowcolor[rgb]{ .2,  .247,  .31} \multicolumn{2}{|c|}{\textcolor[rgb]{ 1,  1,  1}{Elemento}} & \multicolumn{1}{c|}{\textcolor[rgb]{ 1,  1,  1}{Operación }} & \textcolor[rgb]{ 1,  1,  1}{Resultado} \bigstrut[b]\\
    \hline
    \rowcolor[rgb]{ .2,  .247,  .31} \multicolumn{2}{|c|}{\textcolor[rgb]{ 1,  1,  1}{\textbf{CARGA MUERTA (D) [kN/m²]}}} & \cellcolor[rgb]{ 1,  1,  1} Ver tabla \ref{tab:EvaDC} & \cellcolor[rgb]{ 1,  1,  1}5.37 \bigstrut\\
    \hline
    \rowcolor[rgb]{ .2,  .247,  .31} \multicolumn{2}{|c|}{\textcolor[rgb]{ 1,  1,  1}{\textbf{CARGA VIVA (L) [kN/m²]}}} & \cellcolor[rgb]{ 1,  1,  1} Tomado de la norma & \cellcolor[rgb]{ 1,  1,  1}1.8 \bigstrut\\
    \hline
    \rowcolor[rgb]{ .2,  .247,  .31} \multicolumn{2}{|c|}{\textcolor[rgb]{ 1,  1,  1}{\textbf{CARGA TOTAL [kN/m²]}}} & \cellcolor[rgb]{ 1,  1,  1}5.37 $\tfrac{kN}{m^2}$+1.8 $\frac{kN}{m^2}$ & \cellcolor[rgb]{ 1,  1,  1}7.17 \bigstrut\\
    \hline
    \rowcolor[rgb]{ .2,  .247,  .31} \multicolumn{2}{|c|}{\textcolor[rgb]{ 1,  1,  1}{\textbf{CARGA ÚLTIMA [kN/m²]}}} & \cellcolor[rgb]{ 1,  1,  1}$1.2\cdot5.37$ $\tfrac{kN}{m^2}$+$1.6\cdot1.8$ $\tfrac{kN}{m^2}$ & \cellcolor[rgb]{ 1,  1,  1}9.32 \bigstrut\\
    \hline
    \rowcolor[rgb]{ .2,  .247,  .31} \multicolumn{2}{|c|}{\textcolor[rgb]{ 1,  1,  1}{\textbf{FACTOR DE CARGA}}} & \cellcolor[rgb]{ 1,  1,  1}$\dfrac{9.32}{7.17}$ & \cellcolor[rgb]{ 1,  1,  1}1.30 \bigstrut\\
    \hline
    \end{tabular}%
      \caption{Resultados de la carga de cubierta}
  \label{tab:RDR}%
\end{table}%


\subsection{Evaluación de cargas de la escalera}


Partiendo de un ancho de huella de $0.28\;m$ se determina la altura de la contrahuella teniendo en cuenta los requisitos de K.3.8.3.4. Se establecen 18 escalones con una altura de contrahuella de $170\;mm$. La longitud del tramo inclinado es de $2.24\;m$ y la longitud del descanso de $1.2\;m$\\


Teniendo en cuenta las consideraciones del inciso C.9.5, el espesor de la losa de la escalera se calcula teniendo en cuenta que esta es simplemente apoyada, por lo que se toma un espesor de: $l/20$
\begin{equation}
    t=\frac{l}{20}=\frac{3.44\;m}{20}=0.172\;m \Rightarrow 0.15\;m
\end{equation}
Este valor se reduce a $0.15\;m$ \\ %justificar
Para encontrar la pendiente de la losa se toma la suma de las huellas ($2.24\;m$) y la altura de la contrahuella sin tomar la última ($1.33\;m$)

\begin{equation}
    \theta=tan^{-1}\left(\frac{1.33}{2.24}\right)=30.70^{\circ}
\end{equation}


\begin{table}[H]
\centering

\begin{tabular}{|c|c|c|} 
\hline
\multicolumn{3}{|c|}{{\cellcolor[rgb]{0.227,0.227,0.227}}\textcolor{white}{\textbf{EVALUACIÓN CARGAS TRAMO INCLINADO}}}                                 \\ 
\hline
\rowcolor[rgb]{0.227,0.227,0.227} \textcolor{white}{\textbf{Elemento}} & \textcolor{white}{\textbf{Operación}} & \textcolor{white}{\textbf{Resultado}}  \\ 
\hline
{\cellcolor[rgb]{1 ,1 ,1}}Peso propio losa
  kN/m²           &   $\frac{0.15\;m \cdot 24 \tfrac{KN}{m^3}}{\cos{30.70^{\circ}}}$                                  & 4.19                                 \\ 
\hline
{\cellcolor[rgb]{1,1,1}}Peso de peldaños
  kN/m²           & $\frac{0.17\;m \cdot 24 \tfrac{KN}{m^3}}{2}$                                     & 2.04                                 \\ 
\hline  
{\cellcolor[rgb]{1,1,1}}Acabado (baldosa 25
  mm) kN/m²    &  $\frac{0.28\;m + 0.17\;m}{0.28\;m}\cdot 1.1\tfrac{KN}{m^2}$                                    & 1.77                                   \\ 
\hline
{\cellcolor[rgb]{1,1,1}}Cielo raso falso kN/m²                   & $\frac{0.3}{\cos{30.70^{\circ}}}$                                      & 0.35                                   \\ 
\hline
{\cellcolor[rgb]{1,1,1}}Subtotal (Carga
  muerta~ kN/m²)   & $\mathbf{\sum}$                                    & \textbf{8.34  }                                 \\ 
\hline
{\cellcolor[rgb]{1,1,1}}Carga viva kN/m²                   & (Tomado de la norma)                                     & \textbf{3.00}                                   \\ 
\hline
{\cellcolor[rgb]{1,1,1}}\textbf{TOTAL}                     & ~                                     & \textbf{11.34}                         \\
\hline
\end{tabular}
\caption{Evaluación de cargas del tramo inclinado de la escalera}
\label{tab:evalINCLIN}
\end{table}

Con las dimensiones de contrahuella, huella y pendiente se analizan las cargas en $kN/m^2$ para el tramo inclinado,dando como resultado una carga muerta de: $\mathbf{8.34\; \tfrac{kN}{m^2}}$.\\

De acuerdo con la tabla B.4.2.1-1 de la NSR-10 se toma una carga viva para la la escalera de  $\mathbf{3.0 \tfrac{KN}{m^2}}$ \\

El procedimiento mostrado en la tabla \ref{tab:evalINCLIN} se realiza de la misma manera para el descanso de la escalera obteniendo una carga muerta de $\mathbf{5.00\tfrac{KN}{m^2}}$ 

\begin{table}[H]
\centering

\begin{tabular}{|c|c|c|} 
\hline
\multicolumn{3}{|c|}{{\cellcolor[rgb]{ .2,  .247,  .31}}\textcolor{white}{\textbf{EVALUACIÓN CARGAS DESCANSO}}}                                        \\ 
\hline
\rowcolor[rgb]{ .2,  .247,  .31} \textcolor{white}{\textbf{Elemento}} & \textcolor{white}{\textbf{Operación}} & \textcolor{white}{\textbf{Resultado}}  \\ 
\hline
{\cellcolor[rgb]{1,1,1}}Peso propio losa
  kN/m²        & $24\tfrac{KN}{m^3} \cdot 0.15\;m$                                     & 3.60                                   \\ 
\hline
{\cellcolor[rgb]{1,1,1}}Acabado (baldosa 25
  mm) kN/m² & (Tomado de la norma)                                     & 1.10                                   \\ 
\hline
{\cellcolor[rgb]{1,1,1}}Cielo raso falso                    & (Tomado de la norma)                                       & 0.30                                   \\ 
\hline
{\cellcolor[rgb]{1,1,1}}Subtotal (Carga
  muerta kN/m²) & $\mathbf{\sum}$                                     & \textbf{5.00 }                                  \\ 
\hline
{\cellcolor[rgb]{1,1,1}}Carga viva                      & (tomado de la norma)                                     & \textbf{3.00  }                                 \\ 
\hline
{\cellcolor[rgb]{1,1,1}}\textbf{TOTAL}                  & ~                                     & \textbf{8.00   }                                \\
\hline
\end{tabular}
\caption{Evaluación de cargas del descanso de la escalera }
\label{tab:Evaldescanso}
\end{table}




\subsection{Evaluación de cargas de la losa del ascensor}

Se realiza una evaluación del peso que ascensor transmite a la estructura, mostrando tanto el peso muerto como la carga viva sobre la losa que incluye el efecto de impacto. Dicha operación se resume en la siguiente tabla

% Table generated by Excel2LaTeX from sheet 'Eval. Cargas'
\begin{table}[H]
  \centering
    \begin{tabular}{|c|c|c|c|}
    \rowcolor[rgb]{ .2,  .247,  .31} \multicolumn{2}{|c|}{\textcolor[rgb]{ 1,  1,  1}{Elemento}} & \multicolumn{1}{c|}{\textcolor[rgb]{ 1,  1,  1}{Operación }} & \multicolumn{1}{c|}{\textcolor[rgb]{ 1,  1,  1}{Resultado}} \bigstrut[b]\\
    \hline
    \rowcolor[rgb]{ .2,  .247,  .31} \multicolumn{2}{|c|}{\textcolor[rgb]{ 1,  1,  1}{Peso Losa maciza [kN/m²]}} & \cellcolor[rgb]{ 1,  1,  1}$0.2~m\cdot24\tfrac{kN}{m^{3}}$ & \cellcolor[rgb]{ 1,  1,  1}4.8 \bigstrut\\
    \hline
    \rowcolor[rgb]{ .2,  .247,  .31} \multicolumn{2}{|c|}{\textcolor[rgb]{ 1,  1,  1}{Acabado [kN/m²]}} & \cellcolor[rgb]{ 1,  1,  1}$0.05\cdot21\tfrac{kN}{m^{3}}$ & \cellcolor[rgb]{ 1,  1,  1}1.05 \bigstrut\\
    \hline
  
    \rowcolor[rgb]{ .2,  .247,  .31} \multicolumn{2}{|c|}{\textcolor[rgb]{ 1,  1,  1}{TOTAL [kN/m²]}} & \cellcolor[rgb]{ 1,  1,  1}$(4.8+1.05)\tfrac{kN}{m^{2}}$ & \cellcolor[rgb]{ 1,  1,  1}5.85 \bigstrut\\
    \hline
    \rowcolor[rgb]{ .2,  .247,  .31} \multicolumn{2}{|c|}{\textcolor[rgb]{ 1,  1,  1}{Carga viva sobre la losa [kN/m²]}} & \cellcolor[rgb]{ 1,  1,  1}(incluye efecto de impacto) & \cellcolor[rgb]{ 1,  1,  1}20 \bigstrut\\
    \hline
    \end{tabular}%
  \caption{Peso de la losa del ascensor}
  \label{tab:PesoAscensor}%
\end{table}%



\subsection{Peso del piso tipo}

Para calcular el peso del piso tipo se utiliza el resultado de la evaluación de carga muerta del piso tipo junto con el área resultante de donde actúa, la cual se resume en la siguiente tabla.

% Table generated by Excel2LaTeX from sheet 'W_Edificio'
\begin{table}[H]
  \centering
    \begin{tabular}{|c|c|c|}
    \hline
    \rowcolor[rgb]{ .2,  .247,  .31} \multicolumn{3}{|c|}{\textcolor[rgb]{ 1,  1,  1}{\textbf{Áreas}}} \bigstrut\\
    \hline
    \rowcolor[rgb]{ .2,  .247,  .31} \multicolumn{1}{|c|}{\textcolor[rgb]{ 1,  1,  1}{Elemento}} & \multicolumn{1}{c|}{\textcolor[rgb]{ 1,  1,  1}{Operación }} & \multicolumn{1}{c|}{\textcolor[rgb]{ 1,  1,  1}{Resultado $m^{2}$}} \bigstrut[b]\\
    \hline
    \rowcolor[rgb]{ .2,  .247,  .31} \textcolor[rgb]{ 1,  1,  1}{\textbf{Área inicial}} & \cellcolor[rgb]{ 1,  1,  1}$(21.1\cdot17.95+6.1\cdot1.8+0.8\cdot5.6+1.25\cdot7.5)~m^{2}$ & \cellcolor[rgb]{ 1,  1,  1}403.58 \bigstrut\\
    \hline
    \rowcolor[rgb]{ .2,  .247,  .31} \textcolor[rgb]{ 1,  1,  1}{\textbf{Área vacío}} & \cellcolor[rgb]{ 1,  1,  1}$(5.5\cdot5.5)~m^{2}$ & \cellcolor[rgb]{ 1,  1,  1}30.25 \bigstrut\\
    \hline
    \rowcolor[rgb]{ .2,  .247,  .31} \textcolor[rgb]{ 1,  1,  1}{\textbf{Área ascensor}} & \cellcolor[rgb]{ 1,  1,  1}$(1.75\cdot1.75)~m^{2}$ & \cellcolor[rgb]{ 1,  1,  1}3.0625 \bigstrut\\
    \hline
    \rowcolor[rgb]{ .2,  .247,  .31} \textcolor[rgb]{ 1,  1,  1}{\textbf{Área escalera}} & \cellcolor[rgb]{ 1,  1,  1}$(2.4\cdot3.44)~m^{2}$ & \cellcolor[rgb]{ 1,  1,  1}8.256 \bigstrut\\
    \hline
    \rowcolor[rgb]{ .2,  .247,  .31} \textcolor[rgb]{ 1,  1,  1}{\textbf{Área neta}} & \cellcolor[rgb]{ 1,  1,  1}$(403.58-30.25-3.0625-8.256)~m^{2}$ & \cellcolor[rgb]{ 1,  1,  1}362.012 \bigstrut\\
    \hline
    \end{tabular}%
  \caption{Áreas consideradas para el peso del piso tipo}
  \label{tab:ÁreasPisoTipo}%
\end{table}%

Además del área calculada se tiene en cuenta el peso de las columnas  y el peso de las vigas el cual se calcula con las dimensiones del plano, de la siguiente manera:

\begin{gather*}
    \mathbf{W_{vigas}}=[(5.7+6.75)\cdot0.4+(1.25+5.7+5.5+6.75)\cdot0.5\cdot2+(5.7+5.5+6.75)\cdot0.4\\ +((6.75+7.5+6.85)\cdot4+1.8+1.8+0.8-(0.8+1)\cdot4)\cdot0.3]~m^{2}\cdot24\tfrac{kN}{m^{3}}\cdot0.45m=\mathbf{603.072~kN}
\end{gather*}

\begin{equation*}
    \mathbf{W_{col}}=[0.7\cdot0.6~m^{2}\cdot12+0.6\cdot0.5~m^{2}\cdot4]\cdot 24 \tfrac{kN}{m^{3}} \cdot 2.5 m=\mathbf{374.4~kN}
\end{equation*}
Finalmente, el peso de la escalera se calcula como se muestra a continuación, teniendo en cuenta tanto el tramo inclinado como el descanso
\begin{equation*}
    \mathbf{W_{escalera}}=(2.24\;m \cdot 2.4\;m )\cdot 8.344 \tfrac{kN}{m^2}+(1.2\;m \cdot 2.4\;m)\cdot 5.0 \tfrac{kN}{m^2} =\mathbf{59.255\;kN}
\end{equation*}
Teniendo esto en cuenta se calcula el peso del piso tipo con la siguiente ecuación

\begin{gather*}
    \mathbf{W}=Area\cdot D_{PT}+W_{col}+W_{vigas}+W_{escalera}\\
    \mathbf{W}=362.012~m^{2}\cdot 7.5\tfrac{kN}{m^{2}}+603.072~kN+374.4~kN+59.255~kN=\mathbf{3752.37~kN}
\end{gather*}

\subsection{Peso de la cubierta}

De la misma forma se calcula el peso de la cubierta teniendo en cuenta las diferencias en planta mostrados en los planos, y considerando el peso de la losa del ascensor

% Table generated by Excel2LaTeX from sheet 'W_Edificio'
\begin{table}[H]
  \centering
    \begin{tabular}{|c|c|c|}
    \hline
    \rowcolor[rgb]{ .2,  .247,  .31} \multicolumn{3}{|c|}{\textcolor[rgb]{ 1,  1,  1}{\textbf{Áreas}}} \bigstrut\\
    \hline
    \rowcolor[rgb]{ .2,  .247,  .31} \multicolumn{1}{|c|}{\textcolor[rgb]{ 1,  1,  1}{Elemento}} & \multicolumn{1}{c|}{\textcolor[rgb]{ 1,  1,  1}{Operación }} & \multicolumn{1}{c|}{\textcolor[rgb]{ 1,  1,  1}{Resultado $m^{2}$}} \bigstrut[b]\\
    \hline
    \rowcolor[rgb]{ .2,  .247,  .31} \textcolor[rgb]{ 1,  1,  1}{\textbf{Área inicial}} & \cellcolor[rgb]{ 1,  1,  1}$(21.1\cdot17.95+6.1\cdot0.8+0.8\cdot5.6+1.25\cdot7.5)~m^{2}$ & \cellcolor[rgb]{ 1,  1,  1}397.48 \bigstrut\\
    \hline
    \rowcolor[rgb]{ .2,  .247,  .31} \textcolor[rgb]{ 1,  1,  1}{\textbf{Área ascensor}} & \cellcolor[rgb]{ 1,  1,  1}$(1.75\cdot1.75)~m^{2}$ & \cellcolor[rgb]{ 1,  1,  1}3.0625 \bigstrut\\
    \hline
    \rowcolor[rgb]{ .2,  .247,  .31} \textcolor[rgb]{ 1,  1,  1}{\textbf{Área neta}} & \cellcolor[rgb]{ 1,  1,  1}$(397.48-3.0625-4.8)~m^{2}$ & \cellcolor[rgb]{ 1,  1,  1}389.6175 \bigstrut\\
    \hline
    \end{tabular}%
  \caption{Áreas consideradas para el peso de cubierta}
  \label{tab:ÁreasPisoCubierta}%
\end{table}%

Además del área calculada se tiene en cuenta el peso de las columnas  y el peso de las vigas el cual se calcula con las dimensiones del plano, de la siguiente manera:

\begin{gather*}
    \mathbf{W_{vigas}}=[(5.7+5.5+6.75)\cdot0.4+(1.25+5.7+5.5+6.75)\cdot0.5\cdot2+(5.7+5.5+6.75)\cdot0.4\\ +((6.75+7.5+6.85)\cdot4+0.8\cdot3-(0.8+1)\cdot4)\cdot0.3]~m^{2}\cdot24\tfrac{kN}{m^{3}}\cdot0.45m=\mathbf{620.352~kN}
\end{gather*}

\begin{equation*}
    \mathbf{W_{col}}=[0.7\cdot0.6~m^{2}\cdot12+0.6\cdot0.5~m^{2}\cdot4]\cdot 24 \tfrac{kN}{m^{3}} \cdot 2.5 m=\mathbf{374.4~kN}
\end{equation*}

\begin{equation*}
    \mathbf{W_{asc}}=3.0625\;m^2\cdot 5.85\tfrac{kN}{m^2}=\mathbf{17.916~kN}
\end{equation*}

Teniendo esto en cuenta se calcula el peso de cubierta con la siguiente ecuación

\begin{gather*}
    \mathbf{W}=Area\cdot D_{PT}+W_{col}+W_{vigas}+W_{asc}\\
    \mathbf{W}=389.6175~m^{2}\cdot 5.369\tfrac{kN}{m^{2}}+620.352~kN+374.4~kN+17.916~kN=\mathbf{3104.45~kN}
\end{gather*}

\subsection{Resumen peso del edificio por pisos y peso total}

Con los resultados anteriores del peso del piso tipo y el de cubierta se presenta el siguiente resumen:

% Table generated by Excel2LaTeX from sheet 'W_Edificio'
\begin{table}[H]
  \centering
    \begin{tabular}{|c|c|}
    \hline
    \rowcolor[rgb]{ .2,  .247,  .31} \textcolor[rgb]{ 1,  1,  1}{\textbf{Piso}} & \multicolumn{1}{c|}{\textcolor[rgb]{ 1,  1,  1}{\textbf{W (kN)}}} \bigstrut\\
    \hline
    \rowcolor[rgb]{ .2,  .247,  .31} \textcolor[rgb]{ 1,  1,  1}{\textbf{Cubierta}} & \cellcolor[rgb]{ 1,  1,  1}3104.45 \bigstrut\\
    \hline
    \rowcolor[rgb]{ .2,  .247,  .31} \textcolor[rgb]{ 1,  1,  1}{\textbf{4to piso}} & \cellcolor[rgb]{ 1,  1,  1}3752.37 \bigstrut\\
    \hline
    \rowcolor[rgb]{ .2,  .247,  .31} \textcolor[rgb]{ 1,  1, 1}{\textbf{3er piso}} & \cellcolor[rgb]{ 1,  1,  1}3752.37 \bigstrut\\
    \hline
    \rowcolor[rgb]{ .2,  .247,  .31} \textcolor[rgb]{ 1,  1,  1}{\textbf{2do piso}} & \cellcolor[rgb]{ 1,  1,  1}3752.37 \bigstrut\\
    \hline
    \rowcolor[rgb]{ .2,  .247,  .31} \textcolor[rgb]{ 1,  1,  1}{\textbf{TOTAL}} & \cellcolor[rgb]{ 1,  1,  1}14361.57 \bigstrut\\
    \hline
    \end{tabular}%
  \caption{Resumen peso del edificio}
  \label{tab:ResumenWEd}%
\end{table}%




\section{Evaluación de fuerzas sísmicas}
 \subsection{Amenaza sísmica $\mathbf{A_{a}}$, $\mathbf{A_{v}}$}
 A partir de la Tabla A.2.3-2 de la NSR-10 se obtienen los coeficientes de la aceleración pico efectiva  $A_{a}$ y la velocidad pico efectiva $A_{v}$ correspondientes para la ciudad de Bogotá D.C, la cual se encuentra en una zona de amenaza sísmica intermedia.
 \begin{itemize}
     \item $\mathbf{A_{a}}=0.15$
     \item $\mathbf{A_{v}}=0.20$
     
 \end{itemize}
 
\subsection{Micro Zonificación Sísmica de Bogotá}
 A partir de la microzonificación sísmica de Bogotá, dada por el decreto 523 de 2010 se obtienen los coeficientes de amplificación del suelo para períodos cortos del espectro y para periodos intermedios del espectro para Lacustre 100
 \begin{itemize}
     \item $\mathbf{F_{a}}=1.30$
     \item $\mathbf{F_{v}}=3.20$
     \item $\mathbf{T_{c}}=1.58~s$
     \item $\mathbf{T_{L}}=4.0~s$
     
 \end{itemize}
\subsection{Grupo de uso. Coeficiente de importancia}

Como se mencionó en la descripción del proyecto, este consiste en un edificio de tipo residencial y por lo tanto se encuentra catalogado dentro del grupo de uso \textbf{I}. Este grupo de uso está relacionado con un coeficiente de importancia de \textbf{1.00} de acuerdo a la sección A.2.5 de la NSR-10.

\subsection{Periodo de vibración}
Para determinar el período de vibración al cual inicia la zona de aceleraciones constantes del espectro se utiliza la ecuación \textit{A.2.6-6} de la NSR-10 como se muestra a continuación:

\begin{equation}
    T_{0}=0.1 \cdot \frac{A_{v}\cdot F_{v}}{A_{a}\cdot F_{a}}=0.1 \cdot \frac{0.2 \cdot 3.2}{0.15 \cdot 1.3}= 0.33\; s 
    \label{eq:calc_T0}
\end{equation}




Para el calculo del periodo fundamental aproximado de la estructura se emplean los coeficientes de la tabla \textit{A.4.2-1}. Como el sistema estructural de resistencia sísmica del proyecto consiste en pórticos resistentes a momentos de concreto reforzado con capacidad de disipación de energía moderada(DMO) se toman los valores de $C_{t}=0.047$ y $\alpha = 0.9$. Teniendo en cuenta que la altura que la altura total del edificio es de 12m, se encuentra un valor de $T_{a}=0.44~s$.
%completar
\begin{equation}
    T_{a}=C_{t}\cdot h^{\alpha}=0.047 \cdot 12 ^{0.9}=0.44\;s
\end{equation}





\subsection{Espectro de diseño y Sa}
\label{secc:espectro y sa}

Conocido el valor del periodo fundamental ($T_a$) de la estructura se calcula el valor de la aceleración teniendo en cuenta que para este periodo $S_a$ se encuentra en la parte de la meseta. Teniendo en cuenta lo anterior se calcula el valor de la aceleración de la siguiente manera:

\begin{equation*}
    S_{a}=2.5\cdot A_{a} \cdot F_{a} \cdot I= 2.5\cdot 0.15\cdot1.3\cdot1=0.4875
\end{equation*}

\begin{figure}[H]
    \centering
    \includegraphics[scale=1.2]{images/Espectro_aceleraciones_DE.pdf}
    \caption{Espectro de aceleraciones }
    \label{fig:ESP}
\end{figure}


 \subsection{Cortante en la base}
 
 El cortante en la base se calcula con la ecuación (A.4.3.1) de la NSR-10 como se muestra a continuación, en donde se tiene que $Sa=0.4875$ (ver sección \ref{secc:espectro y sa}) y el peso del edificio $W=14361.57~kN$ (ver tabla \ref{tab:ResumenWEd})
 
 \begin{equation}
     V_{s}=S_{a}\cdot g \cdot M = 0.4875 \cdot 14361.57~kN= 7001.267~kN
 \end{equation}
 
\subsection{Distribución vertical de la fuerza sísmica}

Teniendo en cuenta la sección \textit{A.4.3.2} se realiza la distribución del cortante sísmico en la base para cada piso de la estructura utilizando el coeficiente $C_{vx}$ que se define de la siguiente manera:

\begin{equation}
    C_{vx}=\frac{m_{x}\cdot h_{x}^{k}}{\sum_{i=1}^{n}(m_{i}\cdot h_{i}^{k})}
\end{equation}
Donde se tiene que el coeficiente \textbf{k}, el cual se relaciona con el periodo fundamental T y para el caso de la estructura tiene un valor de $\mathbf{k=1.0}$ porque el periodo $T_{a}$ es menor a 0.5. A partir de $\mathbf{C_{vx}}$ se obtiene la fuerza sísmica que se le aplica a cada uno de los pisos.
\begin{equation}
    Fx=C_{vx}\cdot V_s
\end{equation}

Al utilizar esta formulación y con los datos obtenidos anteriormente este procedimiento se resume en la siguiente tabla:

% Table generated by Excel2LaTeX from sheet 'W_Edificio'
\begin{table}[H]
  \centering
  
    \begin{tabular}{rrr|c|c|c|r}
    \hline
    \rowcolor[rgb]{ .2,  .247,  .31} \multicolumn{1}{|c|}{\textcolor[rgb]{ 1,  1,  1}{\textbf{Nivel}}} & \multicolumn{1}{c|}{\textcolor[rgb]{ 1,  1,  1}{\textbf{mx (Wx)}}} & \multicolumn{1}{c|}{\textcolor[rgb]{ 1,  1,  1}{\textbf{hx}}} & \textcolor[rgb]{ 1,  1,  1}{\textbf{mxhxk}} & \textcolor[rgb]{ 1,  1,  1}{\textbf{Cvx}} & \textcolor[rgb]{ 1,  1,  1}{\textbf{Fx(kN)}} & \multicolumn{1}{c|}{\textcolor[rgb]{ 1,  1,  1}{\textbf{Vx (kN)}}} \bigstrut\\
    \hline
    \rowcolor[rgb]{ .2,  .247,  .31} \multicolumn{1}{|c|}{\textcolor[rgb]{ 1,  1,  1}{\textbf{Cubierta}}} & \multicolumn{1}{c|}{\cellcolor[rgb]{ 1,  1,  1}3104.45} & \multicolumn{1}{c|}{\cellcolor[rgb]{ 1,  1,  1}12.1} & \cellcolor[rgb]{ 1,  1,  1}37563.86 & \cellcolor[rgb]{ 1,  1,  1}0.354 & \cellcolor[rgb]{ 1,  1,  1}2475.66 & \multicolumn{1}{c|}{\cellcolor[rgb]{ 1,  1,  1}2475.66} \bigstrut\\
    \hline
    \rowcolor[rgb]{ .2,  .247,  .31} \multicolumn{1}{|c|}{\textcolor[rgb]{ 1,  1,  1}{\textbf{4°}}} & \multicolumn{1}{c|}{\cellcolor[rgb]{ 1,  1,  1}3752.37} & \multicolumn{1}{c|}{\cellcolor[rgb]{ 1,  1,  1}9.1} & \cellcolor[rgb]{ 1,  1,  1}34146.60 & \cellcolor[rgb]{ 1,  1,  1}0.321 & \cellcolor[rgb]{ 1,  1,  1}2250.44 & \multicolumn{1}{c|}{\cellcolor[rgb]{ 1,  1,  1}4726.10} \bigstrut\\
    \hline
    \rowcolor[rgb]{ .2,  .247,  .31} \multicolumn{1}{|c|}{\textcolor[rgb]{ 1,  1,  1}{\textbf{3°}}} & \multicolumn{1}{c|}{\cellcolor[rgb]{ 1,  1,  1}3752.37} & \multicolumn{1}{c|}{\cellcolor[rgb]{ 1,  1,  1}6.1} & \cellcolor[rgb]{ 1,  1,  1}22889.48 & \cellcolor[rgb]{ 1,  1,  1}0.215 & \cellcolor[rgb]{ 1,  1,  1}1508.54 & \multicolumn{1}{c|}{\cellcolor[rgb]{ 1,  1,  1}6234.63} \bigstrut\\
    \hline
    \rowcolor[rgb]{ .2,  .247,  .31} \multicolumn{1}{|c|}{\textcolor[rgb]{ 1,  1,  1}{\textbf{2°}}} & \multicolumn{1}{c|}{\cellcolor[rgb]{ 1,  1,  1}3752.37} & \multicolumn{1}{c|}{\cellcolor[rgb]{ 1,  1,  1}3.1} & \cellcolor[rgb]{ 1,  1,  1}11632.36 & \cellcolor[rgb]{ 1,  1,  1}0.109 & \cellcolor[rgb]{ 1,  1,  1}766.63 & \multicolumn{1}{c|}{\cellcolor[rgb]{ 1,  1,  1}7001.27} \bigstrut\\
    \hline
        &     &     & \textbf{106232.31} & \textbf{1.000} & \textbf{7001.27} &  \bigstrut\\
\cline{4-6}    \end{tabular}%
\caption{Distribución vertical de la fuerza sísmica}
  \label{tab:DVFS}%
\end{table}%



\subsection{Torsión accidental}

La torsión accidental se calcula suponiendo que la masa de casa piso está desplazada del centro de masa un 5\% de la longitud máxima en que se está analizando. Estas longitudes máximas se presentan en la tabla \ref{tab:LMAX_TA}

% Table generated by Excel2LaTeX from sheet 'W_Edificio'
\begin{table}[h]
  \centering
 
    \begin{tabular}{|c|c|}
    \hline
    \rowcolor[rgb]{ .2,  .247,  .31} \textcolor[rgb]{ 1,  1,  1}{\textbf{Lx}} & \cellcolor[rgb]{ 1,  1,  1}22.9 \bigstrut\\
    \hline
    \rowcolor[rgb]{ .2,  .247,  .31} \textcolor[rgb]{ 1,  1,  1}{\textbf{Ly}} & \cellcolor[rgb]{ 1,  1,  1}19.2 \bigstrut\\
    \hline
    \rowcolor[rgb]{ .2,  .247,  .31} \textcolor[rgb]{ 1,  1,  1}{\textbf{eax}} & \cellcolor[rgb]{ 1,  1,  1}1.145 \bigstrut\\
    \hline
    \rowcolor[rgb]{ .2,  .247,  .31} \textcolor[rgb]{ 1,  1,  1}{\textbf{eay}} & \cellcolor[rgb]{ 1,  1,  1}0.96 \bigstrut\\
    \hline
    \end{tabular}%
     \caption{Longitudes máximas para torsión accidental}
  \label{tab:LMAX_TA}%
\end{table}%

La torsión accidental se encuentra multiplicando la fuerza horizontal y las longitudes máximas, como se presenta a continuación.
% Table generated by Excel2LaTeX from sheet 'W_Edificio'
\begin{table}[h]
  \centering
  
    \begin{tabular}{|c|c|c|c|}
    \hline
    \rowcolor[rgb]{ .2,  .247,  .31} \textcolor[rgb]{ 1,  1,  1}{\textbf{Nivel}} & \textcolor[rgb]{ 1,  1,  1}{\textbf{Fx (kN)}} & \textcolor[rgb]{ 1,  1,  1}{\textbf{Tx}} & \textcolor[rgb]{ 1,  1,  1}{\textbf{Ty (kN)}} \bigstrut\\
    \hline
    \rowcolor[rgb]{ .2,  .247,  .31} \textcolor[rgb]{ 1,  1,  1}{\textbf{Cubierta}} & \cellcolor[rgb]{ 1,  1,  1}2475.66 & \cellcolor[rgb]{ 1,  1,  1}2376.63 & \cellcolor[rgb]{ 1,  1,  1}2834.63 \bigstrut\\
    \hline
    \rowcolor[rgb]{ .2,  .247,  .31} \textcolor[rgb]{ 1,  1,  1}{\textbf{4°}} & \cellcolor[rgb]{ 1,  1,  1}2250.44 & \cellcolor[rgb]{ 1,  1,  1}2160.42 & \cellcolor[rgb]{ 1,  1,  1}2576.75 \bigstrut\\
    \hline
    \rowcolor[rgb]{ .2,  .247,  .31} \textcolor[rgb]{ 1,  1,  1}{\textbf{3°}} & \cellcolor[rgb]{ 1,  1,  1}1508.54 & \cellcolor[rgb]{ 1,  1,  1}1448.20 & \cellcolor[rgb]{ 1,  1,  1}1727.27 \bigstrut\\
    \hline
    \rowcolor[rgb]{ .2,  .247,  .31} \textcolor[rgb]{ 1,  1,  1}{\textbf{2°}} & \cellcolor[rgb]{ 1,  1,  1}766.63 & \cellcolor[rgb]{ 1,  1,  1}735.97 & \cellcolor[rgb]{ 1,  1,  1}877.80 \bigstrut\\
    \hline
    \end{tabular}%
    \caption{Valores de torsión accidental}
  \label{tab:TorsiónAccidental}%
\end{table}%

\subsection{Irregularidades en planta, altura y por ausencia de redundancia}

\subsubsection{Irregularidades en planta}
\textbf{Tipo 1aP y 1bP- irregularidad torsional e irregularidad torsional extrema}\\

Para determinar si se presenta irregularidad torsional o irregularidad torsional extrema se compara la deriva mayor con la deriva promedio en cada uno de los lados de la estructura como se presenta a continuación:
% Table generated by Excel2LaTeX from sheet 'Irregularidades '
\begin{table}[H]
  \centering
  
    \begin{tabular}{|c|l|r|r|}
\cline{3-4}    \multicolumn{1}{r}{} &     & \multicolumn{1}{c|}{\cellcolor[rgb]{ .2,  .247,  .31}\textcolor[rgb]{ 1,  1,  1}{$\mathbf{\Delta X (m)}$}} & \multicolumn{1}{c|}{\cellcolor[rgb]{ .2,  .247,  .31}\textcolor[rgb]{ 1,  1,  1}{\textbf{$\mathbf{\Delta Y (m)}$}}} \bigstrut\\
    \hline
    \rowcolor[rgb]{ .2,  .247,  .31} \multicolumn{4}{|c|}{\textcolor[rgb]{ 1,  1,  1}{\textbf{A1-A4}}} \bigstrut\\
    \hline
    \multirow{2}[4]{*}{\textbf{CUB}} & Promedio & 0.0217 & 0.0153 \bigstrut\\
\cline{2-4}        & Relación & 1.00 & 1.14 \bigstrut\\
    \hline
    \multirow{2}[4]{*}{\textbf{4}} & Promedio & 0.0285 & 0.0234 \bigstrut\\
\cline{2-4}        & Relación & 1.00 & 1.12 \bigstrut\\
    \hline
    \multirow{2}[4]{*}{\textbf{3}} & Promedio & 0.0297 & 0.0268 \bigstrut\\
\cline{2-4}        & Relación & 1.00 & 1.09 \bigstrut\\
    \hline
    \multirow{2}[4]{*}{\textbf{2}} & Promedio & 0.0168 & 0.0169 \bigstrut\\
\cline{2-4}        & Relación & 1.00 & 1.07 \bigstrut\\
    \hline
    \end{tabular}%
    \caption{Comparación derivas para irregularidad torsional (A1-A4)}
  \label{tab:A1-A4}%
\end{table}%


% Table generated by Excel2LaTeX from sheet 'Irregularidades '
\begin{table}[H]
  \centering
    \begin{tabular}{|c|l|r|r|}
\cline{3-4}    \multicolumn{1}{r}{} &     & \multicolumn{1}{c|}{\cellcolor[rgb]{ .2,  .247,  .31}\textcolor[rgb]{ 1,  1,  1}{\textbf{$\mathbf{\Delta X (m)}$}}} & \multicolumn{1}{c|}{\cellcolor[rgb]{ .2,  .247,  .31}\textcolor[rgb]{ 1,  1,  1}{\textbf{$\mathbf{\Delta Y (m)}$}}} \bigstrut\\
    \hline
    \rowcolor[rgb]{ .2,  .247,  .31} \multicolumn{4}{|c|}{\textcolor[rgb]{ 1,  1,  1}{\textbf{A1-D1}}} \bigstrut\\
    \hline
    \multirow{2}[4]{*}{\textbf{CUB}} & Promedio & 0.0212 & 0.0175 \bigstrut\\
\cline{2-4}        & Relación & 1.02 & 1.00 \bigstrut\\
    \hline
    \multirow{2}[4]{*}{\textbf{4}} & Promedio & 0.0281 & 0.0261 \bigstrut\\
\cline{2-4}        & Relación & 1.02 & 1.00 \bigstrut\\
    \hline
    \multirow{2}[4]{*}{\textbf{3}} & Promedio & 0.0294 & 0.0293 \bigstrut\\
\cline{2-4}        & Relación & 1.01 & 1.00 \bigstrut\\
    \hline
    \multirow{2}[4]{*}{\textbf{2}} & Promedio & 0.0167 & 0.0180 \bigstrut\\
\cline{2-4}        & Relación & 1.00 & 1.00 \bigstrut\\
    \hline
    \end{tabular}%

    \caption{Comparación derivas para irregularidad torsional (A1-D1)}
  \label{tab:A1-D1}%
\end{table}%

% Table generated by Excel2LaTeX from sheet 'Irregularidades '
\begin{table}[H]
  \centering
    \begin{tabular}{|c|l|r|r|}
\cline{3-4}    \multicolumn{1}{r}{} &     & \multicolumn{1}{c|}{\cellcolor[rgb]{ .2,  .247,  .31}\textcolor[rgb]{ 1,  1,  1}{\textbf{$\mathbf{\Delta X (m)}$}}} & \multicolumn{1}{c|}{\cellcolor[rgb]{ .2,  .247,  .31}\textcolor[rgb]{ 1,  1,  1}{\textbf{$\mathbf{\Delta Y (m)}$}}} \bigstrut\\
    \hline
    \rowcolor[rgb]{ .2,  .247,  .31} \multicolumn{4}{|c|}{\textcolor[rgb]{ 1,  1,  1}{\textbf{D1-D4}}} \bigstrut\\
    \hline
    \multirow{2}[4]{*}{\textbf{CUB}} & Promedio & 0.0207 & 0.0153 \bigstrut\\
\cline{2-4}        & Relación & 1.00 & 1.14 \bigstrut\\
    \hline
    \multirow{2}[4]{*}{\textbf{4}} & Promedio & 0.0277 & 0.0234 \bigstrut\\
\cline{2-4}        & Relación & 1.00 & 1.12 \bigstrut\\
    \hline
    \multirow{2}[4]{*}{\textbf{3}} & Promedio & 0.0292 & 0.0268 \bigstrut\\
\cline{2-4}        & Relación & 1.00 & 1.09 \bigstrut\\
    \hline
    \multirow{2}[4]{*}{\textbf{2}} & Promedio & 0.0167 & 0.0169 \bigstrut\\
\cline{2-4}        & Relación & 0.02 & 0.02 \bigstrut\\
    \hline
    \end{tabular}%
    \caption{Comparación derivas para irregularidad torsional (D1-D4)}
  \label{tab:D1-D4}%
\end{table}%

% Table generated by Excel2LaTeX from sheet 'Irregularidades '
\begin{table}[H]
  \centering
    \begin{tabular}{|c|l|r|r|}
\cline{3-4}    \multicolumn{1}{r}{} &     & \multicolumn{1}{c|}{\cellcolor[rgb]{ .2,  .247,  .31}\textcolor[rgb]{ 1,  1,  1}{\textbf{$\mathbf{\Delta X (m)}$}}} & \multicolumn{1}{c|}{\cellcolor[rgb]{ .2,  .247,  .31}\textcolor[rgb]{ 1,  1,  1}{\textbf{$\mathbf{\Delta Y (m)}$}}} \bigstrut\\
    \hline
    \rowcolor[rgb]{ .2,  .247,  .31} \multicolumn{4}{|c|}{\textcolor[rgb]{ 1,  1,  1}{\textbf{D4-A4}}} \bigstrut\\
    \hline
    \multirow{2}[4]{*}{\textbf{CUB}} & Promedio & 0.0212 & 0.0132 \bigstrut\\
\cline{2-4}        & Relación & 1.02 & 1.00 \bigstrut\\
    \hline
    \multirow{2}[4]{*}{\textbf{4}} & Promedio & 0.0281 & 0.0207 \bigstrut\\
\cline{2-4}        & Relación & 1.02 & 1.00 \bigstrut\\
    \hline
    \multirow{2}[4]{*}{\textbf{3}} & Promedio & 0.0294 & 0.0243 \bigstrut\\
\cline{2-4}        & Relación & 1.01 & 1.00 \bigstrut\\
    \hline
    \multirow{2}[4]{*}{\textbf{2}} & Promedio & 0.0167 & 0.0157 \bigstrut\\
\cline{2-4}        & Relación & 1.00 & 1.00 \bigstrut\\
    \hline
    \end{tabular}%
    \caption{Comparación derivas para irregularidad torsional (D4-A4)}
  \label{tab:D4-A4}%
\end{table}%

La máxima relación entre la deriva máxima y la deriva promedio es de 1.14, como este valor es menor a 1.2 se concluye que no se presenta irregularidad torsional.\\

\textbf{Tipo 2P - Retrocesos en las esquinas}\\

De acuerdo a la geometría de la losa se verifica que la estructura no presenta irregularidad por retrocesos en las esquinas, por lo cual el valor de $\phi=1.0$

\textbf{Tipo 3P- Irregularidad del diafragma }\\

Como se muestra en los planos adjuntos a este documento, la estructura presenta diversos vacíos debido al ascensor, la escalera y los ductos. Sin embargo estos son menores al 50\% del área de cada piso, por lo que no se considera la irregularidad de diafragma.

\textbf{Tipo 4P- Desplazamiento de los planos de Acción }\\
No se presentan desplazamientos de los planos de acción porque las columnas transmiten el peso de manera continua desde la cubierta hasta el suelo.\\

\textbf{Tipo 5P- Sistemas no paralelos}
No se presentan sistemas no paralelos en la estructura de acuerdo a la distribución geométrica de los ejes de la estructura.

Una vez verificado los tipos de irregularidades que se presentan se determina un coeficiente de reducción de la capacidad de disipación de energía por irregularidades en planta de $\mathbf{\phi_{p}=1.0}$

\subsubsection{Irregularidades en la altura}

\textbf{Tipo 1aA y 1bA - Piso flexible y Piso flexible extremo}\\
Teniendo en cuenta que la altura del primer piso solo difiere en $10\;cm$ con respecto a la de los demás pisos, no se presenta irregularidad por piso flexible.\\

\textbf{Tipo 2A - Irregularidad en la distribución de masa }\\
 Debido a que la masa de entrepiso no es 1.5 veces mayor que la de cubierta no se presenta este tipo de irregularidad.\\
 
 \textbf{Tipo 3A - Irregularidad geométrica}\\
 De acuerdo a la distribución geométrica del sistema de resistencia sísmica de entrepisos y cubierta no se presenta una dimensión que sea mayor a 1.3 veces  la dimensión del piso adyacente. Por lo que se determina que la estructura no presenta este tipo de irregularidad\\
 
 \textbf{Tipo 4A - Desplazamiento dentro del plano de acción}\\
 Como se menciona anteriormente, las columnas son continuas a lo largo de todos los pisos , por lo que no se presenta este tipo de irregularidad.\\
 
 \textbf{Tipo 5aA y 5bA - Piso débil y piso débil extremo}\\
 
De acuerdo a la geometría de las plantas, la rigidez es similar en todos los pisos, por lo que no se considera irregularidad por piso débil.\\

Como no se presentan irregularidades en altura, se tiene que el coeficiente de reducción de la capacidad de disipación de energía por irregularidades en altura es igual a $\mathbf{\phi_{a}=1}$

\subsubsection{Irregularidades por ausencia de redundancia}


De acuerdo con el numeral A.3.3.8.2 de la NSR-10 no se tiene irregularidad por ausencia de redundancia debido a que no se presenta falta de repetición de elementos estructurales verticales por lo que no hay una pérdida de resistencia a momento en la conexión viga-columna de los dos extremos de la viga mayor al 33\% de la resistencia ante fuerzas horizontales del piso. Adicionalmente, no se produce ningún tipo de irregularidad torsional en planta extrema, por lo cual se considera un valor de $\mathbf{\phi_{r}=1}$

Para este ejercicio se diseñaron todas las zapatas como zapatas concéntricas. En donde la carga total transmitida al terreno corresponde a la suma de las carga $P_{D}$ y $P_{L}$ y el peso propio de la zapata, que se asumió como un incremento del 10 \% al valor anterior. A continuación se desarrolla una muestra de calculo con la zapata A1:
\begin{equation*}
    \sum P =P_{T}+W_{pp}
\end{equation*}
\begin{equation*}
    \ P_{T}=P_{D}+P_{L}
\end{equation*}
\begin{equation*}    
    \sum P =(P_{D}+P_{L})+W_{pp}=(P_{D}+P_{L})\cdot (1.1)=(568.957 kN+90.537 kN)\cdot (1.10)= 725.443 kN
\end{equation*}

Con $\sum P$ calculado, se determinó el área mínima que debería tener la zapata en su base.
\begin{equation*}
    \ A=\frac{\sum P}{\sigma_{adm}}=\frac{725.443 kN}{220~kPa}=3.30~m^2
\end{equation*}

Sabiendo que el área de las zapatas corresponde a:
\begin{equation*}
    \ A=B^2
\end{equation*}

Utilizando un despeje de esta ecuación es posible determinar el la longitud, $B$, de la zapatas:
\begin{equation*}
    \ B=\sqrt{A}=\sqrt{3.30~m^2}= 1.82~m
\end{equation*}

Lo cual, para cumplir con medidas constructivas se redondeo a $1.80 m$. Seguidamente se recalculo de nuevo el valor del área de la zapata. Para la altura se propuso que todas las zapatas tengan para $H_{1}$ y $H_{2}$, una medida de $0.35 m$. Obteniendo así una altura total, $H$, de:
\begin{equation*}
    \ H=H_{1}+H_{2}=0.35~m+0.35~m=0.7~m
\end{equation*}

Ahora con la altura $H$, para un diámetro de barra correspondiente a la No. 7, y un recubrimiento, $r$ de $0.075 m$, la altura efectiva $d$, es igual a:
\begin{equation*}
     d=H-r-\phi_\#7-\frac{\phi_{\#7}}{2}=700~mm-75~mm-22.2~mm-\frac{22.2~mm}{2}=0.6139m
\end{equation*}
EL procedimiento descrito, se hace de manera igual para cada una de las zapatas del proyecto. A continuación se muestra el dimensionamiento resumido de todas las zapatas del proyecto.

\subsection{Cortante en dos direcciones}
Para el cortante en dos direcciones, se determinan los valores de los lados $l_{ox}$, $l_{oy}$, el perímetro de falla $b_{o}$ y también las longitudes de los voladizos $l_{vx}$ y $l_{vy}$, como se muestra a continuación, respectivamente:
\begin{equation*}
    \ l_{ox}= \frac{d}{2}\cdot 2 + Ancho~x~de~la~columna = \frac{0.6139~m}{2}\cdot 2 + 0.7m=1.31~m
\end{equation*}
\begin{equation*}
    \ l_{oy}= \frac{d}{2}\cdot 2 + Ancho~y~de~la~columna = \frac{0.6139~m}{2}\cdot 2 + 0.6m=1.21~m
\end{equation*}
\begin{equation*}
    \ b_{o}= 2\cdot l_{ox} + 2\cdot l_{oy}= 2\cdot(1.31)+2\cdot(1.21) =5.06~m
\end{equation*}
\begin{equation*}
    \ l_{vx}= \frac{B-Ancho~x~de~Columna}{2}= \frac{1.80~m-0.7~m}{2}=0.55~m
\end{equation*}
\begin{equation*}
    \ l_{vy}= \frac{B-Ancho~y~de~Columna}{2}= \frac{1.80~m-0.6~m}{2}=0.60~m
\end{equation*}

Para determinar el valor de $P_{u}$ para el diseño de la zapata, según B.2.4 se debe usar la siguiente combinación:
\begin{equation*}
    \ P_{u}= 1.2\cdot P_{D}+1.6\cdot P_{L}=1.2\cdot (568.957~kN)+1.6\cdot (90.537~kN)=827.6~kN
\end{equation*}

Se procede a calcular el esfuerzo que actúa sobre la base debido a la carga mayorada.
\begin{equation*}
    \sigma=\frac{P_{u}}{B^2} =\frac{827.6~kN}{3.24~m^2}=255.4~kN/m^2
\end{equation*}

Luego, se calcula el valor de cortante último $V_{u}$ el cual se compara posteriormente con la resistencia del concreto para determinar si es necesario o no reforzar a cortante.
\begin{equation*}
    \ V_{u}=\sigma \cdot(A_{g}-A_{o})= \sigma_{u} \cdot(B^2-(l_{ox}*l_{oy})) = 255.4~kN/m^2\cdot(3.24~m^2-(1.31~m*1.21~m))=420.20 kN
\end{equation*}

Ahora, resulta necesario calcular el cortante que soporta el concreto, $V_{c}$, para lo cual se debe conocer primero la altura de la zapata a una distancia $d/2$, respecto a la cara del apoyo, la cual se denomina como $d_{1}$. Posteriormente se utilizan tres ecuaciones para determinar la resistencia al cortante y de estas se escoge el valor mínimo obtenido para $V_{c}$.
\begin{equation*}
    \ V_{c1}=0.17\cdot \left(1+\frac{2}{\beta} \right)\cdot\lambda\sqrt{f'_{c}}\cdot b_{o} \cdot d_{1}
\end{equation*}
\begin{equation*}
    \ V_{c2}=0.083\cdot \left(\frac{\alpha_{s}\cdot d_{1}}{b_{0}}+2 \right)\cdot\lambda\sqrt{f'_{c}}\cdot b_{o} \cdot d_{1}
\end{equation*}
\begin{equation*}
    \ V_{c1}=0.33\cdot\lambda\sqrt{f'_{c}}\cdot b_{o} \cdot d_{1}
\end{equation*}
Entonces:
\begin{equation*}
    \ V_{c1}=0.17\cdot \left(1+\frac{2}{0.85} \right)\cdot 1 \cdot \sqrt{28~MPa}\cdot 5.06~m \cdot 0.4186~m=5166.81~kN
\end{equation*}
\begin{equation*}
    \ V_{c2}=0.083\cdot \left(\frac{40 \cdot 0.4186~m}{5.06~m}+2 \right)\cdot 1 \sqrt{28~MPa}\cdot 5.06~m \cdot 0.4186~m=4936.64~kN
\end{equation*}
\begin{equation*}
    \ V_{c1}=0.33\cdot1 \sqrt{28~MPa}\cdot 5.06~m \cdot 0.4186~m=3695.15~kN
\end{equation*}
Por lo que:
\begin{equation*}
    \phi V_{c}=0.75\cdot3695.15~kN=2771.4~kN
\end{equation*}
entonces se considera que no se necesita refuerzo a cortante mediante el análisis de cortante en dos direcciones.

\subsection{Cortante como viga}
Para el análisis del cortante de la zapata como viga es necesario primero calcular la capacidad portante de la zapata $W_{u}$, Como se muestra a continuación:
\begin{equation*}
    \ W_{u}=\sigma_{u}\cdot B = 250~kN/m^2\cdot1.80~m=459.8 kN/m
\end{equation*}

Se determina el cortante último $V_{u}$, que actúa sobre la base.
\begin{equation*}
    \ V_{u}=W_{u}\cdot \left(\frac{B}{2}-\frac{Ancho~x~de~la~columna}{2}-d \right)=459.8 kN/m\cdot(0.55~m-0.6139~m)=-29.38kN
\end{equation*}
Es necesario definir la altura de la zapata a una distancia $d$  respecto al apoyo, dicha altura para el presente análisis se denominará $d_{2}$.
\begin{equation*}
    \phi V_{c}=\phi 0.17\cdot\lambda\sqrt{f'_{c}}\cdot B \cdot d_{2}
\end{equation*}
\begin{equation*}
    \phi V_{c}=0.75\cdot 0.17\cdot(1)\sqrt{28~MPa}\cdot 1.80~m \cdot 0.4504~m=546.9 
\end{equation*}
Como $\phi V_{c}>V_{u}$, se considera que no es necesario reforzar a cortante la zapata  mediante el análisis de cortante como viga.
El procedimiento descrito, se hace de manera igual para cada una de las zapatas del proyecto.
\subsection{Diseño a flexión}
Considerando el momento de diseño como el momento ultimo de trabajo, $M_{u}$,  el cual se calcula como:
\begin{equation*}
    \ M_{u}=\frac{W_{u}\cdot l_{vx}^2}{2}=\frac{459.2kN/m\cdot(0.55m)^2}{2}=69.54~kNm
\end{equation*}
De esta manera se procede a calcular el valor de la cuantía de refuerzo a flexión:
\begin{equation*}
    \rho=\left[1-\sqrt{1-\frac{2.62\cdot M_{u}}{B\cdot d^2\cdot f'_{c}}}\right]\cdot \frac{f'_{c}}{1.18 f_{yt}}
\end{equation*}
\begin{equation*}
    \rho=\left[1-\sqrt{1-\frac{2.62\cdot 69.54~kNm}{1.82~m\cdot (0.6139~m)^2\cdot 28\cdot 10^3 kPa}}\right]\cdot \frac{28~MPa}{1.18 (420~MPa)}=0.0033
\end{equation*}

Luego se determina el esfuerzo a flexión como:
\begin{equation*}
    \ A_{s}=\rho\cdot B\cdot d=0.0033\cdot 1.80~m\cdot 0.6139~m\cdot 10^6=3683.4~mm^2
\end{equation*}
Se debe comparar dicho valor del refuerzo obtenido con el valor mínimo calculado para una cuantía minia de refuerzo, como:
\begin{equation*}
    \ A_{s}=0.0018 \cdot B\cdot d=0.0018\cdot 1.80~m\cdot 0.6139~m\cdot 10^6=2268~mm^2
\end{equation*}

Como el valor de refuerzo mínimo es inferior al requerido, se debe determinar la cantidad de barras para el requerido, en otro caso se debe elegir el mayor de los dos. Para garantizar al menos los $3683.4~mm^2$ de refuerzo se proponen 7 barras No. 10, para tener un área de refuerzo efectiva de $3870~mm^2$.
El procedimiento descrito, se hace de manera igual para cada una de las zapatas del proyecto. A continuación se muestra el resumen de los valores obtenidos para cada zapatas del proyecto
\begin{table}[H]
    \centering
    % Table generated by Excel2LaTeX from sheet 'Hoja2'
    \resizebox{\linewidth}{!}{
\begin{tabular}{|c|c|c|c|c|c|c|c|c|c|c|c|c|c|c|}
\hline
\rowcolor[rgb]{ .851,  .851,  .851}     & \multicolumn{5}{c|}{\cellcolor[rgb]{ .988,  .894,  .839}\textbf{CORTANTE COMO VIGA}} & \multicolumn{9}{c|}{\cellcolor[rgb]{ .776,  .878,  .706}\textbf{DISEÑO A FLEXIÓN}} \bigstrut\\
\hline
\rowcolor[rgb]{ .851,  .851,  .851} \multicolumn{1}{|l|}{\textbf{COLUMNA}} & \cellcolor[rgb]{ .988,  .894,  .839}\textbf{$w_u (kN/m)$} & \cellcolor[rgb]{ .988,  .894,  .839}\textbf{$V_u (kN)$} & \cellcolor[rgb]{ .988,  .894,  .839}\textbf{$d_2(m)$} & \cellcolor[rgb]{ .988,  .894,  .839}\textbf{$\phi Vc$} & \cellcolor[rgb]{ .988,  .894,  .839}\textbf{cumple?} & \cellcolor[rgb]{ .776,  .878,  .706}\textbf{$M_u (kN\cdot m)$} & \cellcolor[rgb]{ .776,  .878,  .706}\textbf{$\rho$} & \cellcolor[rgb]{ .776,  .878,  .706}\textbf{$A_s  (mm^2)$} & \cellcolor[rgb]{ .776,  .878,  .706}\textbf{$A_{s ~min} (mm^2)$} & \cellcolor[rgb]{ .776,  .878,  .706}\textbf{$A_{s ~ requerido} (mm^2)$} & \cellcolor[rgb]{ .776,  .878,  .706}\textbf{$barra \#$} & \cellcolor[rgb]{ .776,  .878,  .706}\textbf{$n^{\circ} barras$} & \cellcolor[rgb]{ .776,  .878,  .706}\textbf{$A_{s} sum (mm^2)$} & \cellcolor[rgb]{ .776,  .878,  .706}\textbf{s (mm)} \bigstrut\\
\hline
A1  & 459.782 & -29.380 & 0.4504 & 546.927 & si  & 69.542028 & 0.0033333 & 3683.40 & 2268 & 3683.40 & 7   & 10  & 3870 & 0.175 \bigstrut\\
\hline
A2  & 723.641 & 351.762 & 0.4504 & 881.159 & si  & 437.80278 & 0.0033333 & 5934.37 & 3654 & 5934.37 & 7   & 16  & 6192 & 0.18 \bigstrut\\
\hline
A3  & 726.962 & 353.376 & 0.4504 & 881.159 & si  & 439.81197 & 0.0033333 & 5934.37 & 3654 & 5934.37 & 7   & 16  & 6192 & 0.18 \bigstrut\\
\hline
A4  & 468.460 & -29.935 & 0.4504 & 546.927 & si  & 70.854642 & 0.0033333 & 3683.40 & 2268 & 3683.40 & 7   & 10  & 3870 & 0.175 \bigstrut\\
\hline
B1  & 491.236 & 17.734 & 0.4504 & 607.696 & si  & 103.77369 & 0.0033333 & 4092.67 & 2520 & 4092.67 & 7   & 11  & 4257 & 0.18 \bigstrut\\
\hline
B2  & 813.932 & 599.135 & 0.4369 & 972.705 & si  & 741.69542 & 0.0033333 & 6752.90 & 4158 & 6752.90 & 7   & 18  & 6966 & 0.18 \bigstrut\\
\hline
B3  & 893.704 & 747.226 & 0.4369 & 1031.657 & si  & 939.50597 & 0.0033333 & 7162.17 & 4410 & 7162.17 & 7   & 19  & 7353 & 0.18 \bigstrut\\
\hline
B4  & 703.997 & 271.813 & 0.4504 & 820.390 & si  & 351.99874 & 0.0033333 & 5525.10 & 3402 & 5525.10 & 7   & 15  & 5805 & 0.175 \bigstrut\\
\hline
C1  & 499.402 & 18.028 & 0.4504 & 607.696 & si  & 105.49876 & 0.0033333 & 4092.67 & 2520 & 4092.67 & 7   & 11  & 4257 & 0.18 \bigstrut\\
\hline
C2  & 837.907 & 616.783 & 0.4369 & 972.705 & si  & 763.54245 & 0.0033333 & 6752.90 & 4158 & 6752.90 & 7   & 18  & 6966 & 0.18 \bigstrut\\
\hline
C3  & 893.291 & 791.545 & 0.4369 & 1061.133 & si  & 1004.9525 & 0.0033333 & 7366.80 & 4536 & 7366.80 & 7   & 20  & 7740 & 0.175 \bigstrut\\
\hline
C4  & 725.953 & 352.886 & 0.4504 & 881.159 & si  & 439.20129 & 0.0033333 & 5934.37 & 3654 & 5934.37 & 7   & 16  & 6192 & 0.18 \bigstrut\\
\hline
D1  & 527.743 & 45.439 & 0.4504 & 638.081 & si  & 129.29709 & 0.0033333 & 4297.30 & 2646 & 4297.30 & 7   & 12  & 4644 & 0.17 \bigstrut\\
\hline
D2  & 776.121 & 454.885 & 0.4504 & 941.929 & si  & 558.80714 & 0.0033333 & 6343.63 & 3906 & 6343.63 & 7   & 17  & 6579 & 0.18 \bigstrut\\
\hline
D3  & 775.261 & 454.381 & 0.4504 & 941.929 & si  & 558.18813 & 0.0033333 & 6343.63 & 3906 & 6343.63 & 7   & 17  & 6579 & 0.18 \bigstrut\\
\hline
D4  & 587.781 & 109.386 & 0.4504 & 698.851 & si  & 188.08988 & 0.0033333 & 4706.57 & 2898 & 4706.57 & 7   & 13  & 5031 & 0.175 \bigstrut\\
\hline
\end{tabular}}%

    \caption{Diseño a flexión y cortante como viga}
    \label{tab:cyf_zapata}
\end{table}
\section{Viga de amarre}
Para el diseño de las vigas de amarre, se tienen en cuenta 3 criterios para el cálculo de refuerzo longitudinal y transversal.
\begin{itemize}
    \item Amarre sísmico (compresión y tracción).
    \item Para control de asentamientos diferenciales.
    \item Momentos de empotramiento de columnas.
\end{itemize}
Además se deben conocer  las cargas axiales ultimas, $P_{u}$ de cada una de las columnas:
% Table generated by Excel2LaTeX from sheet 'vigas de amarre'
\begin{table}[H]
  \centering
    \begin{tabular}{|l|r|}
    \hline
    \rowcolor[rgb]{ 1,  .949,  .8} \multicolumn{1}{|c|}{\textbf{COLUMNA}} & \multicolumn{1}{c|}{\textbf{Pu (kN)}} \bigstrut\\
    \hline
    A1  & 827.6076 \bigstrut\\
    \hline
    A2  & 2098.5588 \bigstrut\\
    \hline
    A3  & 2108.1896 \bigstrut\\
    \hline
    A4  & 843.2288 \bigstrut\\
    \hline
    B1  & 982.4728 \bigstrut\\
    \hline
    B2  & 2685.9752 \bigstrut\\
    \hline
    B3  & 3127.9628 \bigstrut\\
    \hline
    B4  & 1900.7932 \bigstrut\\
    \hline
    C1  & 998.8048 \bigstrut\\
    \hline
    C2  & 2765.092 \bigstrut\\
    \hline
    C3  & 3215.848 \bigstrut\\
    \hline
    C4  & 2105.2624 \bigstrut\\
    \hline
    D1  & 1108.2608 \bigstrut\\
    \hline
    D2  & 2405.9752 \bigstrut\\
    \hline
    D3  & 2403.31 \bigstrut\\
    \hline
    D4  & 1351.896 \bigstrut\\
    \hline
    \end{tabular}%
    \caption{Cargas axiales de cada Columna que llega a la cimentación}
  \label{tab:cargascolumnas}%
\end{table}%

El criterio de amarre sísmico sirve para verificar la cantidad de refuerzo mínimo longitudinal. Mientras con el asentamiento diferencial y los momentos de empotramiento de las columnas se determina el momento y cortante aplicados en las vigas, que tienen como fin poder calcular el refuerzo necesario.\\

\subsection{Criterio de amarre sísmico}
Siguiendo los requerimientos establecidos en $A.3.6.4.2. vigas~ de~ amarre~ en~ la~ cimentación$, de la $NSR-10$. Se tiene que para la carga máxima axial modelada:
\begin{equation*}
    \Delta P_{u}=0.25\cdot A_{a}\cdot P_{u}
\end{equation*}
\begin{equation*}
    \Delta P_{u}=0.25\cdot 0.15 \cdot 3215.85 kN=120.60~kN
\end{equation*}
\begin{itemize}
    \item Compresión
\end{itemize}
\begin{equation*}
    \phi P_{n_{max}}=\phi \cdot 0.75 \cdot(0.85\cdot f'_{c}\cdot(A_{g}-A_{st}))+A_{st}f_{y}
\end{equation*}
\begin{equation*}
    \phi P_{n~max}=\phi \cdot 0.75 \cdot(0.85\cdot 28~MPa\cdot((0.5~m\cdot 0.65~m)-(1548~mm^2\cdot 2))+(1548~mm^2\cdot 2)\cdot 420~MPa
\end{equation*}
\begin{equation*}
    \phi P_{n_{max}}=4368.8~kN
\end{equation*}
\begin{itemize}
    \item Tracción
\end{itemize}
\begin{equation*}
    \phi P_{n~max}=\phi\cdot A_{st}\cdot f_{y}
\end{equation*}
\begin{equation*}
    \phi P_{n_{max}}=0.9 \cdot 3096*mm^2 \cdot 420~MPa=1170.3~kN
\end{equation*}
En ambos casos los valores para compresión y tracción son mayores a $0.25A_{a}$ la carga vertical total del elemento que tiene la mayor carga. Por lo que se cumple con el requisito de $A.3.6.4.2$

\subsection{Control de asentamientos diferenciales}
 Por medio del uso de la herramienta SAP 2000, se construye una cuadrícula con apoyos de segundo orden. 
Para las cargas en el modelo se toma el 15 \% de las cargas totales obtenidas para cada columna.igualmente para el análisis del modelo se procedió a hacer varias simulaciones removiendo en cada una un apoyo distinto. Se decide diseñar la viga de amarre C, debido a que esta es la que presenta la mayor carga 
% Table generated by Excel2LaTeX from sheet 'vigas de amarre'
\begin{table}[H]
  \centering
    \resizebox{.95\linewidth}{!}{\begin{tabular}{|c|r|r|r|r|r|c|r|r|r|r|}
\cline{2-5}\cline{8-11}    \multicolumn{1}{r|}{} & \multicolumn{4}{c|}{\cellcolor[rgb]{ 1,  .949,  .8}\textbf{Pu}} & \multicolumn{1}{r}{} &     & \multicolumn{4}{c|}{\cellcolor[rgb]{ 1,  .949,  .8}\textbf{0.15Pu - Cargas del modelo}} \bigstrut\\
\cline{1-5}\cline{7-11}    \rowcolor[rgb]{ 1,  .949,  .8}     & \multicolumn{1}{c|}{\textbf{1}} & \multicolumn{1}{c|}{\textbf{2}} & \multicolumn{1}{c|}{\textbf{3}} & \multicolumn{1}{c|}{\textbf{4}} & \cellcolor[rgb]{ 1,  1,  1} &     & \multicolumn{1}{c|}{\textbf{1}} & \multicolumn{1}{c|}{\textbf{2}} & \multicolumn{1}{c|}{\textbf{3}} & \multicolumn{1}{c|}{\textbf{4}} \bigstrut\\
\cline{1-5}\cline{7-11}    \rowcolor[rgb]{ 1,  .949,  .8} \textbf{A} & \cellcolor[rgb]{ 1,  1,  1}827.6076 & \cellcolor[rgb]{ 1,  1,  1}2098.5588 & \cellcolor[rgb]{ 1,  1,  1}2108.1896 & \cellcolor[rgb]{ 1,  1,  1}843.2288 & \cellcolor[rgb]{ 1,  1,  1} & \textbf{A} & \cellcolor[rgb]{ 1,  1,  1}124.14114 & \cellcolor[rgb]{ 1,  1,  1}314.78382 & \cellcolor[rgb]{ 1,  1,  1}316.22844 & \cellcolor[rgb]{ 1,  1,  1}126.48432 \bigstrut\\
\cline{1-5}\cline{7-11}    \rowcolor[rgb]{ 1,  .949,  .8} \textbf{B} & \cellcolor[rgb]{ 1,  1,  1}982.4728 & \cellcolor[rgb]{ 1,  1,  1}2685.9752 & \cellcolor[rgb]{ 1,  1,  1}3127.9628 & \cellcolor[rgb]{ 1,  1,  1}1900.7932 & \cellcolor[rgb]{ 1,  1,  1} & \textbf{B} & \cellcolor[rgb]{ 1,  1,  1}147.37092 & \cellcolor[rgb]{ 1,  1,  1}402.89628 & \cellcolor[rgb]{ 1,  1,  1}469.19442 & \cellcolor[rgb]{ 1,  1,  1}285.11898 \bigstrut\\
\cline{1-5}\cline{7-11}    \rowcolor[rgb]{ 1,  .949,  .8} \textbf{C} & \cellcolor[rgb]{ .973,  .796,  .678}998.8048 & \cellcolor[rgb]{ .973,  .796,  .678}2765.092 & \cellcolor[rgb]{ .973,  .796,  .678}3215.848 & \cellcolor[rgb]{ .973,  .796,  .678}2105.2624 & \cellcolor[rgb]{ 1,  1,  1} & \textbf{C} & \cellcolor[rgb]{ .973,  .796,  .678}149.82072 & \cellcolor[rgb]{ .973,  .796,  .678}414.7638 & \cellcolor[rgb]{ .973,  .796,  .678}482.3772 & \cellcolor[rgb]{ .973,  .796,  .678}315.78936 \bigstrut\\
\cline{1-5}\cline{7-11}    \rowcolor[rgb]{ 1,  .949,  .8} \textbf{D} & \cellcolor[rgb]{ 1,  1,  1}1108.2608 & \cellcolor[rgb]{ 1,  1,  1}2405.9752 & \cellcolor[rgb]{ 1,  1,  1}2403.31 & \cellcolor[rgb]{ 1,  1,  1}1351.896 & \cellcolor[rgb]{ 1,  1,  1} & \textbf{D} & \cellcolor[rgb]{ 1,  1,  1}166.23912 & \cellcolor[rgb]{ 1,  1,  1}360.89628 & \cellcolor[rgb]{ 1,  1,  1}360.4965 & \cellcolor[rgb]{ 1,  1,  1}202.7844 \bigstrut\\
\cline{1-5}\cline{7-11}    \end{tabular}}%
  \caption{Esquema y distribución de Columnas y sus respectivas cargas}
  \label{tab:Cargasa vigas amarre}%
\end{table}%

 A continuación se muestra el esquema general obtenido con el uso del software:
 \begin{figure}[H]
    \centering
    \includegraphics[scale=1]{images/vigas_de_cimentacion/Modelo_saps.png}
    \caption{Esquema y distribución de elementos estructurales y sus respectivas cargas}
\end{figure}

\newpage
\begin{itemize}
    \item Removiendo apoyo $1$
\end{itemize}

\begin{figure}[H]
\begin{subfigure}{.5\textwidth}
  \centering
  % include first image
  \includegraphics[width=.8\linewidth]{images/vigas_de_cimentacion/apoyo_1_a.png}  
  \caption{Cargas aplicadas con los apoyos correspondientes}
  \label{fig:Cargas removiendo AP1}
\end{subfigure}
\begin{subfigure}{.5\textwidth}
  \centering
  % include second image
  \includegraphics[width=.8\linewidth]{images/vigas_de_cimentacion/apoyo_1_b.png}  
  \caption{Deformada al aplicar la carga}
  \label{fig:Deformada AP1}
\end{subfigure}

\begin{subfigure}{.5\textwidth}
  \centering
  % include third image
  \includegraphics[width=.8\linewidth]{images/vigas_de_cimentacion/apoyo_1_c.png}  
  \caption{Diagrama de cortante viga de amarre C.}
  \label{fig:Cortante AP1}
\end{subfigure}
\begin{subfigure}{.5\textwidth}
  \centering
  % include fourth image
  \includegraphics[width=.8\linewidth]{images/vigas_de_cimentacion/apoyo_1_d.png}  
  \caption{Diagrama de momento viga de amarre C.}
  \label{fig:Momento AP1}
\end{subfigure}
\caption{Resultados al remover el apoyo 1.}
\label{fig:R apoyo 1}
\end{figure}


\newpage
\begin{itemize}
    \item Removiendo apoyo $2$
\end{itemize}

\begin{figure}[H]
\begin{subfigure}{.5\textwidth}
  \centering
  % include first image
  \includegraphics[width=.8\linewidth]{images/vigas_de_cimentacion/apoyo_2_a.png}  
  \caption{Cargas aplicadas con los apoyos correspondientes}
  \label{fig:Cargas removiendo AP2}
\end{subfigure}
\begin{subfigure}{.5\textwidth}
  \centering
  % include second image
  \includegraphics[width=.8\linewidth]{images/vigas_de_cimentacion/apoyo_2_b.png}  
  \caption{Deformada al aplicar la carga}
  \label{fig:Deformada AP2}
\end{subfigure}

\begin{subfigure}{.5\textwidth}
  \centering
  % include third image
  \includegraphics[width=.8\linewidth]{images/vigas_de_cimentacion/apoyo_2_c.png}  
  \caption{Diagrama de cortante viga de amarre C.}
  \label{fig:Cortante AP2}
\end{subfigure}
\begin{subfigure}{.45\textwidth}
  \centering
  % include fourth image
  \includegraphics[width=.8\linewidth]{images/vigas_de_cimentacion/apoyo_2_d.png}  
  \caption{Diagrama de momento viga de amarre C.}
  \label{fig:Momento AP2}
\end{subfigure}
\caption{Resultados al remover el apoyo 2.}
\label{fig:R apoyo 2}
\end{figure}


\newpage
\begin{itemize}
    \item Removiendo apoyo $3$
\end{itemize}

\begin{figure}[H]
\begin{subfigure}{.5\textwidth}
  \centering
  % include first image
  \includegraphics[width=.8\linewidth]{images/vigas_de_cimentacion/apoyo_3_a.png}  
  \caption{Cargas aplicadas con los apoyos correspondientes}
  \label{fig:Cargas removiendo AP3}
\end{subfigure}
\begin{subfigure}{.5\textwidth}
  \centering
  % include second image
  \includegraphics[width=.8\linewidth]{images/vigas_de_cimentacion/apoyo_3_b.png}  
  \caption{Deformada al aplicar la carga}
  \label{fig:Deformada AP3}
\end{subfigure}

\begin{subfigure}{.5\textwidth}
  \centering
  % include third image
  \includegraphics[width=.8\linewidth]{images/vigas_de_cimentacion/apoyo_3_c.png}  
  \caption{Diagrama de cortante viga de amarre C.}
  \label{fig:Cortante AP3}
\end{subfigure}
\begin{subfigure}{.5\textwidth}
  \centering
  % include fourth image
  \includegraphics[width=.8\linewidth]{images/vigas_de_cimentacion/apoyo_3_d.png}  
  \caption{Diagrama de momento viga de amarre C.}
  \label{fig:Momento AP1}
\end{subfigure}
\caption{Resultados al remover el apoyo 3.}
\label{fig:R apoyo 3}
\end{figure}



\newpage
\begin{itemize}
    \item Removiendo apoyo $4$
\end{itemize}

\begin{figure}[H]
\begin{subfigure}{.5\textwidth}
  \centering
  % include first image
  \includegraphics[width=.8\linewidth]{images/vigas_de_cimentacion/apoyo_4_a.png}  
  \caption{Cargas aplicadas con los apoyos correspondientes}
  \label{fig:Cargas removiendo AP4}
\end{subfigure}
\begin{subfigure}{.5\textwidth}
  \centering
  % include second image
  \includegraphics[width=.8\linewidth]{images/vigas_de_cimentacion/apoyo_4_b.png}  
  \caption{Deformada al aplicar la carga}
  \label{fig:Deformada AP4}
\end{subfigure}

\begin{subfigure}{.5\textwidth}
  \centering
  % include third image
  \includegraphics[width=.8\linewidth]{images/vigas_de_cimentacion/apoyo_4_c.png}  
  \caption{Diagrama de cortante viga de amarre C.}
  \label{fig:Cortante AP4}
\end{subfigure}
\begin{subfigure}{.5\textwidth}
  \centering
  % include fourth image
  \includegraphics[width=.8\linewidth]{images/vigas_de_cimentacion/apoyo_4_d.png}  
  \caption{Diagrama de momento viga de amarre C.}
  \label{fig:Momento AP4}
\end{subfigure}
\caption{Resultados al remover el apoyo 4.}
\label{fig:R apoyo 4}
\end{figure}
A partir de los resultados del modelo se obtuvieron los valores de los momentos máximos, positivos y negativos para cada uno de los escenarios evaluados.
\subsubsection{Diseño a flexión}
Para el caso en donde se remueve el apoyo 1. Se diseña para un momento positivo igual a $M_{u}=51.55~kNm$.Conociendo las siguientes propiedades:

\begin{table}[htbp]
  \centering
    \begin{tabular}{|c|c|}
    \hline
    \textbf{$f'_{c} (MPa)$} & 28 \bigstrut\\
    \hline
    \textbf{$f{y} (MPa)$} & 420 \bigstrut\\
    \hline
    $b (m)$   & 0.5 \bigstrut\\
    \hline
    $h (m)$  & 0.65 \bigstrut\\
    \hline
    $d (m)$  & 0.5544 \bigstrut\\
    \hline
    \end{tabular}%
  \label{tab:addlabel}%
\end{table}%


Entonces la cuantía de refuerzo es igual a:
\begin{equation*}
    \rho=\left[1-\sqrt{1-\frac{2.62\cdot M_{u}}{B\cdot d^2\cdot f'_{c}}}\right]\cdot \frac{f'_{c}}{1.18 f_{yt}}
\end{equation*}
\begin{equation*}
    \rho=\left[1-\sqrt{1-\frac{2.62\cdot 51.55~kNm}{0.5~m\cdot (0.5544~m)^2\cdot 28\cdot 10^3 kPa}}\right]\cdot \frac{28~MPa}{1.18 (420~MPa)}=0.00089
\end{equation*}

En este caso el valor obtenido es menor a la cuantía mínima por lo que para el caso se asumirá la cuantía mínima, como la cuantía de refuerzo:
\begin{equation*}
    \rho_{min}=0.00333
\end{equation*}

De esta manera el área de refuerzo requerida es:
\begin{equation*}
    \ A_{s}=0.00333\cdot 500~mm\cdot 650~mm=924~mm^2
\end{equation*}
Se proponen 2 barras No. 8 y se recalcula, para obtener el área de refuerzo suministrado:
\begin{equation*}
    \ A_{suministrado}=2\cdot 510~mm^2=1020~mm^2
\end{equation*}
Siendo este ultimo mayor a la cuantía mínima.
Siguiendo un procedimiento igual para el momento negativo de $101.97~kNm$ se obtiene que, nuevamente la cuantía es igual a la cuantía mínima:
\begin{equation*}
    \rho_{min}=0.00333
\end{equation*}

El área de refuerzo:
\begin{equation*}
    \ A_{s}=924~mm^2
\end{equation*}

Para este caso se proponen 3 barras No.7, para un área suministrada de:
\begin{equation*}
    \ A_{suministrado}=1161~mm^2
\end{equation*}

A continuación se muestra el resumen para los cuatro casos que se modelaron:
% Table generated by Excel2LaTeX from sheet 'vigas de amarre'
\begin{table}[H]
  \centering
  \resizebox{\linewidth}{!}{
    \begin{tabular}{|c|r|r|c|c|c|c|c|c|}
\cline{4-9}    \multicolumn{1}{r}{} & \multicolumn{1}{r}{} &     & \multicolumn{6}{c|}{\textbf{DISEÑO A FLEXIÓN}} \bigstrut\\
\cline{4-9}    \multicolumn{1}{r}{} & \multicolumn{1}{r}{} &     & \multicolumn{6}{c|}{\textbf{MOMENTO POSITIVO}} \bigstrut\\
    \hline
    \rowcolor[rgb]{ 1,  .949,  .8} \textbf{SIN EL APOYO} & \multicolumn{1}{c|}{\textbf{M+}} & \multicolumn{1}{c|}{\textbf{M-}} & \cellcolor[rgb]{ 1,  1,  1}\textbf{$\rho$} & \cellcolor[rgb]{ 1,  1,  1}\textbf{As (mm²)} & \cellcolor[rgb]{ 1,  1,  1}\textbf{Barra \#} & \cellcolor[rgb]{ 1,  1,  1}\textbf{Cantidad} & \cellcolor[rgb]{ 1,  1,  1}\textbf{$A_{suministrada}$} & \cellcolor[rgb]{ 1,  1,  1}\textbf{$As_{sum}>As_{req}$} \bigstrut\\
    \hline
    \textbf{1} & \multicolumn{1}{c|}{51.5467} & \multicolumn{1}{c|}{101.9745} & 0.0033333 & 924 & 8   & 2   & 1020 & \cellcolor[rgb]{ .776,  .937,  .808}\textcolor[rgb]{ 0,  .38,  0}{cumple} \bigstrut\\
    \hline
    \textbf{2} & 354.2352 & 254.118 & 0.0064624 & 1791 & 8   & 4   & 2040 & \cellcolor[rgb]{ .776,  .937,  .808}\textcolor[rgb]{ 0,  .38,  0}{cumple} \bigstrut\\
    \hline
    \textbf{3} & 410.4681 & 297.5059 & 0.0075667 & 2097 & 8   & 4   & 2427 & \cellcolor[rgb]{ .776,  .937,  .808}\textcolor[rgb]{ 0,  .38,  0}{cumple} \bigstrut\\
    \hline
    \textbf{4} & 107.4191 & 211.727 & 0.0033333 & 924 & 8   & 2   & 1020 & \cellcolor[rgb]{ .776,  .937,  .808}\textcolor[rgb]{ 0,  .38,  0}{cumple} \bigstrut\\
    \hline
    \end{tabular}}%
     \caption{Diseño a flexión, momento positivo}
  \label{tab:diseñoflexion}%
\end{table}%
% Table generated by Excel2LaTeX from sheet 'vigas de amarre'
\begin{table}[H]
  \centering
  \resizebox{\linewidth}{!}{
    \begin{tabular}{|c|r|r|c|c|c|c|c|c|}
\cline{4-9}    \multicolumn{1}{r}{} & \multicolumn{1}{r}{} &     & \multicolumn{6}{c|}{\textbf{DISEÑO A FLEXIÓN}} \bigstrut\\
\cline{4-9}    \multicolumn{1}{r}{} & \multicolumn{1}{r}{} &     & \multicolumn{6}{c|}{\textbf{MOMENTO NEGATIVO}} \bigstrut\\
    \hline
    \rowcolor[rgb]{ 1,  .949,  .8} \textbf{SIN EL APOYO} & \multicolumn{1}{c|}{\textbf{M+}} & \multicolumn{1}{c|}{\textbf{M-}} & \cellcolor[rgb]{ 1,  1,  1}\textbf{$\rho$} & \cellcolor[rgb]{ 1,  1,  1}\textbf{As1} & \cellcolor[rgb]{ 1,  1,  1}\textbf{Barra \#} & \cellcolor[rgb]{ 1,  1,  1}\textbf{Cantidad} & \cellcolor[rgb]{ 1,  1,  1}\textbf{$A_{suministrada}$} & \cellcolor[rgb]{ 1,  1,  1}\textbf{$As_{sum}>As_{req}$} \bigstrut\\
    \hline
    \textbf{1} & \multicolumn{1}{c|}{51.5467} & \multicolumn{1}{c|}{101.9745} & 0.0033333 & 924 & 7   & 3   & 1161 & \cellcolor[rgb]{ .776,  .937,  .808}\textcolor[rgb]{ 0,  .38,  0}{cumple} \bigstrut\\
    \hline
    \textbf{2} & 354.2352 & 254.118 & 0.0045543 & 1262 & 7   & 4   & 1548 & \cellcolor[rgb]{ .776,  .937,  .808}\textcolor[rgb]{ 0,  .38,  0}{cumple} \bigstrut\\
    \hline
    \textbf{3} & 410.4681 & 297.5059 & 0.0053725 & 1489 & 7   & 4   & 1548 & \cellcolor[rgb]{ .776,  .937,  .808}\textcolor[rgb]{ 0,  .38,  0}{cumple} \bigstrut\\
    \hline
    \textbf{4} & 107.4191 & 211.727 & 0.0037673 & 1044 & 7   & 3   & 1161 & \cellcolor[rgb]{ .776,  .937,  .808}\textcolor[rgb]{ 0,  .38,  0}{cumple} \bigstrut\\
    \hline
    \end{tabular}}%
    \caption{Diseño a flexión, momento negativo}
  \label{tab:flexion m negativo}%
\end{table}%


\subsubsection{Diseño a cortante}
Se diseña para un valor de cortante igual a $V_{u}=22.60~kNm$
\begin{equation*}
    \phi V_{c}=0.75\cdot 0.17\cdot(1)\sqrt{28~MPa}\cdot 0.5~m \cdot 0.5544~m\cdot 10^3=187.018 kN
\end{equation*}
Como $V_{u}<\phi V$ entonces no se requiere refuerzo a cortante para el primer escenario.
% Table generated by Excel2LaTeX from sheet 'vigas de amarre'
\begin{table}[H]
  \centering
  
    \begin{tabular}{|c|c|c|c|}
    \hline
    \rowcolor[rgb]{ 1,  .949,  .8} \textbf{SIN EL APOYO} & \textbf{$V_u (kN)$} & \cellcolor[rgb]{ 1,  1,  1}\textbf{$\phi V_c (kN)$} & \multicolumn{1}{p{5.355em}|}{\cellcolor[rgb]{ 1,  1,  1}\textbf{Requiere refuerzo}} \bigstrut\\
    \hline
    \textbf{1} & 22.577 & 187.018 & no \bigstrut\\
    \hline
    \textbf{2} & 86.908 & 187.018 & no \bigstrut\\
    \hline
    \textbf{3} & 101.139 & 187.018 & min \bigstrut\\
    \hline
    \textbf{4} & 46.253 & 187.018 & no \bigstrut\\
    \hline
    \end{tabular}%
    \caption{Requisitos de cortante para las vigas de amarre}
  \label{tab:requisitoscortante}%
\end{table}%

\subsection{Momentos de empotramiento}
Finalmente, se realiza un tercer chequeo teniendo en cuenta los momentos de empotramiento de las columnas. Para esto, se realiza el siguiente modelo en el software SAP 2000, teniendo en cuenta los momentos de empotramiento de las columnas, dadas por el modelo en 3D del edificio.
\begin{figure}[H]
    \centering
    \includegraphics[width=\linewidth]{images/modeloempotramiento.pdf}
    \caption{Modelo para momentos de empotramiento}
    \label{fig:empotramiento}
\end{figure}

A partir de los momentos que ocurren en la viga encontrados a partir del anterior modelo se encuentra que la cuantía $\rho$ tanto para momento positivo como para negativo corresponde a la mínima.

\begin{table}[H]
    \centering
    % Table generated by Excel2LaTeX from sheet 'Mempotramiento'
    \resizebox{\linewidth}{!}{
    
    % Table generated by Excel2LaTeX from sheet 'Mempotramiento'
\begin{tabular}{|c|c|c|c|c|c|c|c|c|c|c|c|c|c|c|c|c|}
\hline
\rowcolor[rgb]{ .2,  .8,  .8} \multicolumn{5}{|c|}{\textbf{TABLE:  Element Forces - Frames}} & \multicolumn{6}{c|}{\cellcolor[rgb]{ 1,  1,  1}\textbf{MOMENTO POSITIVO}} & \multicolumn{6}{c|}{\cellcolor[rgb]{ 1,  1,  1}\textbf{MOMENTO NEGATIVO}} \bigstrut\\
\hline
\rowcolor[rgb]{ .8,  1,  1} \textbf{Frame} & \textbf{Station} & \textbf{V2} & \textbf{M3+} & \textbf{M3-} & \cellcolor[rgb]{ 1,  1,  1}\boldmath{}\textbf{$\rho$}\unboldmath{} & \cellcolor[rgb]{ 1,  1,  1}\boldmath{}\textbf{$A_s (mm^2)$}\unboldmath{} & \cellcolor[rgb]{ 1,  1,  1}\textbf{Barra \#} & \cellcolor[rgb]{ 1,  1,  1}\textbf{Cantidad} & \cellcolor[rgb]{ 1,  1,  1}\boldmath{}\textbf{$A_{suministrada}$}\unboldmath{} & \cellcolor[rgb]{ 1,  1,  1}\textbf{Assum>Asreq} & \cellcolor[rgb]{ 1,  1,  1}\boldmath{}\textbf{$\rho$}\unboldmath{} & \cellcolor[rgb]{ 1,  1,  1}\boldmath{}\textbf{$A_s (mm^2)$}\unboldmath{} & \cellcolor[rgb]{ 1,  1,  1}\textbf{Barra \#} & \cellcolor[rgb]{ 1,  1,  1}\textbf{Cantidad} & \cellcolor[rgb]{ 1,  1,  1}\boldmath{}\textbf{$A_{suministrada}$}\unboldmath{} & \cellcolor[rgb]{ 1,  1,  1}\textbf{Assum>Asreq} \bigstrut\\
\hline
1   & 0   & 2.673 & 14.4849 & -   & 0.003333 & 924 & 8   & 2   & 1020 & \cellcolor[rgb]{ .776,  .937,  .808}\textcolor[rgb]{ 0,  .38,  0}{cumple} & -   & -   &     &     &     & NA \bigstrut\\
\hline
1   & 0.48571 & 2.673 & 13.1868 & -   & 0.003333 & 924 & 8   & 2   & 1020 & \cellcolor[rgb]{ .776,  .937,  .808}\textcolor[rgb]{ 0,  .38,  0}{cumple} & -   & -   &     &     &     & NA \bigstrut\\
\hline
1   & 0.97143 & 2.673 & 11.8887 & -   & 0.003333 & 924 & 8   & 2   & 1020 & \cellcolor[rgb]{ .776,  .937,  .808}\textcolor[rgb]{ 0,  .38,  0}{cumple} & -   & -   &     &     &     & NA \bigstrut\\
\hline
1   & 1.45714 & 2.673 & 10.5906 & -   & 0.003333 & 924 & 8   & 2   & 1020 & \cellcolor[rgb]{ .776,  .937,  .808}\textcolor[rgb]{ 0,  .38,  0}{cumple} & -   & -   &     &     &     & NA \bigstrut\\
\hline
1   & 1.94286 & 2.673 & 9.2925 & -   & 0.003333 & 924 & 8   & 2   & 1020 & \cellcolor[rgb]{ .776,  .937,  .808}\textcolor[rgb]{ 0,  .38,  0}{cumple} & -   & -   &     &     &     & NA \bigstrut\\
\hline
1   & 2.42857 & 2.673 & 7.9944 & -   & 0.003333 & 924 & 8   & 2   & 1020 & \cellcolor[rgb]{ .776,  .937,  .808}\textcolor[rgb]{ 0,  .38,  0}{cumple} & -   & -   &     &     &     & NA \bigstrut\\
\hline
1   & 2.91429 & 2.673 & 6.6964 & -   & 0.003333 & 924 & 8   & 2   & 1020 & \cellcolor[rgb]{ .776,  .937,  .808}\textcolor[rgb]{ 0,  .38,  0}{cumple} & -   & -   &     &     &     & NA \bigstrut\\
\hline
1   & 3.4 & 2.673 & 5.3983 & -   & 0.003333 & 924 & 8   & 2   & 1020 & \cellcolor[rgb]{ .776,  .937,  .808}\textcolor[rgb]{ 0,  .38,  0}{cumple} & -   & -   &     &     &     & NA \bigstrut\\
\hline
1   & 3.88571 & 2.673 & 4.1002 & -   & 0.003333 & 924 & 8   & 2   & 1020 & \cellcolor[rgb]{ .776,  .937,  .808}\textcolor[rgb]{ 0,  .38,  0}{cumple} & -   & -   &     &     &     & NA \bigstrut\\
\hline
1   & 4.37143 & 2.673 & 2.8021 & -   & 0.003333 & 924 & 8   & 2   & 1020 & \cellcolor[rgb]{ .776,  .937,  .808}\textcolor[rgb]{ 0,  .38,  0}{cumple} & -   & -   &     &     &     & NA \bigstrut\\
\hline
1   & 4.85714 & 2.673 & 1.504 & -   & 0.003333 & 924 & 8   & 2   & 1020 & \cellcolor[rgb]{ .776,  .937,  .808}\textcolor[rgb]{ 0,  .38,  0}{cumple} & -   & -   &     &     &     & NA \bigstrut\\
\hline
1   & 5.34286 & 2.673 & 0.2059 & -   & 0.003333 & 924 & 8   & 2   & 1020 & \cellcolor[rgb]{ .776,  .937,  .808}\textcolor[rgb]{ 0,  .38,  0}{cumple} & -   & -   &     &     &     & NA \bigstrut\\
\hline
1   & 5.82857 & 2.673 & -   & -1.0922 & -   & -   & -   &     &     & NA  & 0.003333 & 924 & 8   & 2   & 1020 & \cellcolor[rgb]{ .776,  .937,  .808}\textcolor[rgb]{ 0,  .38,  0}{cumple} \bigstrut\\
\hline
1   & 6.31429 & 2.673 & -   & -2.3903 & -   & -   & -   &     &     & NA  & 0.003333 & 924 & 8   & 2   & 1020 & \cellcolor[rgb]{ .776,  .937,  .808}\textcolor[rgb]{ 0,  .38,  0}{cumple} \bigstrut\\
\hline
1   & 6.8 & 2.673 & -   & -3.6884 & -   & -   & -   &     &     & NA  & 0.003333 & 924 & 8   & 2   & 1020 & \cellcolor[rgb]{ .776,  .937,  .808}\textcolor[rgb]{ 0,  .38,  0}{cumple} \bigstrut\\
\hline
2   & 6.8 & 0.069 & -   & -2.0645 & -   & -   & -   &     &     & NA  & 0.003333 & 924 & 8   & 2   & 1020 & \cellcolor[rgb]{ .776,  .937,  .808}\textcolor[rgb]{ 0,  .38,  0}{cumple} \bigstrut\\
\hline
2   & 7.3 & 0.069 & -   & -2.0991 & -   & -   & -   &     &     & NA  & 0.003333 & 924 & 8   & 2   & 1020 & \cellcolor[rgb]{ .776,  .937,  .808}\textcolor[rgb]{ 0,  .38,  0}{cumple} \bigstrut\\
\hline
2   & 7.8 & 0.069 & -   & -2.1338 & -   & -   & -   &     &     & NA  & 0.003333 & 924 & 8   & 2   & 1020 & \cellcolor[rgb]{ .776,  .937,  .808}\textcolor[rgb]{ 0,  .38,  0}{cumple} \bigstrut\\
\hline
2   & 8.3 & 0.069 & -   & -2.1685 & -   & -   & -   &     &     & NA  & 0.003333 & 924 & 8   & 2   & 1020 & \cellcolor[rgb]{ .776,  .937,  .808}\textcolor[rgb]{ 0,  .38,  0}{cumple} \bigstrut\\
\hline
2   & 8.8 & 0.069 & -   & -2.2032 & -   & -   & -   &     &     & NA  & 0.003333 & 924 & 8   & 2   & 1020 & \cellcolor[rgb]{ .776,  .937,  .808}\textcolor[rgb]{ 0,  .38,  0}{cumple} \bigstrut\\
\hline
2   & 9.3 & 0.069 & -   & -2.2378 & -   & -   & -   &     &     & NA  & 0.003333 & 924 & 8   & 2   & 1020 & \cellcolor[rgb]{ .776,  .937,  .808}\textcolor[rgb]{ 0,  .38,  0}{cumple} \bigstrut\\
\hline
2   & 9.8 & 0.069 & -   & -2.2725 & -   & -   & -   &     &     & NA  & 0.003333 & 924 & 8   & 2   & 1020 & \cellcolor[rgb]{ .776,  .937,  .808}\textcolor[rgb]{ 0,  .38,  0}{cumple} \bigstrut\\
\hline
2   & 10.3 & 0.069 & -   & -2.3072 & -   & -   & -   &     &     & NA  & 0.003333 & 924 & 8   & 2   & 1020 & \cellcolor[rgb]{ .776,  .937,  .808}\textcolor[rgb]{ 0,  .38,  0}{cumple} \bigstrut\\
\hline
2   & 10.8 & 0.069 & -   & -2.3419 & -   & -   & -   &     &     & NA  & 0.003333 & 924 & 8   & 2   & 1020 & \cellcolor[rgb]{ .776,  .937,  .808}\textcolor[rgb]{ 0,  .38,  0}{cumple} \bigstrut\\
\hline
2   & 11.3 & 0.069 & -   & -2.3765 & -   & -   & -   &     &     & NA  & 0.003333 & 924 & 8   & 2   & 1020 & \cellcolor[rgb]{ .776,  .937,  .808}\textcolor[rgb]{ 0,  .38,  0}{cumple} \bigstrut\\
\hline
2   & 11.8 & 0.069 & -   & -2.4112 & -   & -   & -   &     &     & NA  & 0.003333 & 924 & 8   & 2   & 1020 & \cellcolor[rgb]{ .776,  .937,  .808}\textcolor[rgb]{ 0,  .38,  0}{cumple} \bigstrut\\
\hline
2   & 12.3 & 0.069 & -   & -2.4459 & -   & -   & -   &     &     & NA  & 0.003333 & 924 & 8   & 2   & 1020 & \cellcolor[rgb]{ .776,  .937,  .808}\textcolor[rgb]{ 0,  .38,  0}{cumple} \bigstrut\\
\hline
2   & 12.8 & 0.069 & -   & -2.4806 & -   & -   & -   &     &     & NA  & 0.003333 & 924 & 8   & 2   & 1020 & \cellcolor[rgb]{ .776,  .937,  .808}\textcolor[rgb]{ 0,  .38,  0}{cumple} \bigstrut\\
\hline
2   & 13.3 & 0.069 & -   & -2.5152 & -   & -   & -   &     &     & NA  & 0.003333 & 924 & 8   & 2   & 1020 & \cellcolor[rgb]{ .776,  .937,  .808}\textcolor[rgb]{ 0,  .38,  0}{cumple} \bigstrut\\
\hline
2   & 13.8 & 0.069 & -   & -2.5499 & -   & -   & -   &     &     & NA  & 0.003333 & 924 & 8   & 2   & 1020 & \cellcolor[rgb]{ .776,  .937,  .808}\textcolor[rgb]{ 0,  .38,  0}{cumple} \bigstrut\\
\hline
3   & 13.8 & -3.934 & -   & -6.5081 & -   & -   & -   &     &     & NA  & 0.003333 & 924 & 8   & 2   & 1020 & \cellcolor[rgb]{ .776,  .937,  .808}\textcolor[rgb]{ 0,  .38,  0}{cumple} \bigstrut\\
\hline
3   & 14.29286 & -3.934 & -   & -4.5691 & -   & -   & -   &     &     & NA  & 0.003333 & 924 & 8   & 2   & 1020 & \cellcolor[rgb]{ .776,  .937,  .808}\textcolor[rgb]{ 0,  .38,  0}{cumple} \bigstrut\\
\hline
3   & 14.78571 & -3.934 & -   & -2.6302 & -   & -   & -   &     &     & NA  & 0.003333 & 924 & 8   & 2   & 1020 & \cellcolor[rgb]{ .776,  .937,  .808}\textcolor[rgb]{ 0,  .38,  0}{cumple} \bigstrut\\
\hline
3   & 15.27857 & -3.934 & -   & -0.6912 & -   & -   & -   &     &     & NA  & 0.003333 & 924 & 8   & 2   & 1020 & \cellcolor[rgb]{ .776,  .937,  .808}\textcolor[rgb]{ 0,  .38,  0}{cumple} \bigstrut\\
\hline
3   & 15.77143 & -3.934 & 1.2477 & -   & 0.003333 & 924 & 8   & 2   & 1020 & \cellcolor[rgb]{ .776,  .937,  .808}\textcolor[rgb]{ 0,  .38,  0}{cumple} & -   & -   &     &     &     & NA \bigstrut\\
\hline
3   & 16.26429 & -3.934 & 3.1867 & -   & 0.003333 & 924 & 8   & 2   & 1020 & \cellcolor[rgb]{ .776,  .937,  .808}\textcolor[rgb]{ 0,  .38,  0}{cumple} & -   & -   &     &     &     & NA \bigstrut\\
\hline
3   & 16.75714 & -3.934 & 5.1256 & -   & 0.003333 & 924 & 8   & 2   & 1020 & \cellcolor[rgb]{ .776,  .937,  .808}\textcolor[rgb]{ 0,  .38,  0}{cumple} & -   & -   &     &     &     & NA \bigstrut\\
\hline
3   & 17.25 & -3.934 & 7.0646 & -   & 0.003333 & 924 & 8   & 2   & 1020 & \cellcolor[rgb]{ .776,  .937,  .808}\textcolor[rgb]{ 0,  .38,  0}{cumple} & -   & -   &     &     &     & NA \bigstrut\\
\hline
3   & 17.74286 & -3.934 & 9.0036 & -   & 0.003333 & 924 & 8   & 2   & 1020 & \cellcolor[rgb]{ .776,  .937,  .808}\textcolor[rgb]{ 0,  .38,  0}{cumple} & -   & -   &     &     &     & NA \bigstrut\\
\hline
3   & 18.23571 & -3.934 & 10.9425 & -   & 0.003333 & 924 & 8   & 2   & 1020 & \cellcolor[rgb]{ .776,  .937,  .808}\textcolor[rgb]{ 0,  .38,  0}{cumple} & -   & -   &     &     &     & NA \bigstrut\\
\hline
3   & 18.72857 & -3.934 & 12.8815 & -   & 0.003333 & 924 & 8   & 2   & 1020 & \cellcolor[rgb]{ .776,  .937,  .808}\textcolor[rgb]{ 0,  .38,  0}{cumple} & -   & -   &     &     &     & NA \bigstrut\\
\hline
3   & 19.22143 & -3.934 & 14.8204 & -   & 0.003333 & 924 & 8   & 2   & 1020 & \cellcolor[rgb]{ .776,  .937,  .808}\textcolor[rgb]{ 0,  .38,  0}{cumple} & -   & -   &     &     &     & NA \bigstrut\\
\hline
3   & 19.71429 & -3.934 & 16.7594 & -   & 0.003333 & 924 & 8   & 2   & 1020 & \cellcolor[rgb]{ .776,  .937,  .808}\textcolor[rgb]{ 0,  .38,  0}{cumple} & -   & -   &     &     &     & NA \bigstrut\\
\hline
3   & 20.20714 & -3.934 & 18.6983 & -   & 0.003333 & 924 & 8   & 2   & 1020 & \cellcolor[rgb]{ .776,  .937,  .808}\textcolor[rgb]{ 0,  .38,  0}{cumple} & -   & -   &     &     &     & NA \bigstrut\\
\hline
3   & 20.7 & -3.934 & 20.6373 & -   & 0.003333 & 924 & 8   & 2   & 1020 & \cellcolor[rgb]{ .776,  .937,  .808}\textcolor[rgb]{ 0,  .38,  0}{cumple} & -   & -   &     &     &     & NA \bigstrut\\
\hline
\end{tabular}%


        
}%
    \caption{Diseño a partir de momentos de empotramiento de las columnas.}
    \label{tab:moment.emp}
\end{table}


% % Table generated by Excel2LaTeX from sheet '1'
\begin{table}[htbp]
  \centering
  \caption{Tabla de resultados del software para viga sin apoyo 1}
  \resizebox{\linewidth}{!}{
    \begin{tabular}{rrrlrrrrr}
    \rowcolor[rgb]{ .2,  .8,  .8} \multicolumn{1}{l}{\textbf{TABLE:  Element Forces - Frames}} &     &     &     &     &     & \cellcolor[rgb]{ 1,  1,  1} & \cellcolor[rgb]{ 1,  1,  1} & \cellcolor[rgb]{ 1,  1,  1} \bigstrut[b]\\
\cline{1-6}\cline{8-9}    \rowcolor[rgb]{ .8,  1,  1} \multicolumn{1}{|c|}{\textbf{Frame}} & \multicolumn{1}{c|}{\textbf{Station}} & \multicolumn{1}{c|}{\textbf{OutputCase}} & \multicolumn{1}{c|}{\textbf{CaseType}} & \multicolumn{1}{c|}{\textbf{V2}} & \multicolumn{1}{c|}{\textbf{M3}} & \multicolumn{1}{r|}{\cellcolor[rgb]{ 1,  1,  1}} & \multicolumn{1}{c|}{\textbf{V2}} & \multicolumn{1}{c|}{\textbf{M3}} \bigstrut\\
\cline{1-6}\cline{8-9}    \rowcolor[rgb]{ .8,  1,  1} \multicolumn{1}{|c|}{Text} & \multicolumn{1}{c|}{m} & \multicolumn{1}{c|}{Text} & \multicolumn{1}{c|}{Text} & \multicolumn{1}{c|}{KN} & \multicolumn{1}{c|}{KN-m} & \multicolumn{1}{r|}{\cellcolor[rgb]{ 1,  1,  1}} & \multicolumn{1}{c|}{KN} & \multicolumn{1}{c|}{KN-m} \bigstrut\\
\cline{1-6}\cline{8-9}    \multicolumn{1}{l}{7} & 0   & \multicolumn{1}{l}{PU} & LinStatic & 22.577 & 51.5467 &     & 22.577 & 51.5467 \bigstrut[t]\\
    \multicolumn{1}{l}{7} & 0.48571 & \multicolumn{1}{l}{PU} & LinStatic & 22.577 & 40.5809 &     & 22.577 & 40.5809 \\
    \multicolumn{1}{l}{7} & 0.97143 & \multicolumn{1}{l}{PU} & LinStatic & 22.577 & 29.6151 &     & 22.577 & 29.6151 \\
    \multicolumn{1}{l}{7} & 1.45714 & \multicolumn{1}{l}{PU} & LinStatic & 22.577 & 18.6493 &     & 22.577 & 18.6493 \\
    \multicolumn{1}{l}{7} & 1.94286 & \multicolumn{1}{l}{PU} & LinStatic & 22.577 & 7.6835 &     & 22.577 & 7.6835 \\
    \multicolumn{1}{l}{7} & 2.42857 & \multicolumn{1}{l}{PU} & LinStatic & 22.577 & -3.2823 &     & 22.577 & -3.2823 \\
    \multicolumn{1}{l}{7} & 2.91429 & \multicolumn{1}{l}{PU} & LinStatic & 22.577 & -14.2481 &     & 22.577 & -14.2481 \\
    \multicolumn{1}{l}{7} & 3.4 & \multicolumn{1}{l}{PU} & LinStatic & 22.577 & -25.2139 &     & 22.577 & -25.2139 \\
    \multicolumn{1}{l}{7} & 3.88571 & \multicolumn{1}{l}{PU} & LinStatic & 22.577 & -36.1797 &     & 22.577 & -36.1797 \\
    \multicolumn{1}{l}{7} & 4.37143 & \multicolumn{1}{l}{PU} & LinStatic & 22.577 & -47.1455 &     & 22.577 & -47.1455 \\
    \multicolumn{1}{l}{7} & 4.85714 & \multicolumn{1}{l}{PU} & LinStatic & 22.577 & -58.1113 &     & 22.577 & -58.1113 \\
    \multicolumn{1}{l}{7} & 5.34286 & \multicolumn{1}{l}{PU} & LinStatic & 22.577 & -69.0771 &     & 22.577 & -69.0771 \\
    \multicolumn{1}{l}{7} & 5.82857 & \multicolumn{1}{l}{PU} & LinStatic & 22.577 & -80.0429 &     & 22.577 & -80.0429 \\
    \multicolumn{1}{l}{7} & 6.31429 & \multicolumn{1}{l}{PU} & LinStatic & 22.577 & -91.0087 &     & 22.577 & -91.0087 \\
    \multicolumn{1}{l}{7} & 6.8 & \multicolumn{1}{l}{PU} & LinStatic & 22.577 & -101.9745 &     & 22.577 & -101.9745 \\
    \multicolumn{1}{l}{8} & 0   & \multicolumn{1}{l}{PU} & LinStatic & -13.806 & -74.8394 &     & -13.806 & -74.8394 \\
    \multicolumn{1}{l}{8} & 0.5 & \multicolumn{1}{l}{PU} & LinStatic & -13.806 & -67.9366 &     & -13.806 & -67.9366 \\
    \multicolumn{1}{l}{8} & 1   & \multicolumn{1}{l}{PU} & LinStatic & -13.806 & -61.0337 &     & -13.806 & -61.0337 \\
    \multicolumn{1}{l}{8} & 1.5 & \multicolumn{1}{l}{PU} & LinStatic & -13.806 & -54.1308 &     & -13.806 & -54.1308 \\
    \multicolumn{1}{l}{8} & 2   & \multicolumn{1}{l}{PU} & LinStatic & -13.806 & -47.2279 &     & -13.806 & -47.2279 \\
    \multicolumn{1}{l}{8} & 2.5 & \multicolumn{1}{l}{PU} & LinStatic & -13.806 & -40.325 &     & -13.806 & -40.325 \\
    \multicolumn{1}{l}{8} & 3   & \multicolumn{1}{l}{PU} & LinStatic & -13.806 & -33.4221 &     & -13.806 & -33.4221 \\
    \multicolumn{1}{l}{8} & 3.5 & \multicolumn{1}{l}{PU} & LinStatic & -13.806 & -26.5192 &     & -13.806 & -26.5192 \\
    \multicolumn{1}{l}{8} & 4   & \multicolumn{1}{l}{PU} & LinStatic & -13.806 & -19.6163 &     & -13.806 & -19.6163 \\
    \multicolumn{1}{l}{8} & 4.5 & \multicolumn{1}{l}{PU} & LinStatic & -13.806 & -12.7135 &     & -13.806 & -12.7135 \\
    \multicolumn{1}{l}{8} & 5   & \multicolumn{1}{l}{PU} & LinStatic & -13.806 & -5.8106 &     & -13.806 & -5.8106 \\
    \multicolumn{1}{l}{8} & 5.5 & \multicolumn{1}{l}{PU} & LinStatic & -13.806 & 1.0923 &     & -13.806 & 1.0923 \\
    \multicolumn{1}{l}{8} & 6   & \multicolumn{1}{l}{PU} & LinStatic & -13.806 & 7.9952 &     & -13.806 & 7.9952 \\
    \multicolumn{1}{l}{8} & 6.5 & \multicolumn{1}{l}{PU} & LinStatic & -13.806 & 14.8981 &     & -13.806 & 14.8981 \\
    \multicolumn{1}{l}{8} & 7   & \multicolumn{1}{l}{PU} & LinStatic & -13.806 & 21.801 &     & -13.806 & 21.801 \\
    \multicolumn{1}{l}{9} & 0   & \multicolumn{1}{l}{PU} & LinStatic & 2.608 & 16.0804 &     & 2.608 & 16.0804 \\
    \multicolumn{1}{l}{9} & 0.49286 & \multicolumn{1}{l}{PU} & LinStatic & 2.608 & 14.7948 &     & 2.608 & 14.7948 \\
    \multicolumn{1}{l}{9} & 0.98571 & \multicolumn{1}{l}{PU} & LinStatic & 2.608 & 13.5092 &     & 2.608 & 13.5092 \\
    \multicolumn{1}{l}{9} & 1.47857 & \multicolumn{1}{l}{PU} & LinStatic & 2.608 & 12.2236 &     & 2.608 & 12.2236 \\
    \multicolumn{1}{l}{9} & 1.97143 & \multicolumn{1}{l}{PU} & LinStatic & 2.608 & 10.938 &     & 2.608 & 10.938 \\
    \multicolumn{1}{l}{9} & 2.46429 & \multicolumn{1}{l}{PU} & LinStatic & 2.608 & 9.6524 &     & 2.608 & 9.6524 \\
    \multicolumn{1}{l}{9} & 2.95714 & \multicolumn{1}{l}{PU} & LinStatic & 2.608 & 8.3668 &     & 2.608 & 8.3668 \\
    \multicolumn{1}{l}{9} & 3.45 & \multicolumn{1}{l}{PU} & LinStatic & 2.608 & 7.0812 &     & 2.608 & 7.0812 \\
    \multicolumn{1}{l}{9} & 3.94286 & \multicolumn{1}{l}{PU} & LinStatic & 2.608 & 5.7956 &     & 2.608 & 5.7956 \\
    \multicolumn{1}{l}{9} & 4.43571 & \multicolumn{1}{l}{PU} & LinStatic & 2.608 & 4.51 &     & 2.608 & 4.51 \\
    \multicolumn{1}{l}{9} & 4.92857 & \multicolumn{1}{l}{PU} & LinStatic & 2.608 & 3.2244 &     & 2.608 & 3.2244 \\
    \multicolumn{1}{l}{9} & 5.42143 & \multicolumn{1}{l}{PU} & LinStatic & 2.608 & 1.9388 &     & 2.608 & 1.9388 \\
    \multicolumn{1}{l}{9} & 5.91429 & \multicolumn{1}{l}{PU} & LinStatic & 2.608 & 0.6532 &     & 2.608 & 0.6532 \\
    \multicolumn{1}{l}{9} & 6.40714 & \multicolumn{1}{l}{PU} & LinStatic & 2.608 & -0.6324 &     & 2.608 & -0.6324 \\
    \multicolumn{1}{l}{9} & 6.9 & \multicolumn{1}{l}{PU} & LinStatic & 2.608 & -1.918 &     & 2.608 & -1.918 \\
        &     &     &     &     &     &     &     &  \\
        &     &     & Max & 22.577 & 51.5467 &     &     &  \\
        &     &     & Min & -13.806 & -101.9745 &     &     &  \\
    \end{tabular}%
    }
  \label{tab:addlabel}%
\end{table}%

% % Table generated by Excel2LaTeX from sheet '2'
\begin{table}[H]
  \centering
  \caption{Tabla de resultados del software para viga sin apoyo 2}
  \resizebox{\linewidth}{!}{
    \begin{tabular}{rrrlrr}
    \rowcolor[rgb]{ .2,  .8,  .8} \multicolumn{1}{l}{\textbf{TABLE:  Element Forces - Frames}} &     &     &     &     &  \bigstrut[b]\\
    \hline
    \rowcolor[rgb]{ .8,  1,  1} \multicolumn{1}{|c|}{\textbf{Frame}} & \multicolumn{1}{c|}{\textbf{Station}} & \multicolumn{1}{c|}{\textbf{OutputCase}} & \multicolumn{1}{c|}{\textbf{CaseType}} & \multicolumn{1}{c|}{\textbf{V2}} & \multicolumn{1}{c|}{\textbf{M3}} \bigstrut\\
    \hline
    \rowcolor[rgb]{ .8,  1,  1} \multicolumn{1}{|c|}{Text} & \multicolumn{1}{c|}{m} & \multicolumn{1}{c|}{Text} & \multicolumn{1}{c|}{Text} & \multicolumn{1}{c|}{KN} & \multicolumn{1}{c|}{KN-m} \bigstrut\\
    \hline
    \multicolumn{1}{l}{7} & 0   & \multicolumn{1}{l}{SI} & LinStatic & -68.206 & -115.7989 \bigstrut[t]\\
    \multicolumn{1}{l}{7} & 0.48571 & \multicolumn{1}{l}{SI} & LinStatic & -68.206 & -82.6703 \\
    \multicolumn{1}{l}{7} & 0.97143 & \multicolumn{1}{l}{SI} & LinStatic & -68.206 & -49.5417 \\
    \multicolumn{1}{l}{7} & 1.45714 & \multicolumn{1}{l}{SI} & LinStatic & -68.206 & -16.4131 \\
    \multicolumn{1}{l}{7} & 1.94286 & \multicolumn{1}{l}{SI} & LinStatic & -68.206 & 16.7155 \\
    \multicolumn{1}{l}{7} & 2.42857 & \multicolumn{1}{l}{SI} & LinStatic & -68.206 & 49.8441 \\
    \multicolumn{1}{l}{7} & 2.91429 & \multicolumn{1}{l}{SI} & LinStatic & -68.206 & 82.9727 \\
    \multicolumn{1}{l}{7} & 3.4 & \multicolumn{1}{l}{SI} & LinStatic & -68.206 & 116.1012 \\
    \multicolumn{1}{l}{7} & 3.88571 & \multicolumn{1}{l}{SI} & LinStatic & -68.206 & 149.2298 \\
    \multicolumn{1}{l}{7} & 4.37143 & \multicolumn{1}{l}{SI} & LinStatic & -68.206 & 182.3584 \\
    \multicolumn{1}{l}{7} & 4.85714 & \multicolumn{1}{l}{SI} & LinStatic & -68.206 & 215.487 \\
    \multicolumn{1}{l}{7} & 5.34286 & \multicolumn{1}{l}{SI} & LinStatic & -68.206 & 248.6156 \\
    \multicolumn{1}{l}{7} & 5.82857 & \multicolumn{1}{l}{SI} & LinStatic & -68.206 & 281.7442 \\
    \multicolumn{1}{l}{7} & 6.31429 & \multicolumn{1}{l}{SI} & LinStatic & -68.206 & 314.8728 \\
    \multicolumn{1}{l}{7} & 6.8 & \multicolumn{1}{l}{SI} & LinStatic & -68.206 & 348.0014 \\
    \multicolumn{1}{l}{8} & 0   & \multicolumn{1}{l}{SI} & LinStatic & 86.908 & 354.2352 \\
    \multicolumn{1}{l}{8} & 0.5 & \multicolumn{1}{l}{SI} & LinStatic & 86.908 & 310.7814 \\
    \multicolumn{1}{l}{8} & 1   & \multicolumn{1}{l}{SI} & LinStatic & 86.908 & 267.3276 \\
    \multicolumn{1}{l}{8} & 1.5 & \multicolumn{1}{l}{SI} & LinStatic & 86.908 & 223.8738 \\
    \multicolumn{1}{l}{8} & 2   & \multicolumn{1}{l}{SI} & LinStatic & 86.908 & 180.42 \\
    \multicolumn{1}{l}{8} & 2.5 & \multicolumn{1}{l}{SI} & LinStatic & 86.908 & 136.9662 \\
    \multicolumn{1}{l}{8} & 3   & \multicolumn{1}{l}{SI} & LinStatic & 86.908 & 93.5124 \\
    \multicolumn{1}{l}{8} & 3.5 & \multicolumn{1}{l}{SI} & LinStatic & 86.908 & 50.0586 \\
    \multicolumn{1}{l}{8} & 4   & \multicolumn{1}{l}{SI} & LinStatic & 86.908 & 6.6048 \\
    \multicolumn{1}{l}{8} & 4.5 & \multicolumn{1}{l}{SI} & LinStatic & 86.908 & -36.849 \\
    \multicolumn{1}{l}{8} & 5   & \multicolumn{1}{l}{SI} & LinStatic & 86.908 & -80.3028 \\
    \multicolumn{1}{l}{8} & 5.5 & \multicolumn{1}{l}{SI} & LinStatic & 86.908 & -123.7566 \\
    \multicolumn{1}{l}{8} & 6   & \multicolumn{1}{l}{SI} & LinStatic & 86.908 & -167.2104 \\
    \multicolumn{1}{l}{8} & 6.5 & \multicolumn{1}{l}{SI} & LinStatic & 86.908 & -210.6642 \\
    \multicolumn{1}{l}{8} & 7   & \multicolumn{1}{l}{SI} & LinStatic & 86.908 & -254.118 \\
    \multicolumn{1}{l}{9} & 0   & \multicolumn{1}{l}{SI} & LinStatic & -30.252 & -185.9331 \\
    \multicolumn{1}{l}{9} & 0.49286 & \multicolumn{1}{l}{SI} & LinStatic & -30.252 & -171.0231 \\
    \multicolumn{1}{l}{9} & 0.98571 & \multicolumn{1}{l}{SI} & LinStatic & -30.252 & -156.113 \\
    \multicolumn{1}{l}{9} & 1.47857 & \multicolumn{1}{l}{SI} & LinStatic & -30.252 & -141.203 \\
    \multicolumn{1}{l}{9} & 1.97143 & \multicolumn{1}{l}{SI} & LinStatic & -30.252 & -126.2929 \\
    \multicolumn{1}{l}{9} & 2.46429 & \multicolumn{1}{l}{SI} & LinStatic & -30.252 & -111.3829 \\
    \multicolumn{1}{l}{9} & 2.95714 & \multicolumn{1}{l}{SI} & LinStatic & -30.252 & -96.4729 \\
    \multicolumn{1}{l}{9} & 3.45 & \multicolumn{1}{l}{SI} & LinStatic & -30.252 & -81.5628 \\
    \multicolumn{1}{l}{9} & 3.94286 & \multicolumn{1}{l}{SI} & LinStatic & -30.252 & -66.6528 \\
    \multicolumn{1}{l}{9} & 4.43571 & \multicolumn{1}{l}{SI} & LinStatic & -30.252 & -51.7427 \\
    \multicolumn{1}{l}{9} & 4.92857 & \multicolumn{1}{l}{SI} & LinStatic & -30.252 & -36.8327 \\
    \multicolumn{1}{l}{9} & 5.42143 & \multicolumn{1}{l}{SI} & LinStatic & -30.252 & -21.9226 \\
    \multicolumn{1}{l}{9} & 5.91429 & \multicolumn{1}{l}{SI} & LinStatic & -30.252 & -7.0126 \\
    \multicolumn{1}{l}{9} & 6.40714 & \multicolumn{1}{l}{SI} & LinStatic & -30.252 & 7.8975 \\
    \multicolumn{1}{l}{9} & 6.9 & \multicolumn{1}{l}{SI} & LinStatic & -30.252 & 22.8075 \\
        &     &     &     &     &  \\
        &     &     & Max & 86.908 & 354.2352 \\
        &     &     & Min & -68.206 & -254.118 \\
    \end{tabular}%
  \label{tab:addlabel}%
  }
\end{table}%

% % Table generated by Excel2LaTeX from sheet '3'
\begin{table}[htbp]
  \centering
  \caption{Tabla de resultados del software para viga sin apoyo 3}
  \resizebox{\linewidth}{!}{
    \begin{tabular}{rrrlrr}
    \rowcolor[rgb]{ .2,  .8,  .8} \multicolumn{1}{l}{\textbf{TABLE:  Element Forces - Frames}} &     &     &     &     &  \bigstrut[b]\\
    \hline
    \rowcolor[rgb]{ .8,  1,  1} \multicolumn{1}{|c|}{\textbf{Frame}} & \multicolumn{1}{c|}{\textbf{Station}} & \multicolumn{1}{c|}{\textbf{OutputCase}} & \multicolumn{1}{c|}{\textbf{CaseType}} & \multicolumn{1}{c|}{\textbf{V2}} & \multicolumn{1}{c|}{\textbf{M3}} \bigstrut\\
    \hline
    \rowcolor[rgb]{ .8,  1,  1} \multicolumn{1}{|c|}{Text} & \multicolumn{1}{c|}{m} & \multicolumn{1}{c|}{Text} & \multicolumn{1}{c|}{Text} & \multicolumn{1}{c|}{KN} & \multicolumn{1}{c|}{KN-m} \bigstrut\\
    \hline
    \multicolumn{1}{l}{7} & 0   & \multicolumn{1}{l}{SI} & LinStatic & 36.026 & 26.5515 \bigstrut[t]\\
    \multicolumn{1}{l}{7} & 0.48571 & \multicolumn{1}{l}{SI} & LinStatic & 36.026 & 9.0532 \\
    \multicolumn{1}{l}{7} & 0.97143 & \multicolumn{1}{l}{SI} & LinStatic & 36.026 & -8.4452 \\
    \multicolumn{1}{l}{7} & 1.45714 & \multicolumn{1}{l}{SI} & LinStatic & 36.026 & -25.9435 \\
    \multicolumn{1}{l}{7} & 1.94286 & \multicolumn{1}{l}{SI} & LinStatic & 36.026 & -43.4419 \\
    \multicolumn{1}{l}{7} & 2.42857 & \multicolumn{1}{l}{SI} & LinStatic & 36.026 & -60.9402 \\
    \multicolumn{1}{l}{7} & 2.91429 & \multicolumn{1}{l}{SI} & LinStatic & 36.026 & -78.4385 \\
    \multicolumn{1}{l}{7} & 3.4 & \multicolumn{1}{l}{SI} & LinStatic & 36.026 & -95.9369 \\
    \multicolumn{1}{l}{7} & 3.88571 & \multicolumn{1}{l}{SI} & LinStatic & 36.026 & -113.4352 \\
    \multicolumn{1}{l}{7} & 4.37143 & \multicolumn{1}{l}{SI} & LinStatic & 36.026 & -130.9336 \\
    \multicolumn{1}{l}{7} & 4.85714 & \multicolumn{1}{l}{SI} & LinStatic & 36.026 & -148.4319 \\
    \multicolumn{1}{l}{7} & 5.34286 & \multicolumn{1}{l}{SI} & LinStatic & 36.026 & -165.9303 \\
    \multicolumn{1}{l}{7} & 5.82857 & \multicolumn{1}{l}{SI} & LinStatic & 36.026 & -183.4286 \\
    \multicolumn{1}{l}{7} & 6.31429 & \multicolumn{1}{l}{SI} & LinStatic & 36.026 & -200.927 \\
    \multicolumn{1}{l}{7} & 6.8 & \multicolumn{1}{l}{SI} & LinStatic & 36.026 & -218.4253 \\
    \multicolumn{1}{l}{8} & 0   & \multicolumn{1}{l}{SI} & LinStatic & -101.139 & -297.5059 \\
    \multicolumn{1}{l}{8} & 0.5 & \multicolumn{1}{l}{SI} & LinStatic & -101.139 & -246.9363 \\
    \multicolumn{1}{l}{8} & 1   & \multicolumn{1}{l}{SI} & LinStatic & -101.139 & -196.3668 \\
    \multicolumn{1}{l}{8} & 1.5 & \multicolumn{1}{l}{SI} & LinStatic & -101.139 & -145.7972 \\
    \multicolumn{1}{l}{8} & 2   & \multicolumn{1}{l}{SI} & LinStatic & -101.139 & -95.2276 \\
    \multicolumn{1}{l}{8} & 2.5 & \multicolumn{1}{l}{SI} & LinStatic & -101.139 & -44.6581 \\
    \multicolumn{1}{l}{8} & 3   & \multicolumn{1}{l}{SI} & LinStatic & -101.139 & 5.9115 \\
    \multicolumn{1}{l}{8} & 3.5 & \multicolumn{1}{l}{SI} & LinStatic & -101.139 & 56.4811 \\
    \multicolumn{1}{l}{8} & 4   & \multicolumn{1}{l}{SI} & LinStatic & -101.139 & 107.0507 \\
    \multicolumn{1}{l}{8} & 4.5 & \multicolumn{1}{l}{SI} & LinStatic & -101.139 & 157.6202 \\
    \multicolumn{1}{l}{8} & 5   & \multicolumn{1}{l}{SI} & LinStatic & -101.139 & 208.1898 \\
    \multicolumn{1}{l}{8} & 5.5 & \multicolumn{1}{l}{SI} & LinStatic & -101.139 & 258.7594 \\
    \multicolumn{1}{l}{8} & 6   & \multicolumn{1}{l}{SI} & LinStatic & -101.139 & 309.3289 \\
    \multicolumn{1}{l}{8} & 6.5 & \multicolumn{1}{l}{SI} & LinStatic & -101.139 & 359.8985 \\
    \multicolumn{1}{l}{8} & 7   & \multicolumn{1}{l}{SI} & LinStatic & -101.139 & 410.4681 \\
    \multicolumn{1}{l}{9} & 0   & \multicolumn{1}{l}{SI} & LinStatic & 77.554 & 401.4768 \\
    \multicolumn{1}{l}{9} & 0.49286 & \multicolumn{1}{l}{SI} & LinStatic & 77.554 & 363.2537 \\
    \multicolumn{1}{l}{9} & 0.98571 & \multicolumn{1}{l}{SI} & LinStatic & 77.554 & 325.0306 \\
    \multicolumn{1}{l}{9} & 1.47857 & \multicolumn{1}{l}{SI} & LinStatic & 77.554 & 286.8076 \\
    \multicolumn{1}{l}{9} & 1.97143 & \multicolumn{1}{l}{SI} & LinStatic & 77.554 & 248.5845 \\
    \multicolumn{1}{l}{9} & 2.46429 & \multicolumn{1}{l}{SI} & LinStatic & 77.554 & 210.3614 \\
    \multicolumn{1}{l}{9} & 2.95714 & \multicolumn{1}{l}{SI} & LinStatic & 77.554 & 172.1383 \\
    \multicolumn{1}{l}{9} & 3.45 & \multicolumn{1}{l}{SI} & LinStatic & 77.554 & 133.9152 \\
    \multicolumn{1}{l}{9} & 3.94286 & \multicolumn{1}{l}{SI} & LinStatic & 77.554 & 95.6921 \\
    \multicolumn{1}{l}{9} & 4.43571 & \multicolumn{1}{l}{SI} & LinStatic & 77.554 & 57.4691 \\
    \multicolumn{1}{l}{9} & 4.92857 & \multicolumn{1}{l}{SI} & LinStatic & 77.554 & 19.246 \\
    \multicolumn{1}{l}{9} & 5.42143 & \multicolumn{1}{l}{SI} & LinStatic & 77.554 & -18.9771 \\
    \multicolumn{1}{l}{9} & 5.91429 & \multicolumn{1}{l}{SI} & LinStatic & 77.554 & -57.2002 \\
    \multicolumn{1}{l}{9} & 6.40714 & \multicolumn{1}{l}{SI} & LinStatic & 77.554 & -95.4233 \\
    \multicolumn{1}{l}{9} & 6.9 & \multicolumn{1}{l}{SI} & LinStatic & 77.554 & -133.6464 \\
        &     &     &     &     &  \\
        &     &     & Max & 77.554 & 410.4681 \\
        &     &     & Min & -101.139 & -297.5059 \\
    \end{tabular}%
    }
  \label{tab:addlabel}%
\end{table}%

% % Table generated by Excel2LaTeX from sheet '4'
\begin{table}[htbp]
  \centering
  \caption{Tabla de resultados del software para viga sin apoyo 4}
  \resizebox{\linewidth}{!}{
    \begin{tabular}{rrrlrr}
    \rowcolor[rgb]{ .2,  .8,  .8} \multicolumn{1}{l}{\textbf{TABLE:  Element Forces - Frames}} &     &     &     &     &  \bigstrut[b]\\
    \hline
    \rowcolor[rgb]{ .8,  1,  1} \multicolumn{1}{|c|}{\textbf{Frame}} & \multicolumn{1}{c|}{\textbf{Station}} & \multicolumn{1}{c|}{\textbf{OutputCase}} & \multicolumn{1}{c|}{\textbf{CaseType}} & \multicolumn{1}{c|}{\textbf{V2}} & \multicolumn{1}{c|}{\textbf{M3}} \bigstrut\\
    \hline
    \rowcolor[rgb]{ .8,  1,  1} \multicolumn{1}{|c|}{Text} & \multicolumn{1}{c|}{m} & \multicolumn{1}{c|}{Text} & \multicolumn{1}{c|}{Text} & \multicolumn{1}{c|}{KN} & \multicolumn{1}{c|}{KN-m} \bigstrut\\
    \hline
    \multicolumn{1}{l}{7} & 0   & \multicolumn{1}{l}{SI} & LinStatic & -5.533 & -3.9763 \bigstrut[t]\\
    \multicolumn{1}{l}{7} & 0.48571 & \multicolumn{1}{l}{SI} & LinStatic & -5.533 & -1.289 \\
    \multicolumn{1}{l}{7} & 0.97143 & \multicolumn{1}{l}{SI} & LinStatic & -5.533 & 1.3983 \\
    \multicolumn{1}{l}{7} & 1.45714 & \multicolumn{1}{l}{SI} & LinStatic & -5.533 & 4.0856 \\
    \multicolumn{1}{l}{7} & 1.94286 & \multicolumn{1}{l}{SI} & LinStatic & -5.533 & 6.7729 \\
    \multicolumn{1}{l}{7} & 2.42857 & \multicolumn{1}{l}{SI} & LinStatic & -5.533 & 9.4602 \\
    \multicolumn{1}{l}{7} & 2.91429 & \multicolumn{1}{l}{SI} & LinStatic & -5.533 & 12.1475 \\
    \multicolumn{1}{l}{7} & 3.4 & \multicolumn{1}{l}{SI} & LinStatic & -5.533 & 14.8348 \\
    \multicolumn{1}{l}{7} & 3.88571 & \multicolumn{1}{l}{SI} & LinStatic & -5.533 & 17.5221 \\
    \multicolumn{1}{l}{7} & 4.37143 & \multicolumn{1}{l}{SI} & LinStatic & -5.533 & 20.2094 \\
    \multicolumn{1}{l}{7} & 4.85714 & \multicolumn{1}{l}{SI} & LinStatic & -5.533 & 22.8967 \\
    \multicolumn{1}{l}{7} & 5.34286 & \multicolumn{1}{l}{SI} & LinStatic & -5.533 & 25.584 \\
    \multicolumn{1}{l}{7} & 5.82857 & \multicolumn{1}{l}{SI} & LinStatic & -5.533 & 28.2713 \\
    \multicolumn{1}{l}{7} & 6.31429 & \multicolumn{1}{l}{SI} & LinStatic & -5.533 & 30.9586 \\
    \multicolumn{1}{l}{7} & 6.8 & \multicolumn{1}{l}{SI} & LinStatic & -5.533 & 33.6459 \\
    \multicolumn{1}{l}{8} & 0   & \multicolumn{1}{l}{SI} & LinStatic & 28.695 & 45.4615 \\
    \multicolumn{1}{l}{8} & 0.5 & \multicolumn{1}{l}{SI} & LinStatic & 28.695 & 31.1139 \\
    \multicolumn{1}{l}{8} & 1   & \multicolumn{1}{l}{SI} & LinStatic & 28.695 & 16.7664 \\
    \multicolumn{1}{l}{8} & 1.5 & \multicolumn{1}{l}{SI} & LinStatic & 28.695 & 2.4188 \\
    \multicolumn{1}{l}{8} & 2   & \multicolumn{1}{l}{SI} & LinStatic & 28.695 & -11.9288 \\
    \multicolumn{1}{l}{8} & 2.5 & \multicolumn{1}{l}{SI} & LinStatic & 28.695 & -26.2763 \\
    \multicolumn{1}{l}{8} & 3   & \multicolumn{1}{l}{SI} & LinStatic & 28.695 & -40.6239 \\
    \multicolumn{1}{l}{8} & 3.5 & \multicolumn{1}{l}{SI} & LinStatic & 28.695 & -54.9715 \\
    \multicolumn{1}{l}{8} & 4   & \multicolumn{1}{l}{SI} & LinStatic & 28.695 & -69.319 \\
    \multicolumn{1}{l}{8} & 4.5 & \multicolumn{1}{l}{SI} & LinStatic & 28.695 & -83.6666 \\
    \multicolumn{1}{l}{8} & 5   & \multicolumn{1}{l}{SI} & LinStatic & 28.695 & -98.0142 \\
    \multicolumn{1}{l}{8} & 5.5 & \multicolumn{1}{l}{SI} & LinStatic & 28.695 & -112.3618 \\
    \multicolumn{1}{l}{8} & 6   & \multicolumn{1}{l}{SI} & LinStatic & 28.695 & -126.7093 \\
    \multicolumn{1}{l}{8} & 6.5 & \multicolumn{1}{l}{SI} & LinStatic & 28.695 & -141.0569 \\
    \multicolumn{1}{l}{8} & 7   & \multicolumn{1}{l}{SI} & LinStatic & 28.695 & -155.4045 \\
    \multicolumn{1}{l}{9} & 0   & \multicolumn{1}{l}{SI} & LinStatic & -46.253 & -211.727 \\
    \multicolumn{1}{l}{9} & 0.49286 & \multicolumn{1}{l}{SI} & LinStatic & -46.253 & -188.9308 \\
    \multicolumn{1}{l}{9} & 0.98571 & \multicolumn{1}{l}{SI} & LinStatic & -46.253 & -166.1347 \\
    \multicolumn{1}{l}{9} & 1.47857 & \multicolumn{1}{l}{SI} & LinStatic & -46.253 & -143.3386 \\
    \multicolumn{1}{l}{9} & 1.97143 & \multicolumn{1}{l}{SI} & LinStatic & -46.253 & -120.5424 \\
    \multicolumn{1}{l}{9} & 2.46429 & \multicolumn{1}{l}{SI} & LinStatic & -46.253 & -97.7463 \\
    \multicolumn{1}{l}{9} & 2.95714 & \multicolumn{1}{l}{SI} & LinStatic & -46.253 & -74.9501 \\
    \multicolumn{1}{l}{9} & 3.45 & \multicolumn{1}{l}{SI} & LinStatic & -46.253 & -52.154 \\
    \multicolumn{1}{l}{9} & 3.94286 & \multicolumn{1}{l}{SI} & LinStatic & -46.253 & -29.3578 \\
    \multicolumn{1}{l}{9} & 4.43571 & \multicolumn{1}{l}{SI} & LinStatic & -46.253 & -6.5617 \\
    \multicolumn{1}{l}{9} & 4.92857 & \multicolumn{1}{l}{SI} & LinStatic & -46.253 & 16.2345 \\
    \multicolumn{1}{l}{9} & 5.42143 & \multicolumn{1}{l}{SI} & LinStatic & -46.253 & 39.0306 \\
    \multicolumn{1}{l}{9} & 5.91429 & \multicolumn{1}{l}{SI} & LinStatic & -46.253 & 61.8268 \\
    \multicolumn{1}{l}{9} & 6.40714 & \multicolumn{1}{l}{SI} & LinStatic & -46.253 & 84.6229 \\
    \multicolumn{1}{l}{9} & 6.9 & \multicolumn{1}{l}{SI} & LinStatic & -46.253 & 107.4191 \\
        &     &     &     &     &  \\
        &     &     & Max & 28.695 & 107.4191 \\
        &     &     & Min & -46.253 & -211.727 \\
    \end{tabular}%
  \label{tab:addlabel}%
  }
\end{table}%

\end{document}