% Table generated by Excel2LaTeX from sheet 'Losa ascensor'
\begin{table}[H]
  \centering
    \begin{tabular}{|c|c|}
    \hline
    \rowcolor[rgb]{ .2,  .247,  .31} \multicolumn{2}{|c|}{\textcolor[rgb]{ 1,  1,  1}{Retracción de fraguado}} \bigstrut\\
    \hline
    \rowcolor[rgb]{ .2,  .247,  .31} \textcolor[rgb]{ 1,  1,  1}{$\rho_{t}$} & \cellcolor[rgb]{ 1,  1,  1}0.0018 \bigstrut\\
    \hline
    \rowcolor[rgb]{ .2,  .247,  .31} \textcolor[rgb]{ 1,  1,  1}{Ast [mm²]} & \cellcolor[rgb]{ 1,  1,  1}360 \bigstrut\\
\cline{2-2}    \multicolumn{1}{r}{} & \multicolumn{1}{r}{} \bigstrut\\
    \hline
    \rowcolor[rgb]{ .2,  .247,  .31} \textcolor[rgb]{ 1,  1,  1}{Diámetro} & \cellcolor[rgb]{ 1,  1,  1}3 \bigstrut\\
    \hline
    \rowcolor[rgb]{ .2,  .247,  .31} \textcolor[rgb]{ 1,  1,  1}{Área} & \cellcolor[rgb]{ 1,  1,  1}71 \bigstrut\\
    \hline
    \rowcolor[rgb]{ .2,  .247,  .31} \textcolor[rgb]{ 1,  1,  1}{Separación} & \cellcolor[rgb]{ 1,  1,  1}0.20 \bigstrut\\
\cline{2-2}    \end{tabular}%
  \caption{Retracción de fraguado para la losa}
  \label{tab:R fraguado}%
\end{table}%
