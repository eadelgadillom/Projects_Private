% Table generated by Excel2LaTeX from sheet 'Losa ascensor'
\begin{table}[H]
  \centering
    \begin{tabular}{|c|c|c}
    \rowcolor[rgb]{ .2,  .247,  .31} \textcolor[rgb]{ 1,  1,  1}{Elemento} & \multicolumn{1}{c|}{\textcolor[rgb]{ 1,  1,  1}{Operación }} & \multicolumn{1}{c|}{\textcolor[rgb]{ 1,  1,  1}{Resultado}} \bigstrut[b]\\
    \hline
    \rowcolor[rgb]{ .2,  .247,  .31} \textcolor[rgb]{ 1,  1,  1}{M+u} & \cellcolor[rgb]{ 1,  1,  1}{$\tfrac{39.02\cdot 1.75^{2}}{8}$} & \multicolumn{1}{r|}{\cellcolor[rgb]{ 1,  1,  1}14.937} \bigstrut\\
    \hline
    \rowcolor[rgb]{ .2,  .247,  .31} \textcolor[rgb]{ 1,  1,  1}{d} & \cellcolor[rgb]{ 1,  1,  1}{200-20-6.3} & \multicolumn{1}{r|}{\cellcolor[rgb]{ 1,  1,  1}173.7} \bigstrut\\
\cline{2-3}    \multicolumn{1}{r}{} & \multicolumn{1}{r}{} &  \bigstrut\\
\cline{1-2}    \rowcolor[rgb]{ .2,  .247,  .31} \multicolumn{2}{|c|}{\textcolor[rgb]{ 1,  1,  1}{Momento positivo}} & \cellcolor[rgb]{ 1,  1,  1} \bigstrut\\
\cline{1-2}    \rowcolor[rgb]{ .2,  .247,  .31} \textcolor[rgb]{ 1,  1,  1}{$\rho$} & \cellcolor[rgb]{ 1,  1,  1}0.001324138 & \cellcolor[rgb]{ 1,  1,  1} \bigstrut\\
\cline{1-2}    \rowcolor[rgb]{ .2,  .247,  .31} \textcolor[rgb]{ 1,  1,  1}{As} & \cellcolor[rgb]{ 1,  1,  1}230 & \cellcolor[rgb]{ 1,  1,  1} \bigstrut\\
\cline{2-2}    \multicolumn{1}{r}{} & \multicolumn{1}{r}{} &  \bigstrut\\
\cline{1-2}    \rowcolor[rgb]{ .2,  .247,  .31} \textcolor[rgb]{ 1,  1,  1}{Diámetro} & \cellcolor[rgb]{ 1,  1,  1}3 & \cellcolor[rgb]{ 1,  1,  1} \bigstrut\\
\cline{1-2}    \rowcolor[rgb]{ .2,  .247,  .31} \textcolor[rgb]{ 1,  1,  1}{Área} & \cellcolor[rgb]{ 1,  1,  1}71 & \cellcolor[rgb]{ 1,  1,  1} \bigstrut\\
\cline{1-2}    \rowcolor[rgb]{ .2,  .247,  .31} \textcolor[rgb]{ 1,  1,  1}{Separación} & \cellcolor[rgb]{ 1,  1,  1}0.31 & \cellcolor[rgb]{ 1,  1,  1} \bigstrut\\
\cline{1-2}    \end{tabular}%
  \caption{Diseño a flexión de la losa macisa}
  \label{tab:D flexión losa macisa}%
\end{table}%
